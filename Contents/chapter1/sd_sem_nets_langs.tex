\begin{SCn}

\scnsectionheader{Предметная область и онтология семантических сетей, семантических языков и семантических моделей баз знаний}

\scnstartsubstruct

\scnheader{Предметная область семантических сетей, семантических языков и семантических моделей баз знаний}
\scnsdmainclasssingle{***}
\scnsdclass{смысловое представление информации}
\scnsdrelation{***}

\scnheader{смысловое представление информации}
\scnexplanation{Объективным ориентиром для \textbf{унификации представления информации} в памяти компьютерных систем и ключом к решению многих проблем эволюции компьютерных систем и технологий является \textbf{формализация смысла представляемой информации}.

Уточнение принципов \textbf{смыслового представления информации} основано, во-первых, на четком противопоставление \textbf{внутреннего языка компьютерной системы}, используемого для хранения информации в памяти компьютера, и \textbf{внешних языков компьютерной системы}, используемых для общения (обмена сообщениями) компьютерной системы с пользователями и другими компьютерными системами (смысловое представление используется исключительно для \textbf{внутреннего представления} информации в памяти компьютерной системы), и, во-вторых, на максимально возможном упрощении синтаксиса внутреннего языка компьютерной системы при обеспечении универсальности  путем исключения из такого внутреннего универсального языка средств, обеспечивающих коммуникационную функцию языка (т. е. обмен сообщениями).

Так, например, для внутреннего языка компьютерной системы излишними являются такие коммуникационные средства языка, как союзы, предлоги, разделители, ограничители, склонения, спряжения и другие.

Внешние языки компьютерной системы могут быть как близки ее внутреннему языку, так и весьма далеки от него (как, например, естественные языки).

\textbf{Смысл} – это \textbf{абстрактная} знаковая конструкция, принадлежащая внутреннему языку компьютерной системы, являющаяся \textbf{инвариантом} максимального класса семантически эквивалентных знаковых конструкций (текстов), принадлежащих самым разным языкам, и удовлетворяющая следующим требованиям:
\begin{scnitemize}
    \item \textbf{универсальность} - возможность представления любой информации;
    \item \textbf{отсутствие синонимии знаков} (многократного вхождения знаков с одинаковыми денотатами);
    \item \textbf{отсутствие дублирования информации} в виде семантически эквивалентных текстов (не путать с логической эквивалентностью);
    \item \textbf{отсутствие омонимичных знаков} (в том числе местоимений);
    \item \textbf{отсутствие у знаков внутренней структуры} (атомарный характер знаков);
    \item \textbf{отсутствие склонений, спряжений} (как следствие отсутствия у знаков внутренней структуры);
    \item \textbf{отсутствие фрагментов} знаковой конструкции, \textbf{не являющихся знаками} (разделителей, ограничителей, и т.д.);
    \item \textbf{выделение знаков связей}, компонентами которых могут быть любые знаки, с которыми знаки связей связываются синтаксически задаваемыми отношениями инцидентности.
\end{scnitemize}

Следствием указанных принципов смыслового представления информации в памяти компьютерной системы является то, что знаки сущностей, входящие в смысловое представление информации, \textbf{не являются именами} (терминами) и, следовательно, не привязаны ни к какому естественному языку и не зависят от субъективных терминотворческих пристрастий различных авторов. Это значит, что при коллективной разработке смыслового представления каких-либо информационных ресурсов терминологические споры исключены.

Следствием указанных принципов смыслового представления информации  является также то, что эти принципы приводят к нелинейным знаковым конструкциям (к графовым структурам), что усложняет реализацию памяти компьютерных систем, но существенно упрощает ее логическую организацию (в частности, ассоциативный доступ).

Нелинейность смыслового представления информации обусловлена тем, что: 
\begin{scnitemize}
    \item каждая описываемая сущность, т.е. сущность, имеющая соответствующий ей знак, может иметь неограниченное число связей с другими описываемыми сущностями;
    \item каждая описываемая сущность в смысловом представлении имеет единственный знак, т.к. синонимия знаков здесь запрещена;
    \item все связи между описываемыми сущностями описываются (отражаются, моделируются) связями между знаками этих описываемых сущностей.
\end{scnitemize}

Суть \textbf{универсального смыслового представления информации} можно сформулировать в виде следующих положений:
\begin{scnitemize}
    \item Смысловая знаковая конструкция трактуется как множество знаков, взаимно-однозначно обозначающих различные сущности (денотаты этих знаков) и множество связей между этими знаками;
    \item Каждая связь между знаками трактуется, с одной стороны, как множество знаков, связываемых этой связью, а, с другой стороны, как описание (отражение, модель) соответствующей связи, которая связывает денотаты указанных знаков или денотаты одних знаков непосредственно с другими знаками, или сами эти знаки. Примером первого вида связи между знаками является связь между знаками материальных сущностей, одна из которых является частью другой. Примером второго вида связи между знаками является связь между знаком множества знаков и одним из знаком, принадлежащих этому множеству, а также связь между знаком и знаком файла, являющегося электронным отражением структуры представления указанного знака во внешних знаковых конструкциях. Примерами третьего вида связи между знаками является связь между синонимичными знаками;
    \item Денотатами знаков могут быть (1) не только конкретные (константные, фиксированные), но и произвольные (переменные, нефиксированные) сущности, \scnqq{пробегающие} различные множества знаков (возможных значений), 
    (2) не только реальные (материальные), но и абстрактные сущности (например, числа, точки различных абстрактных пространств), 
    (3) не только \scnqq{внешние}, но и \scnqq{внутренние} сущности, являющиеся множествами знаков, входящих в состав той же самой знаковой конструкции.
\end{scnitemize}

Ключевым свойством языка смыслового представления информации является однозначность представления информации в памяти каждой компьютерной системы, т. е. отсутствие семантически эквивалентных знаковых конструкций, принадлежащих смысловому языку и хранимых в одной смысловой памяти. При этом логическая эквивалентность таких знаковых конструкций допускается и используются, например, для компактного представления некоторых знаний, хранимых в смысловой памяти.

Тем не менее, логической эквивалентностью хранимых в памяти знаковых конструкций увлекаться не следует, т.к. \textbf{логически эквивалентные} знаковые конструкции -- это представление одного и того же знания, но с помощью \textbf{разных наборов понятий}. В отличие от этого \textbf{семантически эквивалентные} знаковые конструкции -- это представление одного и того же знания с помощью одних и тех же понятий. Очевидно, что многообразие возможных вариантов представления одних и тех же знаний в памяти компьютерной системы существенно усложняет решение задач. Поэтому, полностью исключив \textbf{семантическую эквивалентность} в смысловой памяти, необходимо стремиться к минимизации \textbf{логической эквивалентности}. Для этого необходимо грамотное построение системы используемых понятий в виде иерархической системы формальных онтологий ~\cite{Davydenko2018}.

Важным этапом создания универсального формального способа смыслового кодирования знаний был разработанный В.В. Мартыновым Универсальный Семантический Код (УСК)~\cite{Martynov}.

В качестве \textbf{стандарта} универсального смыслового представления информации \textbf{в памяти компьютерных систем} нами предложен \textbf{\textit{SC-код}} (Semantic Computer Code). В отличие от УСК В.В. Мартынова он, во-первых, носит нелинейный характер и, во-вторых, специально ориентирован на кодирование информации в памяти компьютеров нового поколения, ориентированных на разработку семантически совместимых интеллектуальных систем и названных нами \textbf{семантическими ассоциативными компьютерами}. Более подробно это понятие (\textbf{\textit{SC-код}}) рассмотрено в разделе \textit{Предметная область и онтология внутреннего языка ostis-систем -- SC-кода}. Таким образом, основным лейтмотивом предлагаемого нами смыслового представления информации является ориентация на формальную модель памяти нефоннеймановского компьютера, предназначенного для реализации интеллектуальных систем, использующих смысловое представление информации. Особенностями такого представления являются следующие:
\begin{scnitemize}
    \item ассоциативность;
    \item вся информация заключена в конфигурации связей, т.е. переработка информации сводится к реконфигурации связей (к графодинамическим процессам);
    \item прозрачная семантическая интерпретируемость и, как следствие, семантическая совместимость.
\end{scnitemize}

Неявная привязка к фоннеймановским компьютерам присутствует во всех известных моделях представления знаний. Одним из примеров такой зависимости, является, например, обязательность именования описываемых объектов.}

\scnheader{смысловое представление информации}
\scnadvantages{Почему целесообразен переход к \textit{смысловому представлению информации} в памяти \textit{компьютерной системы}: 
\begin{scnitemize}
    \item \textit{смысловое представление информации} есть \uline{объективный}, не зависящий от субъективизма и многообразия синтаксических решений, способ представления информации;
    \item в рамках смыслового представления существенно упрощается процедура интеграции знаний и погружения новых знаний в \textit{базу знаний};
    \item cущественно упрощается процедура приведения различного вида знаний к общему виду (к согласованной системе используемых понятий);
    \item cущественно упрощается процедура интеграции различных \textit{~решателей задач~} и целых \textit{компьютерных систем}; 
    \item существенно упрощается автоматизация перманентного процесса поддержки семантической совместимости (согласованности понятий и онтологий) для \textit{компьютерных систем} в условиях их постоянного совершенствования;
    \item в рамках смыслового представления информации достаточно легко осуществляется переход от информационных конструкций к информационным метаконструкциям путем введения узлов семантической сети, обозначающих информационные конструкции, а также дуг, связывающих эти узлы со всеми элементами обозначаемой им информационной конструкции;
    \item на основе \textit{стандарта смыслового представления информации} существенно упрощается интеграция различных дисциплин в области искусственного интеллекта, т.е. построение общей формальной теории интеллектуальных компьютерных систем, так как для построения общей формальной модели интеллектуальных компьютерных систем необходим базовый язык, в рамках которого можно было бы легко переходить от информации (от знаний) к \textbf{метаинформации} (к метазнаниям, к спецификациям исходных знаний).  Это подтверждается тем, что:
    \begin{scnitemizeii}
        \item подавляющее число понятий искусственного интеллекта носит метаязыковой характер;
        \item формальное смысловое уточнение почти каждого понятия искусственного интеллекта требует предшествующего формального уточнения соответствующего языка-объекта. Так, например, как можно строго говорить о языке онтологий (т.е. языке спецификации предметных областей), не уточнив язык представления самих этих предметных областей. Как можно строго говорить о языке описания способов обработки информации, не уточнив язык представления самой этой обрабатываемой информации.
    \end{scnitemizeii}
\end{scnitemize}}

\scnendstruct

\end{SCn}