\scsection[\scneditor{Банцевич К.А.}\protect\scnmonographychapter{Глава 2.3. Структура баз знаний интеллектуальных компьютерных систем нового поколения: иерархическая система предметных областей и онтологий. Онтологии верхнего уровня. Формализация понятий семантической окрестности, предметной области и онтологии в интеллектуальных компьютерных системах нового поколения}]{Предметная область и онтология знаний и баз знаний ostis-систем}
\label{sd_knowledge}
\begin{SCn}

\scnsectionheader{\currentname}

\scnstartsubstruct

\scnrelfromlist{дочерний раздел}{Предметная область и онтология множеств
    \scnaddlevel{1}
    \scnidtf{Предметная область и онтология \textit{знаний о множествах}}
        \scnaddlevel{1}
        \scnnote{\textit{знания о множествах} являются \uline{частным видом} \textit{знаний} и, следовательно, общие свойства сущностей, описываемых знаниями, могут наследоваться \textit{Предметной областью и онтологией множеств}}
        \scnaddlevel{-1}
    \scnaddlevel{-1}
;Предметная область и онтология связок и отношений
;Предметная область и онтология параметров, величин и шкал
;Предметная область и онтология чисел и числовых структур
;Предметная область и онтология структур
;Предметная область и онтология темпоральных сущностей
;Предметная область и онтология темпоральных сущностей баз знаний ostis-систем
;Предметная область и онтология семантических окрестностей
;Предметная область и онтология предметных областей
;Предметная область и онтология онтологий
;Предметная область и онтология логических формул, высказываний и формальных теорий
;Предметная область и онтология внешних информационных конструкций и файлов ostis-систем
;Глобальная предметная область действий и задач и соответствующая ей онтология методов и технологий}

\scnheader{Предметная область знаний и баз знаний ostis-систем}
\scniselement{предметная область}
\scnsdmainclasssingle{знание}
\scnhaselementlist{исследуемый класс классов}{вид знаний;отношение, заданное на множестве знаний}

\scnheader{знание}
\scnidtf{синтаксически корректная (для соответствующего языка) и семантически целостная информационная конструкция}
\scnsubset{информационная конструкция}
    \scnaddlevel{1}
    \scniselementrole{класс объектов исследования}{\nameref{intro_lang}}
    \scnaddlevel{-1}
\scnrelfrom{покрытие}{вид знаний
    \scnidtf{Множество \uline{всевозможных} видов знаний}
    \scnnote{Тот факт, что семейство \textit{видов знаний} является \textit{покрытием} Множества всевозможных \textit{знаний}, означает то, что каждое \textit{знание} принадлежит по крайней мере одному выделенному нами \textit{виду знаний}}}

    
    
\scnheader{вид знаний}
\scnhaselement{спецификация}
    \scnaddlevel{1}
    \scnidtf{описание заданной сущности}
    \scnsuperset{спецификация материальной сущности}
    \scnsuperset{спецификация обратной сущности, не являющейся множеством}
        \scnaddlevel{1}
        \scnsuperset{спецификация геометрической точки}
        \scnsuperset{спецификация числа}
        \scnaddlevel{-1}
    \scnsuperset{спецификация множества}
        \scnaddlevel{1}
        \scnsuperset{спецификация связи}
        \scnsuperset{спецификация структуры}
        \scnsuperset{спецификация класса}
            \scnaddlevel{1}
            \scnsuperset{спецификация класса сущностей, не являющихся множествами}
            \scnsuperset{спецификация отношения}
                \scnaddlevel{1}
                \scnidtf{спецификация класса связей (связок)}
                \scnaddlevel{-1}
            \scnsuperset{спецификация класса классов}
                \scnaddlevel{1}
                \scnsuperset{спецификация параметра}
                \scnaddlevel{-1}
            \scnsuperset{спецификация класса структур}
            \scnsuperset{спецификация понятий}
                \scnaddlevel{1}
                \scnsuperset{пояснение}
                \scnsuperset{определение}
                \scnsuperset{утверждение}
                    \scnaddlevel{1}
                    \scnidtf{утверждение, описывающее свойства экземпляров (элементов) специфицируемого понятия}
                    \scnidtf{закономерность}
                    \scnaddlevel{-1}
                \scnaddlevel{-1}
            \scnaddlevel{-1}
        \scnaddlevel{-1}
    \scnsuperset{семантическая окрестность}
    \scnsuperset{однозначная спецификация}
    \scnsuperset{сравнительный анализ}
    \scnsuperset{достоинства}
    \scnsuperset{недостатки}
    \scnsuperset{структура специфицируемой сущности}
    \scnsuperset{принципы, лежащие в основе}
    \scnsuperset{обоснование предлагаемого решения}
        \scnaddlevel{1}
        \scnidtf{аргументация предлагаемого решения}
        \scnaddlevel{-1}
    \scnaddlevel{-1}

\scnhaselement{сравнение}

\scnhaselement{высказывание}
    \scnaddlevel{1}
    \scnsuperset{фактографическое высказывание}
    \scnsuperset{закономерность}
    \scnaddlevel{-1}

\scnhaselement{формальная теория}

\scnhaselement{предметная область}

\scnhaselement{предметная область и онтология
    \scnaddlevel{1}
    \scnidtf{предметная область и её онтология}
    \scnidtf{предметная область и соответствующая ей объединенная онтология}
    \scnaddlevel{-1}} 
 
\scnhaselement{метазнание}
    \scnaddlevel{1}
    \scnidtf{спецификация знания}
    \scnsuperset{аннотация}
    \scnsuperset{введение}
    \scnsuperset{предисловие}
    \scnsuperset{заключение}
    \scnsuperset{онтология}
        \scnaddlevel{1}
        \scnsuperset{онтология предметной области}
            \scnaddlevel{1}
            \scnsuperset{структурная онтология предметной области}
            \scnsuperset{теоретико-множественная онтология предметной области}
            \scnsuperset{логическая онтология предметной области}
            \scnsuperset{терминологическая онтология предметной области}
            \scnsuperset{объединенная онтология предметной области}
            \scnaddlevel{-1}
        \scnaddlevel{-1}
    \scnaddlevel{-1}

\scnhaselement{задача}
    \scnaddlevel{1}
    \scnidtf{спецификация действия}
    \scnaddlevel{-1}
\scnhaselement{план}

\scnhaselement{протокол}

\scnhaselement{результативная часть протокола}

\scnhaselement{метод}

\scnhaselement{технология}

\scnhaselement{история использования предметной области и её онтологии по решению информационных задач}
\scnhaselement{история использования предметной области и её онтологии по решению задач во внешней среде}
\scnhaselement{история эволюции предметной области и её онтологии}

\scnhaselement{база знаний}
    \scnaddlevel{1}
    \scnidtf{совокупность знаний, хранимых в памяти интеллектуальной компьютерной системы и \uline{достаточных} для того, чтобы указанная система удовлетворяла соответствующим предъявляемым к ней требованиям (в частности, чтобы она имела соответствующий уровень интеллекта)}
    \scnidtf{систематизированная совокупность знаний, хранимая в памяти интеллектуальной компьютерной системы и достаточная для обеспечения целенаправленного (целесообразного, адекватного) функционирования (поведения) этой системы как в своей внешней среде, так и в своей внутренней среде (в собственной базе знаний)}
    \scnrelfromset{обобщенная декомпозиция}{согласованная часть базы знаний
        \scnaddlevel{1}
        \scnidtf{часть базы знаний, признанная коллективом авторов на текущий момент}
        \scnaddlevel{-1}
    ;история эксплуатации базы знаний;история эволюции базы знаний;план эволюции базы знаний
        \scnaddlevel{1}
        \scnidtf{система специфицированных и согласованных действий авторов базы знаний, направленных на повышение её качества}
        \scnaddlevel{-1}}
    \scnnote{Основным факторами, определяющими качество интеллектуальной компьютерной системы, являются:
    \begin{scnitemize}
        \item качественная структуризация (систематизация) и \uline{стратификация} базы знаний интеллектуальной компьютерной системы, а также
        \item систематизация и стратификация \uline{деятельности}, которая осуществляется интеллектуальной компьютерной системой и спецификация которой является важнейшей частью базы знаний этой системы (Смотрите Раздел \textit{Глобальная предметная область действий и задач и соответствующая ей онтология методов и технологий}).
    \end{scnitemize}}
    \scnaddlevel{-1}
\scnnote{Даже небольшой перечень \textit{видов знаний} свидетельствует об огромном многообразии \textit{видов знаний}}

\scnheader{знание}
\scnsubdividing{декларативное знание
    \scnaddlevel{1}
    \scnidtf{\textit{знание}, имеющее \uline{только} \textit{денотационную семантику}, которая представляется в виде семантической \textit{спецификации} системы \textit{понятий}, используемых в этом \textit{знании}}
    \scnaddlevel{-1}
;процедурное знание
    \scnaddlevel{1}
    \scnidtf{\textit{знание}, имеющее не только \textit{денотационную семантику}, но и \textit{операционную семантику}, которая представляется в виде семейства \textit{спецификаций агентов}, осуществляющих интерпретацию \textit{процедурного знания}, направленную на решение некоторой инициированной \textit{задачи}}
    \scnidtf{функционально интерпретируемое знание, обеспечивающее решение либо конкретной задачи, либо некоторого множества инициируемых задач}
    \scnsuperset{задача}
        \scnaddlevel{1}
        \scnidtf{формулировка конкретной задачи}
        \scnsuperset{декларативная формулировка задачи}
        \scnsuperset{процедурная формулировка задачи}
        \scnaddlevel{-1}
    \scnsuperset{план}
        \scnaddlevel{1}
        \scnidtf{план решения конкретной задачи}
        \scnidtf{контекст конкретной задачи, предоставляющий всю информацию для решения всех подзадач для указанной конкретной задачи}
        \scnidtf{описание системы подзадач некоторой задачи}
        \scnaddlevel{-1}
    \scnsuperset{метод}
        \scnaddlevel{1}
        \scnidtf{обобщенное описание плана решения любой задачи из некоторого заданного класса задач}
        \scnaddlevel{-1}
    \scnsuperset{навык}
        \scnaddlevel{1}
        \scnidtf{метод, детализированный до уровня элементарных подзадач}
        \scnaddlevel{-1}
    \scnaddlevel{-1}}
    
\scnheader{отношение, заданное на множестве знаний}
\scnhaselement{дочернее знание*}
    \scnaddlevel{1}
    \scnidtf{знание, которое от \scnqq{материнского} знания наследует все описанные там свойства объектов исследования}
    \scnnote{Факт наследования свойств описываемых объектов от \scnqq{материнского} знания подчеркивается использованием прилагательного \scnqq{дочернее} в sc-идентификаторе данного отношения, заданного на множестве знаний}
    \scnsuperset{дочерний раздел*}
        \scnaddlevel{1}
        \scnidtf{частный раздел*}
        \scnaddlevel{-1}
    \scnsuperset{дочерняя предметная область и онтология*}
    \scnaddlevel{-1}
\scnhaselement{спецификация*}
    \scnaddlevel{1}
    \scnidtf{быть знанием, которое является спецификацией (описанием) заданной сущности}
    \scnnote{специфицируемой сущностью может быть сущность любого вида, в том числе, и другое знание}
    \scnaddlevel{-1}
\scnhaselement{онтология*}
    \scnaddlevel{1}
    \scnidtf{быть семантической спецификацией заданного знания*}
    \scnaddlevel{-1}
\scnhaselement{семантическая эквивалентность*}
\scnhaselement{следовательно*}
    \scnaddlevel{1}
    \scnidtf{логическое следствие*}
    \scnaddlevel{-1}
\scnhaselement{логическая эквивалентность*}   
    
\bigskip    
\scnendstruct \scnendcurrentsectioncomment

\end{SCn}

\scsubsection[\scnidtf{Предметная область и онтология знаний о множествах}\protect\scnmonographychapter{Глава 2.4. Формальные онтологии базовых классов сущностей - множеств, связей, отношений, параметров, величин, чисел, структур, темпоральных сущностей}]{Предметная область и онтология множеств}
\label{sd_sets}
\input{Contents/chapter2/sd_ostis_sys_models/sd_kb/sd_sets.tex}

\scsubsection[\scnmonographychapter{Глава 2.4. Формальные онтологии базовых классов сущностей - множеств, связей, отношений, параметров, величин, чисел, структур, темпоральных сущностей}]{Предметная область и онтология связок и отношений}
\label{sd_rels}
\input{Contents/chapter2/sd_ostis_sys_models/sd_kb/sd_relations.tex}

\scsubsection[\scnmonographychapter{Глава 2.4. Формальные онтологии базовых классов сущностей - множеств, связей, отношений, параметров, величин, чисел, структур, темпоральных сущностей}]{Предметная область и онтология параметров, величин и шкал}
\label{sd_params}
\begin{SCn}

\scnsectionheader{\currentname}

\scnstartsubstruct

\scnheader{Предметная область параметров, величин и шкал}
\scnidtf{Предметная область параметров и классов эквивалентности, являющихся их элементами (значениями, величинами)}
\scniselement{предметная область}
\scnsdmainclasssingle{параметр}
\scnsdclass{измеряемый параметр;неизмеряемый параметр;уровень класса эквивалентности;величина;точная величина;неточная величина;интервальная величина;параметрическая модель;измерение с фиксированной единицей измерения ;измерение по шкале;арифметическое выражение на величинах;арифметическая операция на величинах;действие. измерение;задача. измерение}
\scnsdrelation{область определения параметра*;эталон';измерение*;точность*;единица измерения*;нулевая отметка*;единичная отметка*;сумма величин*;произведение величин*;возведение величин в степень*;большая величина*;равенство величин*;большая или равная величина*}

\scnheader{параметр}
\scnidtf{характеристика}
\scnidtf{свойство}
\scnidtf{признак}
\scnidtf{класс классов}
\scnidtf{измеряемое свойство}
\scnidtf{признак классификации или покрытия некоторого класса сущностей}
\scnidtf{признак разбиения или покрытия некоторого класса сущностей}
\scnidtf{семейство множеств, разбивающих или покрывающих некоторый класс сущностей}
\scnidtf{семейство классов сущностей, обладающих одинаковым соответствующим свойством}
\scnidtf{фактор-множество, соответствующее некоторому отношению эквивалентности, или аналог фактор-множества, соответствующий некоторому отношению толерантности}
\scnsubdividing{измеряемый параметр;неизмеряемый параметр}
\scnsuperset{ориентированный параметр}
\scnexplanation{Каждый \textbf{\textit{параметр}} представляет собой класс, являющийся семейством всевозможных классов эквивалентности или толерантности, задаваемых либо \textit{отношением эквивалентности}, либо \textit{отношением толерантности} (симметричным, рефлексивным, но частично транзитивным). Так, например, элементами (значениями, величинами) \textbf{\textit{параметра}} \textit{длина} являются либо классы эквивалентности, задаваемые отношением эквивалентности ``иметь точно одинаковую длину*'', либо классы толерантности, задаваемые отношением вида ``иметь приблизительно одинаковую длину с указываемой точностью*'', либо интервальные классы, задаваемые бинарными отношениями вида ``иметь длину, находящуюся в одном и том же указываемом интервале*'' (например, от 1 метра до 2 метров).\\
Примерами параметров как отношений эквивалентности являются:
\begin{scnitemize}
    \item равновеликость геометрических фигур (по длине, площади, объему -- в зависимости от размерности этих фигур);
    \item иметь одинаковый цвет (быть эквивалентными по цвету);
    \item эквивалентность, по вкусу, запаху, твердости и т.д.
\end{scnitemize}

Заметим, что среди элементов (значений, величин) параметра могут встречаться пересекающиеся множества (классы), но объединение всех элементов каждого параметра есть не что иное, как класс всевозможных сущностей, обладающих этим параметром (свойством, характеристикой). Например, класс всех сущностей, имеющих длину, класс всех сущностей, обладающих цветом.

Каждый конкретный параметр (характеристика), т.е. каждый элемент класса всевозможных параметров (характеристик) есть, по сути, признак классификации сущностей, обладающих это характеристикой, по принципу эквивалентности (одинаковости значения) этой характеристики. Например, параметр \textit{цвет} разбивает множество сущностей имеющих цвет на классы, каждый из которых включает в себя сущности, имеющие одинаковый цвет. Параметр может разбиваться на классы для уточнения некоторого свойства, например элементами параметра цвет будут классы, соответствующие конкретным цветам (синий, красный и т.д.), в свою очередь каждый конкретный цвет может включать более частные классы, уточняющие данное свойство, например, темно-синий, светло-красный и т.д.

Другими словами, каждому множеству сущностей может ставиться в соответствие набор (семейство) параметров (параметрическое пространство), которыми обладают сущности этого множества -- аналог семейства отношений, определенных (заданных) на этом множестве. Часто бывает важно построить такое параметрическое пространство, \scnqq{точки} которого взаимнооднозначно соответствуют параметризуемым сущностям (например, набор параметров, позволяющих однозначно идентифицировать, установить личность каждого человека). 

Таким образом, для каждого используемого элемента (значения) какого-либо параметра, необходимо явно указывать спецификацию этого значения (точное значение, неточное значение, интервальное значение, точность, интервал).}
\scnrelfrom{описание примера}{
\scnfilescg{figures/sd_parameters_and_quantities/parameterDescription.png}
}

\scnheader{область определения параметра*}
\scnidtf{множество всех тех и только тех сущностей, которые являются компонентами значений заданного параметра*}
\scnidtf{множество всех тех и только тех сущностей, которые обладают заданным параметром*}
\scnrelto{включение}{объединение*}

\scnheader{измеряемый параметр}
\scnidtf{количественный параметр}
\scnidtf{семейство измеряемых величин}
\scnidtf{семейство классов эквивалентности, каждому из которых может быть поставлено в соответствие числовое значение}
\scnexplanation{Каждый \textbf{\textit{измеряемый параметр}} представляет собой \textit{параметр}, значение (элемент, экземпляр) которого трактуется как \textit{величина}, которой можно поставить в соответствие ее числовое значение на основании выбранной единицы измерения и/или точки отсчета (нулевой отметки выбранной шкалы).}
\scnsuperset{параметр, измеряемый по шкале}

\scnheader{неизмеряемый параметр}
\scnidtf{качественный параметр}

\scnheader{ориентированный параметр}
\scnidtf{упорядоченный параметр}
\scnsuperset{параметр, измеряемый по шкале}
\scnidtf{параметр, на значениях которого может быть задано некоторое отношение порядка, семантика которого уточняется в зависимости от семантики параметра}

\scnheader{порядок величин*}
\scnsubset{отношение строгого порядка}
\scnsuperset{большая величина*}
\scnnote{Связки отношения \textit{порядок величин*} могут связывать только величины \uline{одного и того же} \textit{ориентированного параметра}.}

\scnheader{соединение значений ориентированного параметра*}
\scnexplanation{Связки отношения \textit{соединение значений ориентированного параметра*} связывают связки отношения \textit{порядок величин*} и величину (возможно, интервальную) того же параметра, элементами которой являются все сущности, значение данного параметра для которых лежит в интервале, границы которого задаются величинами, являющимися компонентами указанной связки отношения \textit{порядок величин*}.}
\scnrelfrom{описание примера}{
	\scnfilescg{figures/sd_parameters_and_quantities/values_join.png}
}
\scnaddlevel{1}
	\scnnote{В приведенном примере множество сущностей, имеющий длину 1м и множество сущностей, имеющих длину 3м, при помощи отношения \textit{соединение значений ориентированного параметра*} образуют множество сущностей, имеющих длину от 1 до 3м.}
\scnaddlevel{-1}

\scnheader{уровень класса эквивалентности}
\scnidtf{уровень параметра}
\scniselement{параметр}
\scnexplanation{Параметр \textbf{\textit{уровень класса эквивалентности}} задает уровень некоторого значения некоторого \textit{параметра} в иерархии значений этого параметра. Уровень класса эквивалентности равен 1, если значение параметра не является частным по отношению к другому значению этого параметра, равен 2, если значение параметра является частным по отношению к значению этого параметра с уровнем 1 и т.д.}
\scnrelfrom{описание примера}{
\scnfilescg{figures/sd_parameters_and_quantities/color.png}
}

\scnheader{величина}
\scnidtf{значение количественного параметра}
\scnidtf{значение измеряемого параметра}
\scnidtf{класс сущностей, имеющих одинаковое значение соответствующего параметра}
\scnrelfromlist{включение}{точная величина;неточная величина;интервальная величина}
\scnexplanation{Каждая \textbf{\textit{величина}} представляет собой однозначный и независящий от шкалы измерения результат измерения некоторой характеристики у некоторой сущности.

Каждой \textbf{\textit{величине}} можно поставить в соответствие ее числовое значение на основании выбранной единицы измерения и точки отсчета (нулевой отметки выбранной шкалы, в случае, если измерение осуществляется по шкале).

Нельзя путать значение параметра (\textbf{\textit{величину}}) и значение величины по некоторой шкале, которое может быть скалярным и векторным.}

\scnheader{точная величина}
\scnidtf{точное значение параметра}
\scnidtf{множество всех точных значений параметра}
\scnidtf{значение параметра, являющееся семейством классов эквивалентности, соответствующим некоторому отношению эквивалентности}
\scnidtf{класс эквивалентности}
\scnexplanation{Каждая \textbf{\textit{точная величина}} имеет одно фиксированное значение в некоторой единице измерения или по какой-либо шкале. При этом считается, что все элементы такого класса имеют одинаковое значение данного параметра и отклонениями можно пренебречь.

Каждой \textbf{\textit{точной величине}} можно поставить в соответствие группу \textit{неточных величин}, являющихся не разбиениями, а покрытиями того же множества, но с разной степенью точности.}
\scnrelfrom{описание примера}{
\scnfilescg{figures/sd_parameters_and_quantities/exactLength.png}
\scnexplanation{В данном примере \textbf{\textit{ki}} обозначает класс сущностей, имеющих длину ровно 5 метров. Конкретный пример такой сущности - \textbf{\textit{bi}}.}}

\scnheader{неточная величина}
\scnidtf{множество неточных значений параметра}
\scnidtf{приблизительная величина}
\scnidtf{приблизительное значение параметра}
\scnidtf{значение параметра в интервале с нефиксированными границами}
\scnexplanation{Каждой \textbf{\textit{неточной величине}} ставится в соответствие ее значение в некоторой единице измерения или по какой-либо шкале, а также дополнительно указывается \textit{точность*}, т.е. возможное отклонение от данного значения.}
\scnrelfrom{описание примера}{
\scnfilescg{figures/sd_parameters_and_quantities/approximateLength.png}
\scnexplanation{В данном примере \textbf{\textit{ki}} обозначает класс сущностей, имеющих длину примерно 25 метров, при этом максимально возможное отклонение от этого значения составляет \textbf{\textit{kj}}, то есть 2 метра. Конкретный пример такой сущности - \textbf{\textit{bi}}.}}

\scnheader{интервальная  величина}
\scnidtf{интервальное значение параметра}
\scnidtf{значение параметра в интервале с фиксированными границами}
\scnidtf{интервал значения параметра из множества пересекающихся интервалов разной длины, имеющих нефиксированные границы}
\scnexplanation{Каждая \textbf{\textit{интервальная величина}} представляет собой класс сущностей, находящихся в рамках точно заданного интервала, минимальная и максимальная точка которого являются \textit{точными величинами}. Результатом \textit{измерения*} такой величины является ориентированная пара, первым компонентом которой является левая (меньшая) граница интервала, вторым компонентом -- правая (большая) граница интервала.}
\scnrelfrom{описание примера}{
\scnfilescg{figures/sd_parameters_and_quantities/intervalLength.png}
\scnexplanation{В данном примере \textbf{\textit{ki}} обозначает класс сущностей, имеющих длину, которая лежит в интервале от \textbf{\textit{kj}} до \textit{kl}, то есть в интервале от 4 до 5 метров, а \textbf{\textit{bi}} -- конкретный пример такой сущности.}}

\scnheader{эталон\scnrolesign}
\scnidtf{образец\scnrolesign}
\scniselement{ролевое отношение}
\scnexplanation{Ролевое отношение \textit{эталон\scnrolesign} указывает на тот элемент значения некоторого параметра, который в рамках данного класса эквивалентности считается эталонным, то есть он используется как образец при определении данного параметра.

\textit{эталон\scnrolesign} может задаваться как для измеряемых, так и для неизмеряемых параметров, например, эталон метра или эталон красоты.}

\scnheader{измерение*}
\scnidtf{значение параметра*}
\scnidtf{значение заданной величины заданного параметра*}
\scnidtf{измерение как соответствие*}
\scnidtf{результат измерения заданной величины в заданной единице измерения и по заданной шкале*}
\scnidtf{бинарное ориентированное отношение, связывающее различные величины с результатами их измерения в различных единицах измерения и по различным шкалам*}
\scnexplanation{Связки отношения \textit{измерение*} связывают величину и ее значение в некоторой единице измерения (в том числе, в интервале) или по некоторой шкале. Конкретная единица измерения или шкала указывается дополнительно при помощи соответствующего отношения. Одной величине может соответствовать только одно значение в каждой возможной единице измерения или одна точка на некоторой шкале.}

\scnheader{точность*}
\scnidtf{отклонение*}
\scnidtf{степень точности неточного значения параметра*}
\scniselement{бинарное отношение}
\scnexplanation{Связки отношения \textbf{\textit{точность*}} связывают \textit{неточную величину} и \textit{точную величину} того же класса, задающую максимальное возможное отклонение указанной \textit{неточной величины} от своего значения.}

\scnheader{параметрическая модель}
\scnidtf{параметрическая спецификация}
\scnidtf{параметрическое sc-описание заданной сущности}
\scnidtf{описание сущности как точки в некотором параметрическом (признаковом) пространстве}
\scnrelto{включение}{семантическая окрестность}
\scnexplanation{Каждая \textbf{\textit{параметрическая модель}} представляет собой описание заданной сущности в некотором параметрическом пространстве количественных и качественных \textit{параметров}, т.е. указание того, какие значения заданных параметров (характеристик) соответствуют описываемой (заданной) сущности.}

\scnheader{единица измерения*}
\scniselement{бинарное отношение}
\scnidtf{единица по шкале*}
\scnidtf{единичная отметка по шкале*}
\scnexplanation{Связки отношения \textbf{\textit{единица измерения*}} связывают знак конкретного \textbf{\textit{измерения с фиксированной единицей измерения}} и некоторую \textit{точную величину}, входящую в тот же конкретный \textit{параметр}, что и первый компонент связок этого конкретного измерения, и которая используется в данном случае в качестве единицы измерения.}

\scnheader{измерение с фиксированной единицей измерения }
\scnrelto{семейство подмножеств}{измерение*}
\scnexplanation{Каждая \textbf{\textit{измерение с фиксированной единицей измерения}} представляет собой подмножество отношения \textit{измерение*} и характеризуется некоторой \textit{единицей измерения*}, которая является элементом того же параметра (семейством сущностей, имеющих значение данного параметра, совпадающее с этой единицей измерения).}

\scnheader{измерение по шкале}
\scnidtf{шкала}
\scnrelto{семейство подмножеств}{измерение*}
\scnexplanation{Каждая \textbf{\textit{измерение по шкале}} представляет собой подмножество отношения \textit{измерение*} и характеризуется не единицей измерения, а некоторой точкой отсчета для данной \textbf{\textit{шкалы}}. Результатом \textbf{\textit{измерения по шкале}} будет некоторая точка шкалы, отстоящая от точки отсчета на определенное расстояние в нужную сторону (меньшую или большую). Понятно, что это расстояние может быть измерено любыми единицами измерения, но его величина при этом останется неизменной.

Не стоит путать измерение по \textbf{\textit{измерение по шкале}}, которое зависит от \textit{нулевой отметки*}, с измерением изменения того же \textit{параметра}, которое характеризуется единицей измерения и не зависит от точки отсчета. Например, не стоит путать дату по некоторому календарю, соответствующую \textit{началу} какого-либо процесса, и \textit{длительность} этого процесса, которая не зависит от выбранного календаря.}
\scnrelfrom{описание примера}{
\scnfilescg{figures/sd_parameters_and_quantities/scale.png}
\scnexplanation{В данном примере \textbf{\textit{ki}} обозначает класс сущностей, имеющих точную температуру в 330 К, а \textbf{\textit{bi}} -- конкретный пример такой сущности.}}

\scnheader{нулевая отметка*}
\scnidtf{нуль по шкале*}
\scnidtf{начало отсчета*}
\scnidtf{точка отсчета*}
\scniselement{бинарное отношение}
\scnexplanation{Связки отношения \textbf{\textit{нулевая отметка*}} связывают знак некоторого \textit{измерения по шкале} со знаком \textit{точной величины} того же \textit{параметра}, которая в рамках данной шкалы принимается за точку отсчета.}

\scnheader{арифметическое выражение на величинах}
\scnexplanation{Каждое \textbf{\textit{арифметическое выражение на величинах}} представляет собой \textit{связку}, компонентами которой являются элементы или подмножества некоторого \textit{количественного параметра}.}

\newpage
\scnheader{арифметическая операция на величинах}
\scnrelto{семейство подмножеств}{арифметическое выражение на величинах}
\scnexplanation{Каждая \textbf{\textit{арифметическая операция на величинах}} представляет собой \textit{отношение}, элементами которого являются \textit{арифметические выражения на величинах}, то есть множество \textit{арифметических выражений на величинах} какого-либо одного вида.}

\scnheader{сумма величин*}
\scnidtf{сложение величин*}
\scniselement{арифметическая операция на величинах}
\scniselement{квазибинарное отношение}
\scnexplanation{\textbf{\textit{сумма величин*}} -- это \textit{арифметическая операция на величинах}, аналогичная \textit{арифметической операции сумма*} для \textit{чисел}.

Первым компонентом связки отношения \textbf{\textit{сумма величин*}} является подмножество некоторого \textit{количественного параметра} (слагаемые \textit{величины}), содержащее два или более элемента, вторым компонентом -- элемент этого же \textit{количественного параметра}, значение которого в любой \textit{единице измерения*} является результатом сложения значений всех слагаемых \textit{величин} в той же \textit{единице измерения*}. При несовпадении \textit{единиц измерения} слагаемых величин необходимо воспользоваться соотношениями между \textit{единицами измерения}, которые задаются при помощи операций \textit{произведение величин*} и \textit{возведение величин в степень*}.}


\scnheader{произведение величин*}
\scnidtf{умножение величин*}
\scniselement{арифметическая операция на величинах}
\scniselement{квазибинарное отношение}
\scnexplanation{\textbf{\textit{произведение величин*}} -- это \textit{арифметическая операция на величинах}, аналогичная \textit{арифметической операции} \scnbigspace \textit{произведения*} для \textit{чисел}.

Первым компонентом связки отношения \textbf{\textit{произведение величин*}} является \textit{связка}, элементами которой являются либо \textit{величины количественных параметров}, либо \textit{числа}, но при этом хотя бы один элемент должен быть \textit{величиной}. Вторым компонентов является \textit{величина количественного параметра}.

Операция \textbf{\textit{произведение величин*}} может быть использована для задания соотношения между \textit{единицами измерения*} в рамках одного \textit{количественного параметра}.}
\scnrelfrom{описание примера}{
\scnfilescg{figures/sd_parameters_and_quantities/multiplicationOfQuantities.png}}

\scnrelfrom{описание примера}{
\scnfilescg{figures/sd_parameters_and_quantities/multiplicationOfQuantities2.png}}

\scnheader{возведение величин в степень*}
\scniselement{арифметическая операция на величинах}
\scniselement{бинарное отношение}
\scnexplanation{\textbf{\textit{возведение величин в степень*}} -- это \textit{арифметическая операция на величинах}, аналогичная \textit{арифметической операции возведение в степень*} для \textit{чисел}.

Первым компонентом связки отношения \textbf{\textit{возведение величин в степень*}} является ориентированная пара, первым компонентом которой является \textit{величина количественного параметра} (основание степени), вторым -- \textit{число} (показатель степени); Вторым компонентом связки отношения \textbf{\textit{возведение величин в степень*}} является \textit{величина количественного параметра} (результат возведения в степень).}
\scnrelfrom{описание примера}{
\scnfilescg{figures/sd_parameters_and_quantities/exponentiation.png}}

\scnrelfrom{описание примера}{
\scnfilescg{figures/sd_parameters_and_quantities/exponentiationTo2.png}}

\scnheader{порядок величин*}
\scnidtf{большая величина*}
\scnidtf{сравнение величин*}
\scniselement{бинарное отношение}
\scniselement{отношение строгого порядка}
\scnexplanation{\textbf{\textit{порядок величин*}} -- это \textit{отношение на величинах}, аналогичное отношению \textit{больше*} для \textit{чисел}.\\
Из двух величин большей является та, \textit{значение} которой в любой \textit{единице измерения*} \textit{больше*} значения другой \textit{величины} в той же \textit{единице измерения*}. В ряде случаев нет возможности говорить о большей или меньшей величине, но можно говорить об их упорядоченности.}

\scnheader{действие. измерение}
\scnidtf{измерение как действие}
\scnidtf{действие, направленное на установление связи, принадлежащей отношению измерение* и связывающей величину, которая принадлежит заданному параметру, и которой принадлежит заданная сущность, и соответствующее значение этой величины на некоторой шкале}
\scnidtf{действие, направленное на решение задачи измерения заданного параметра у заданной сущности}
\scnrelto{включение}{действие}

\scnheader{задача. измерение}
\scnidtf{спецификация действия измерения}
\scnidtf{спецификация действия, целью которого является измерение заданного параметра у заданной сущности}
\scnrelto{включение}{задача}

\bigskip
\scnendstruct \scnendcurrentsectioncomment

\end{SCn}

\scsubsection[\scnmonographychapter{Глава 2.4. Формальные онтологии базовых классов сущностей - множеств, связей, отношений, параметров, величин, чисел, структур, темпоральных сущностей}]{Предметная область и онтология чисел и числовых структур}
\input{Contents/chapter2/sd_ostis_sys_models/sd_kb/sd_numbers.tex}

\scsubsection[\scnmonographychapter{Глава 2.3. Структура баз знаний интеллектуальных компьютерных систем нового поколения: иерархическая система предметных областей и онтологий. Онтологии верхнего уровня. Формализация понятий семантической окрестности, предметной области и онтологии в интеллектуальных компьютерных системах нового поколения}]{Предметная область и онтология структур}
\label{sd_structures}
\input{Contents/chapter2/sd_ostis_sys_models/sd_kb/sd_structures.tex}

\scsubsection[\scnmonographychapter{Глава 2.4. Формальные онтологии базовых классов сущностей - множеств, связей, отношений, параметров, величин, чисел, структур, темпоральных сущностей}]{Предметная область и онтология темпоральных сущностей}
\label{sd_temp_entities}
\begin{SCn}

\scnsectionheader{\currentname}

\scnstartsubstruct

\scnheader{Предметная область темпоральных сущностей}
\scnidtf{Предметная область темпоральных связей и отношений}
\scnidtf{Предметная область временных сущностей}
\scniselement{предметная область}
\scnsdmainclasssingle{временная сущность}
\scnsdclass{прошлая сущность;настоящая сущность;будущая сущность;временная связь;ситуация;процесс;процесс в sc-памяти;процесс во внешней среде ostis-системы;материальная сущность;воздействие;отношение;класс временных связей;класс временных и постоянных связей;множество;ситуативное множество;неситуативное множество;частично ситуативное множество;темпоральная связь;темпоральное отношение;начало\scnsupergroupsign;завершение\scnsupergroupsign;длительность\scnsupergroupsign;тысячелетие;век;год;месяц;сутки;час;минута;секунда}
\scnsdrelation{воздействующая сущность*;объект воздействия*;начальная ситуация*;причинная ситуация*;конечная ситуация*;событие*;последний добавленный sc-элемент\scnrolesign;темпоральное включение*;темпоральная часть*;начальный этап*;конечный этап*;промежуточный этап*;темпоральное включение без совпадения начальных и конечных моментов*;темпоральное включение с совпадением начальных моментов*;темпоральное включение с совпадением конечных моментов*;темпоральное совпадение*;темпоральное объединение*;темпоральная декомпозиция*;темпоральная смежность*;темпоральная последовательность с промежутком*;темпоральная последовательность с пересечением*;номер тысячелетия\scnrolesign;номер века\scnrolesign;номер года\scnrolesign;номер месяца в году\scnrolesign;номер суток в месяце\scnrolesign;номер часа в дне\scnrolesign;номер минуты в часе\scnrolesign;номер секунды в минуте\scnrolesign}

\scnheader{временная сущность}
\scnidtf{временно существующая сущность}
\scnidtf{нестационарная сущность}
\scnidtf{сущность, имеющая и/или начало, и/или конец своего существования}
\scnidtf{sc-элемент, являющийся знаком некоторой временно существующей сущности}
\scnidtf{сущность, обладающая темпоральными характеристиками (длительностью, начальным моментом, конечным моментом и т.д.)}
\scnsubdividing{прошлая сущность;настоящая сущность;будущая сущность}
\scnsubdividing{временная связь;темпоральная структура\\
	\scnaddlevel{1}
		\scnidtf{структура, содержащая хотя бы одну временную сущность}
		\scnrelfrom{включение}{структура}
		\scnnote{Следует отличать:
			\begin{scnitemize}
				\item временный характер самой структуры как sc-элемента;
				\item временный характер sc-элементов, принадлежащих данной структуре, и сущностей, обозначаемых этими sc-элементами;
				\item временный характер пар принадлежности, связывающих структуру с ее элементами.
			\end{scnitemize}}
		\scnidtf{структура, описывающая темпоральные свойства (свойства, связанные со временем) окружающей среды, частью которой являются также и различные базы знаний кибернетических систем (в том числе и собственная база знаний).}
		\scnsubdividing{ситуация\\
			\scnaddlevel{1}
				\scnidtf{статическая темпоральная структура}
			\scnaddlevel{-1}
			;процесс\\
			\scnaddlevel{1}
				\scnidtf{динамическая структура}
				\scnidtf{динамическая темпоральная структура}
			\scnaddlevel{-1}}
	\scnaddlevel{-1}
	;материальная сущность}
\scnsubdividing{непрерывная временная сущность\\
	\scnaddlevel{1}
	\scnsubdividing{точечная временная сущность\\
		\scnaddlevel{1}
			\scnidtf{атомарная временная сущность}
			\scnidtf{условно мгновенная временная сущность}
			\scnidtf{временная сущность, длительность которой в данном контексте считается несущественной (пренебрежительно малой)}
		\scnaddlevel{-1}
		;длительная непрерывная временная сущность}
	\scnaddlevel{-1}	
	;дискретная временная сущность\\
	\scnaddlevel{1}
	\scnidtf{временная сущность, которая может быть декомпозирована на последовательность точечных временных сущностей}
	\scnidtf{временная сущность, которой соответствует некоторый временной ряд параметров (состояний) точечных временных сущностей, на которые декомпозируется исходная временная сущность}
	\scnaddlevel{-1}	
	;прерывистая временная сущность\\
	\scnaddlevel{1}
	\scnidtf{временная сущность, являющаяся результатом соединения нескольких не только точечных временных сущностей}
	\scnidtf{временная сущность с прерываниями}
	\scnaddlevel{-1}
}
\scnaddlevel{1}
	\scnnote{Следует отметить, что приведенная классификация \textit{временных сущностей} характеризует не столько сами \textit{временные сущности}, сколько наши знания о них и степень детализации знаний об этих сущностях, с которой они описаны в базе знаний. Так, если для решения конкретных задач не важно, как изменялась некоторая \textit{временная сущность} в рамках какого-либо периода времени, а важно только ее начальное и конечное состояние, то она может рассматриваться как \textit{точечная временная сущность}. Впоследствии же та же \textit{временная сущность} может быть рассмотрена и описана с большей степенью детализации, и таким образом, уже не будет точечной.}
\scnaddlevel{-1}
\scnexplanation{Следует отличать:
\begin{scnitemize}
    \item временный характер сущности, обозначаемой \textit{sc-элементом};
    \item временный характер существования самого \textit{sc-элемента} в рамках \textit{sc-памяти}, поскольку в ходе обработки информации каждый \textit{sc-элемент} может быть удален из \textit{sc-памяти}; 
    \item временный характер описываемых ситуаций, событий и самих процессов;
    \item временный характер хранения в sc-памяти тех sc-конструкций, которые являются самими описаниями соответствующих ситуаций, событий и процессов.
\end{scnitemize}
}

\scnheader{следует отличать*}
\scnhaselementset{временная сущность;процесс}
\scnnote{Следует отличать, например, \textit{материальную сущность} (некоторый физический или, в частности, биологический объект) от различных динамических структур (\textit{процессов}), которые с той или иной степенью детализации и в том или ином ракурсе отражают (описывают) динамику изменений этой \textit{материальной сущности}. 

При этом следует помнить, что сам \textit{процесс} как уточнение динамики некоторой последовательности ситуаций и событий, также является сущностью, принадлежащей к классу \textit{временных сущностей}.}

\scnheader{прошлая сущность}
\scnidtf{сущность, существовавшая в прошлом времени}
\scnidtf{сущность прошлого времени}
\scnidtf{сущность, завершившая свое существование}

\scnheader{настоящая сущность}
\scnidtf{сущность, существующая в текущий момент времени}
\scnidtf{сущность, существующая сейчас}
\scnidtf{сущность настоящего времени}

\scnheader{будущая сущность}
\scnidtf{возможно будущая сущность}
\scnidtf{прогнозируемая временная сущность}
\scnidtf{временная сущность, которая может существовать в будущем}
\scnidtf{сущность, которая может или должна начать свое существование в будущем времени}
\scnrelfrom{включение}{инициированное действие}
\scnexplanation{Каждой \textbf{\textit{будущей сущности}} можно поставить в соответствие вероятность ее возникновения.}

\scnheader{временная связь}
\scnidtf{нестационарная связь}
\scnidtf{временно существующая связь}
\scnexplanation{Каждая \textbf{\textit{временная связь}} представляет собой \textit{связку}, принадлежащую множеству \textit{временных сущностей}.

Понятие \textbf{\textit{временной связи}} не следует путать с понятием \textit{темпоральной связи}, которая сама является \textit{постоянной сущностью}, описывающей то, как связаны во времени некоторые \textit{временные сущности}.
}

\scnheader{ситуация}
\scnidtf{состояние}
\scnidtf{временная структура}
\scnidtf{временно существующая структура}
\scnidtf{квазистационарная структура}
\scnidtf{состояние некоторой динамической системы, описываемое с некоторой степенью детализации (подробности)}
\scnidtf{квазистационарная структура, существующая временно (в течение некоторого отрезка времени)}
\scnexplanation{Под ситуацией понимается \textit{структура}, содержащая, по крайней мере, один элемент, который является \textit{временной сущностью}. Наличие в рамках ситуации нескольких \textit{временных сущностей} говорит о том, что существует момент времени (в прошлом, настоящем или будущем), в который все они существуют одновременно.}

\scnheader{процесс}
\scnidtf{процесс преобразования некоторой временной сущности из одного состояния в другое}
\scnidtf{процесс перехода от одной ситуации к другой}
\scnidtf{абстрактный процесс}
\scnidtf{информационная модель некоторых преобразований}
\scnidtf{динамическая sc-модель}
\scnidtf{динамическая структура}
\scnrelfrom{включение}{воздействие}
\scnexplanation{Каждый \textbf{\textit{процесс}} определяется (задается) либо возникновением или исчезновением связей, связывающих заданную \textit{временную сущность} с другими сущностями, либо возникновением или исчезновением связей, связывающих части указанной \textit{временной сущности} с другими сущностями. 

Многим \textbf{\textit{процессам}} можно поставить в соответствие \textit{ситуацию}, которая является его \textit{начальной ситуацией*} и \textit{ситуацию}, которая является его \textit{конечной ситуацией*}, то есть указать \textit{ситуации}, переход между которыми осуществляется в ходе \textbf{\textit{процесса}}.

При этом знаки тех \textit{временных сущностей}, с которыми связаны изменения, описываемые некоторым \textbf{\textit{процессом}}, входят в данный \textbf{\textit{процесс}} как элементы и при необходимости уточняются дополнительными \textit{ролевыми отношениями}.}
\scnsubdividing{процесс в sc-памяти;процесс во внешней среде ostis-системы}
\scnnote{Каждой \textbf{\textit{материальной сущности}} можно поставить в соответствие различные \textit{процессы}, описывающие ее преобразование из одного состояния в другое.}
\scnnote{Поскольку \textit{процесс} представляет собой изменяющуюся во времени динамическую структуру, то полностью представить процесс в базе знаний в общем случае не представляется возможным. Однако, можно ввести sc-элемент, обозначающий конкретный процесс, можно с необходимой степенью детализации описать декомпозицию процесса на более частные подпроцессы, описать ситуации, соответствующие состояниям процесса в разные моменты времени. В данном случае можно провести некоторую аналогию с \textit{бесконечными множествами}, все элементы которых физически не могут быть представлены в базе знаний одновременно, тем не менее, само множество и некоторые из его элементов могут быть описаны с необходимой степенью детализации.}

\scnheader{воздействие}
\scnidtf{процесс, осуществляющийся на основе (в результате) воздействия одной сущности на другую}
\scnrelfrom{включение}{действие}
\scnexplanation{Каждому \textbf{\textit{воздействию}} может быть поставлена в соответствие (1) некоторая \textit{воздействующая сущность*}, т.е. сущность, осуществляющая \textbf{\textit{воздействие}} (в частности, это может быть некоторое физическое поле), и (2) некоторый \textit{объект воздействия*}, т.е. сущность, на которую воздействие направлено. Если \textbf{\textit{воздействие}} связано с \textit{материальной сущностью}, то его объектом воздействия является либо сама эта \textit{материальная сущность}, либо некоторая ее пространственная часть.}

\scnheader{исходная ситуация*}
\scnidtf{начальная ситуация процесса*}
\scnidtf{начальная ситуация*}
\scniselement{бинарное отношение}
\scnrelfrom{первый домен}{процесс}
\scnrelfrom{второй домен}{ситуация}
\scnexplanation{Связки отношения \textbf{\textit{исходная ситуация*}} связывают некоторый \textit{процесс} и некоторую ситуацию, являющуюся начальной для этого \textit{процесса}, и, как правило, изменяемой в течение выполнения этого \textit{процесса}.

Первым компонентом каждой связки отношения \textbf{\textit{исходная ситуация*}} является знак \textit{процесса}, вторым -- знак начальной \textit{ситуации}.}

\scnheader{причинная ситуация*}
\scnsubset{начальная ситуация*}
\scnexplanation{Под причинной ситуацией понимается такая \textit{начальная ситуация*}, которая обладает достаточной полнотой для однозначного задания инициируемого \textit{процесса}.}

\scnheader{конечная ситуация*}
\scnidtf{конечная ситуация процесса*}
\scnidtf{результирующая ситуация*}
\scniselement{бинарное отношение}
\scnrelfrom{первый домен}{процесс}
\scnrelfrom{второй домен}{ситуация}
\scnexplanation{Связки отношения \textbf{\textit{конечная ситуация*}} связывают некоторый \textit{процесс} и некоторую \textit{ситуацию}, ставшую результатом выполнения этого \textit{процесса}, то есть его следствием.

Первым компонентом каждой связки отношения \textbf{\textit{конечная ситуация*}} является знак \textit{процесса}, вторым -- знак конечной \textit{ситуации}.}

\scnheader{точечный процесс}
\scnidtf{атомарный процесс}
\scnidtf{условно мгновенный процесс}
\scnidtf{процесс, длительность которого в данном контексте считается несущественной (пренебрежимо малой)}
\scnsubset{точечная временная сущность}

\scnheader{элементарный процесс}
\scnidtf{процесс, детализация которого на входящие в него подпроцессы в текущем контексте не производится}
\scnsuperset{точечный процесс}
\scnnote{Элементарные процессы могут иметь длительность и, следовательно, не обязательно являются атомарными процессами.}
\scnnote{Понятия \textit{точечного процесса} и \textit{элементарного процесса}, как и понятие \textit{точечной временной сущности} в целом, характеризуют не столько характеристики самого \textit{процесса}, сколько степень наших знаний о нем и степень детализации описания процесса в базе знаний. Так, очевидно, что любой процесс, протекающий в компьютерной системе, может быть при необходимости детализирован до уровня команд процессора, затем до уровня микропрограмм и даже до уровня физических процессов (изменения физических характеристик сигналов), однако чаще всего такая детализация не требуется.}
\scnaddlevel{1}
	\scnrelto{примечание}{точечный процесс}
\scnaddlevel{-1}

\scnheader{событие}
\scnsubset{точечная временная сущность}
\scnidtf{точечная временная сущность, являющаяся началом и/или завершением какой-либо временной сущности, например, процесса}
\scnidtf{граничная точка временной сущности}
\scnrelfrom{описание примера}{
\scnfilescg{figures/sd_temp_entities/event.png}
}
\scnaddlevel{1}
	\scnnote{Одно и то же событие может быть одновременно завершением одной временной сущности и началом другой. В приведенном примере событие $\bm{ei}$ является завершением временной сущности $\bm{si}$ и началом временной сущности $\bm{sj}$.}
\scnaddlevel{-1}

\scnheader{начало*}
\scnidtf{быть начальным событием заданной временной сущности*}
\scnrelfrom{первый домен}{временная сущность}
\scnrelfrom{второй домен}{событие}
\scnidtf{быть начальной точечной временной частью заданной временной сущности*}

\scnheader{завершение*}
\scnidtf{конец*}
\scnidtf{быть конечным событием заданной временной сущности*}
\scnidtf{быть конечной точечной временной частью заданной временной сущности*}
\scnrelfrom{первый домен}{временная сущность}
\scnrelfrom{второй домен}{событие}

\scnheader{событие*}
\scniselement{бинарное отношение}
\scnexplanation{Связки отношения \textbf{\textit{событие*}} связывают знак процесса и ориентированную пару, первым компонентом которой является знак \textit{начальной ситуации*} данного процесса, вторым компонентом -- знак \textit{конечной ситуации*} данного процесса.}
\scnrelfrom{описание примера}{
\scnfilescg{figures/sd_temp_entities/nrel_event.png}
}

\scnheader{детализация процесса*}
\scnidtf{Бинарное ориентированное отношение, каждая связка которого связывает некоторый процесс с более детальным его описанием, что предполагает представление декомпозиции этого процесса на систему взаимосвязанных его подпроцессов (в том числе элементарных).}
\scnrelfrom{пример}{Переход от процесса, соответствующего какой-либо программе, к рассмотрению декомпозиции этого процесса (протокола) в терминах языка программирования высокого уровня, затем переход для каждого из полученных подпроцессов (операторов языка высокого уровня) к детализации выполнения этих подпроцессов на уровне машинных операций, выполняемых процессором компьютера (на уровне ассемблера), и далее к детализации выполнения подпроцессов уровня машинных операций к подпроцессам на уровне языка микропрограммирования. Таким образом, детализация процесса может быть иерархической, вплоть до уровня \textit{элементарных процессов}.}

\scnheader{отношение}
\scnsubdividing{класс временных связей;класс постоянных связей;класс временных и постоянных связей}

\scnheader{класс временных связей}
\scnidtf{отношение, все связки которого являются нестационарными}
\scnexplanation{В общем случае \textbf{\textit{класс временных связей}} не является \textit{ситуативным множеством}, поскольку факт принадлежности некоторой \textit{временной связи} такому классу следует считать постоянным, а не временным, поскольку временность/постоянство связи и ее семантический тип, задаваемый классом (отношением), это принципиально разные параметры (характеристики, признаки) любой связи.}

\scnheader{класс постоянных связей}
\scnidtf{отношение, все связки которого являются стационарными}

\scnheader{класс временных и постоянных связей}
\scnidtf{отношение, некоторые (но не все) связки которого являются нестационарными}

\scnheader{множество}
\scnsubdividing{ситуативное множество;неситуативное множество;частично ситуативное множество}

\scnheader{ситуативное множество}
\scnidtf{полностью ситуативное множество}
\scnexplanation{Под \textbf{\textit{ситуативным множеством}} понимается постоянное множество, у которого все выходящие из него связи принадлежности являются \textit{временными сущностями}.

В частности, ситуативное множество может использоваться как вспомогательная динамическая структура, которая содержит элементы некоторых структур, обрабатываемые в данный момент, например, это может быть копия некоторого множества, из которой постепенно удаляются элементы по мере их просмотра и обработки. В случае, когда такая структура содержит всего один элемент, ее можно считать \underline{указателем} на данный элемент, при этом в разные моменты времени это могут быть разные элементы.}

\scnheader{последний добавленный sc-элемент’}
\scniselement{ролевое отношение}

\scnheader{неситуативное множество}
\scnexplanation{Под \textbf{\textit{неситуативным множеством}} понимается постоянное множество, у которого все выходящие из него связи принадлежности являются \textit{постоянными сущностями}.}

\scnheader{частично ситуативное множество}
\scnexplanation{Под \textbf{\textit{частично ситуативным множеством}} понимается постоянное множество, у которого некоторые (но не все) выходящие из него связи принадлежности являются \textit{временными сущностями}.}

\scnheader{темпоральная связь}
\scnidtf{связь во времени}
\scnidtf{\uline{постоянная} связь, описывающая связь во времени между временными сущностями}

\scnheader{темпоральное отношение}
\scnrelto{семейство подмножеств}{темпоральная связь}
\scnidtf{класс темпоральных связей}
\scnidtf{отношение, задающее темпоральные связи между временными сущностями}
\scnhaselement{темпоральное включение*}
\scnhaselement{темпоральное объединение*}
\scnhaselement{темпоральная декомпозиция*}
\scnhaselement{темпоральная последовательность*}
\scnaddlevel{1}
\scnsubdividing{темпоральная смежность*;темпоральная последовательность с промежутком*;темпоральная последовательность с пересечением*}
\scnaddlevel{-1}

\scnheader{темпоральное включение*}
\scnexplanation{Связки отношения \textbf{\textit{темпоральное включение*}} связывают две \textit{временные сущности}, период существования одной из которых полностью включается в период существования второй.\\
Первым компонентом каждой связки отношения \textbf{\textit{темпоральное включение*}} является знак \textit{временной сущности}, \textit{длительность} существования которой больше.}
\scnsuperset{темпоральная часть*}
\scnsuperset{темпоральное включение без совпадения начальных и конечных моментов*}
\scnsuperset{темпоральное совпадение*}
\scnsuperset{темпоральное включение с совпадением начальных моментов*}
\scnsuperset{темпоральное включение с совпадением конечных моментов*}

\scnheader{темпоральная часть*}
\scnidtf{этап (период) заданной временной сущности*}
\scnidtf{этап процесса существования временной сущности*}
\scnsuperset{начальный этап*}
\scnsuperset{конечный этап*}
\scnsuperset{промежуточный этап*}
\scnsuperset{подпроцесс*}
\scnaddlevel{1}
	\scnrelfrom{первый домен}{процесс}
	\scnrelfrom{второй домен}{процесс}
\scnaddlevel{-1}
\scnrelfrom{описание примера}{
\scnfilescg{figures/sd_temp_entities/temporal_part.png}
}
\scnrelfrom{иллюстрация}{
\scnfileimage{\includegraphics[width=1\linewidth]{figures/sd_temp_entities/img_temporal_part.png}}}
\scntext{примечание}{Связки отношения \textbf{\textit{темпоральная часть*}} связывают две \textit{временные сущности}, одна из которых является частью другой, например, действие и одно из его поддействий. Соответственно, период существования одной из этих сущностей всегда будет включаться в период существования другой (большей).

В отличие от более общего отношения \textit{темпоральное включение*}, связки которого могут связывать любые \textit{временные сущности}, связки отношения \textbf{\textit{темпоральная часть*}} связывают только \textit{временные сущности}, одна из которых является частью другой.}

\scnheader{следует отличать*}
\scnhaselementset{темпоральная часть*\\
	\scnaddlevel{1}
		\scnsuperset{подпроцесс*}
	\scnaddlevel{-1}
	;темпоральное включение*\\
\scnaddlevel{1}
	\scnnote{Связь \textit{темпорального включения*} может связывать абсолютно разные \textit{временные сущности}, существующие в общем случае в разных местах, а не только \textit{временные сущности}, одна из которых является частью другой. Хотя формально и можно объединить любые разные \textit{временные сущности} в одну общую \textit{временную сущность}, далеко не всегда имеет смысл это делать.}
\scnaddlevel{-1}}

\scnheader{темпоральное включение без совпадения начальных и конечных моментов*}
\scnidtf{строгое темпоральное включение*}
\scnrelfrom{описание примера}{
\scnfilescg{figures/sd_temp_entities/strict_temporal_inclusion.png}}
\scnrelfrom{иллюстрация}{
\scnfileimage{\includegraphics[width=1\linewidth]{figures/sd_temp_entities/img_strict_temporal_inclusion.png}}}

\scnheader{темпоральное включение с совпадением начальных моментов*}
\scnrelfrom{описание примера}{
\scnfilescg{figures/sd_temp_entities/temporal_include_with_match_start_points.png}}
\scnrelfrom{иллюстрация}{
\scnfileimage{\includegraphics[width=1\linewidth]{figures/sd_temp_entities/img_temporal_include_with_match_start_points.png}}}

\scnheader{темпоральное включение с совпадением конечных моментов*}
\scnrelfrom{описание примера}{
\scnfilescg{figures/sd_temp_entities/temporal_include_with_terminal_point_match.png}}
\scnrelfrom{иллюстрация}{
\scnfileimage{\includegraphics[width=1\linewidth]{figures/sd_temp_entities/img_temporal_include_with_terminal_point_match.png}}}

\scnheader{темпоральное совпадение*}
\scnidtf{совпадение начала и завершения*}
\scniselement{отношение эквивалентности}

\scnheader{темпоральное объединение*}
\scnidtf{преобразование нескольких временных сущностей в одну общую временную сущность, которая может оказаться прерывистой или даже дискретной*}
\scnrelboth{аналог}{объединение множеств*}
\scnnote{С формальной точки зрения объединять можно любые временные сущности. Но делать это надо только тогда, когда это имеет смысл, точно так же, как и в случае объединения множеств.}
\scnrelfrom{описание примера}{
\scnfilescg{figures/sd_temp_entities/temporal_union.png}}
\scnrelfrom{иллюстрация}{
\scnfileimage{\includegraphics[width=1\linewidth]{figures/sd_temp_entities/img_temporal_union.png}}}

\scnheader{темпоральная декомпозиция*}
\scnidtf{Темпоральное отношение, связывающее временную сущность и множество смежных во времени временных сущностей, которые являются темпоральными частями исходной сущности и результатом темпорального объединения которых является исходная сущность*}
\scnrelboth{аналог}{разбиение*}
\scnrelfrom{описание примера}{
\scnfilescg{figures/sd_temp_entities/temporal_decomposition.png}}
\scnrelfrom{иллюстрация}{
\scnfileimage{\includegraphics[width=1\linewidth]{figures/sd_temp_entities/img_temporal_decomposition.png}
}}

\scnheader{темпоральная смежность*}
\scnidtf{сразу позже*}
\scnidtf{смежность во времени*}
\scnidtf{строгая темпоральная последовательность (без темпорального промежутка)*}
\scnidtf{темпоральная последовательность без промежутка*}
\scnrelfrom{описание примера}{
\scnfilescg{figures/sd_temp_entities/temporal_adjacency.png}}
\scnrelfrom{иллюстрация}{
\scnfileimage{\includegraphics[width=1\linewidth]{figures/sd_temp_entities/img_temporal_adjacency.png}
}}

\scnheader{темпоральная последовательность с промежутком*}
\scnidtf{позже*}
\scnrelfrom{описание примера}{
\scnfilescg{figures/sd_temp_entities/temporal_sequence_with_intermediate.png}}
\scnrelfrom{иллюстрация}{
\scnfileimage{\includegraphics[width=1\linewidth]{figures/sd_temp_entities/img_temporal_sequence_with_intermediate.png}}}

\scnheader{темпоральная последовательность с пересечением*}
\scnrelfrom{описание примера}{
\scnfilescg{figures/sd_temp_entities/temporal_sequence_with_intersection.png}
}
\scnrelfrom{иллюстрация}{
\scnfileimage{\includegraphics[width=1\linewidth]{figures/sd_temp_entities/img_temporal_cross_sequence.png}
}}

\scnheader{начало\scnsupergroupsign}
\scnidtf{одновременность начинаний\scnsupergroupsign}
\scnidtf{класс одновременно начавшихся сущностей\scnsupergroupsign}
\scniselement{параметр}
\scnexplanation{Каждый элемент множества \textbf{начало} представляет собой класс \textit{временных сущностей}, у которых совпадает момент начала их существования. Конкретное значение данного \textit{параметра} может быть как \textit{точной величиной}, так и \textit{неточной величиной} или \textit{интервальной величиной}.}
\scnrelfrom{описание примера}{
\scnfilescg{figures/sd_temp_entities/start.png}}
\scnaddlevel{1}
\scnexplanation{В данном примере \textbf{\textbf{\textit{ki}}} обозначает класс сущностей, начавших свое существование 19 февраля 2015 года по григорианскому календарю. Конкретные примеры таких сущностей -- \textbf{\textit{bi}} и \textbf{\textit{bj}}. \textbf{\textit{ti}} обозначает временную точку григорианского календаря, соответствующую 19 февраля 2015 года.}
\scnaddlevel{-1}

\scnheader{завершение\scnsupergroupsign}
\scnidtf{конец\scnsupergroupsign}
\scnidtf{одновременность завершений\scnsupergroupsign}
\scnidtf{класс одновременно завершившихся сущностей\scnsupergroupsign}
\scniselement{параметр}
\scnexplanation{Каждый элемент множества \textbf{\textit{завершение}} представляет собой класс \textit{временных сущностей}, у которых совпадает конечный момент их существования (момент завершения существования). Конкретное значение данного \textit{параметра} может быть как \textit{точной величиной}, так и \textit{неточной величиной} или \textit{интервальной величиной}.}
\scnrelfrom{описание примера}{
\scnfilescg{figures/sd_temp_entities/completion.png}}
\scnaddlevel{1}
\scnexplanation{В данном примере \textbf{\textit{ki}} обозначает класс сущностей, завершивших свое существование 21 февраля 2015 года по григорианскому календарю. Конкретные примеры таких сущностей -- \textbf{\textit{bi}} и \textbf{\textit{bj}}. \textbf{\textit{ti}} обозначает временную точку григорианского календаря, соответствующую 21 февраля 2015 года.}
\scnaddlevel{-1}

\scnheader{одновременность\scnsupergroupsign}
\scnidtf{параметр, значениями (элементами) которого являются классы либо одновременно существующих (происходящих) \textit{точечных временных сущностей}, одновременность которых рассматривается с заданной степенью точности, либо одновременно начинающихся и заканчивающихся длительных процессов}
\scnexplanation{Важно отметить, что элементами некоторого значения параметра \textit{одновременности} с заданной точностью могут быть только те временные сущности, которые и начались, и завершились в течение периода времени, заданного указанным значением этого параметра, но при этом начало и завершение этих временных сущностей не обязательно должно совпадать с началом и завершением указанного периода времени. Так, например, можно ввести значение параметра \textit{одновременности} ``\textit{2022 год по Григорианскому календарю}'', элементами которого будут все временные сущности, начавшие и закончившие свое существовавшие в рамках 2022 года. При этом не обязательно, чтобы эти временные сущности начались именно в полночь 1 января 2022 года и закончились в полночь 1 января 2023 года, это могут быть временные сущности, существовавшие, например, в течение июля 2022 года.}
\scnrelfrom{описание примера}{
\textit{}\scnfilescg{figures/sd_temp_entities/simultaneity.png}}
\scnaddlevel{1}
	\scnnote{Некоторые значения параметра одновременность могут быть подмножествами других значений того же параметра. С семантической точки зрения такая взаимосвязь будет означать, что первое из указанных значений описывает \textit{одновременность} \textit{временных сущностей} с большей точностью. Так, в приведенном примере величина ``\textit{2002 год}'' описывает одновременность временных сущностей с точностью до года, а величина ``\textit{июль 2022 года}'' описывает одновременность временных сущностей с точностью до месяца. При этом очевидно, что сущности, входящие во величину ``\textit{июль 2022 года}'' будут также входить и в величину ``\textit{2022 год}'' (как например временная сущность $\bm{sk})$. В приведенном примере для простоты предполагается, что все измерения производятся по Григорианскому календарю.}
\scnaddlevel{-1}

\scnheader{соединение значений ориентированного параметра*}
\scnrelfrom{описание примера}{
	\scnfilescg{figures/sd_temp_entities/temporal_values_join.png}
}
\scnaddlevel{1}
\scnnote{В приведенном примере множество сущностей, существовавших 10.01.2022, и множество сущностей, существовавших 12.01.2022, при помощи отношения \textit{соединение значений ориентированного параметра*} образуют множество сущностей, существовавших в период 10-12.02.2022.}
\scnaddlevel{-1}

\scnheader{следует отличать*}
\scnhaselementset{темпоральное совпадение*\\
	\scnaddlevel{1}
		\scniselement{отношение эквивалентности}
	\scnaddlevel{-1}
	;одновременность\scnsupergroupsign\\
	\scnaddlevel{1}
		\scnidtf{фактор-множество для отношения темпоральное совпадение*}
	\scnaddlevel{-1}}

\scnheader{длительность\scnsupergroupsign}
\scnidtf{класс временных сущностей, имеющих одинаковую длительность\scnsupergroupsign}
\scniselement{параметр}
\scnhaselement{тысячелетие}
\scnhaselement{век}
\scnhaselement{год}
\scnhaselement{месяц}
\scnhaselement{день}
\scnhaselement{час}
\scnhaselement{минута}
\scnhaselement{секунда}
\scnexplanation{Каждый элемент множества \textbf{\textit{длительность}} представляет собой класс \textit{временных сущностей}, у которых совпадает длительность их существования. Конкретное значение данного \textit{параметра} может быть как \textit{точной величиной}, так и \textit{неточной величиной} или \textit{интервальной величиной}.}
\scnrelfrom{описание примера}{
\scnfilescg{figures/sd_temp_entities/duration.png}}
\scnaddlevel{1}
\scnexplanation{В данном примере \textbf{\textit{ki}} обозначает класс сущностей, существовавших в течение 2 месяцев. Конкретный пример такой сущности -- \textbf{\textit{bi}}.}
\scnaddlevel{-1}

\bigskip
\scnendstruct \scnendcurrentsectioncomment

\end{SCn}

\scsubsubsection[\scnmonographychapter{Глава 2.4. Формальные онтологии базовых классов сущностей - множеств, связей, отношений, параметров, величин, чисел, структур, темпоральных сущностей}]{Предметная область и онтология ситуаций и событий, описывающих динамику баз знаний ostis-систем}
\label{sd_temp_know_base}
\begin{SCn}

\scnsectionheader{\currentname}

\scnstartsubstruct

\scntext{введение}{Обработка информации в \textit{sc-памяти} (т.е. динамика базы знаний, хранимой в \textit{sc-памяти}) в конечном счете сводится:
	\begin{scnitemize}
		\item к появлению в \textit{sc-памяти} новых актуальных \textit{sc-узлов} и \textit{sc-коннекторов};
		\item к логическому удалению актуальных \textit{sc-элементов}, т.е. к переводу их в неактуальное состояние (это необходимо для хранения протокола изменения состояния базы знаний, в рамках которого могут описываться действия по удалению \textit{sc-элементов});
		\item к возврату логически удаленных \textit{sс-элементов} в статус актуальных (необходимость в этом может возникнуть при откате базы знаний к какой-нибудь ее прошлой версии);
		\item к физическому удалению \textit{sc-элементов};
		\item к изменению состояния актуальных (логически не удаленных \textit{sc-элементов}) -- \textit{sc-узел} может превратиться в \textit{sc-ребро}, \textit{sc-ребро} может превратиться в \textit{sc-дугу}, \textit{sc-дуга} может поменять направленность, \textit{sc-дуга} общего вида может превратиться в \textit{константную стационарную sc-дугу принадлежности}, и т.д.;
	\end{scnitemize}
	Подчеркнем, что временный характер самого \textit{sc-элемента} (т.к. он может появиться или исчезнуть) никак не связан с возможно временным характером сущности, обозначаемой этим \textit{sc-элементом}. Т.е. временный характер самого sc-элемента и временный характер сущности, которую он обозначает -- абсолютно разные вещи.
	
	Таким образом, следует четко отличать динамику внешнего мира, описываемого базой знаний, а динамику самой базы знаний (динамику внутреннего мира). При этом очень важно, чтобы описание динамики базы знаний также входило в состав каждой базы знаний.
	
	К числу понятий, используемых для описания динамики базы знаний относятся:
	\begin{scnitemize}
		\item логически удаленный sc-элемент;
		\item сформированное множество;
		\item вычисленное число;
		\item сформированное высказывание;
\end{scnitemize}}

\scnheader{Предметная область темпоральных сущностей базы знаний ostis-системы}
\scnidtf{Предметная область, описывающая динамику базы знаний, хранимой в sc-памяти}
\scniselement{предметная область}
\scnsdmainclasssingle{ситуация}
\scnsdclass{sc-элемент;наcтоящий sc-элемент;логически удаленный sc-элемент;число;невычисленное число;вычисленное число;понятие;основное понятие;неосновное понятие;понятие, переходящее из основного в неосновное;понятие, переходящее из неосновного в основное;специфицированная сущность;недостаточно специфицированная сущность;достаточно специфицированная сущность;средне специфицированная сущность;структура;файл;событие в sc-памяти*;элементарное событие в sc-памяти*;событие добавления sc-дуги, выходящей из заданного sc-элемента*;событие добавления sc-дуги, входящей в заданный sc-элемент*;событие добавления sc-ребра, инцидентного заданному sc-элементу*;событие удаления sc-дуги, выходящей из заданного sc-элемента*;событие удаления sc-дуги, входящей в заданный sc-элемент*;событие удаления sc-ребра, инцидентного заданному sc-элементу*;событие удаления sc-элемента*;событие изменения содержимого файла*}

\scnheader{sc-элемент}
\scnreltoset{разбиение}{наcтоящий sc-элемент;логически удаленный sc-элемент}

\scnheader{наcтоящий sc-элемент}
\scniselement{ситуативное множество}

\scnheader{логически удаленный sc-элемент}
\scniselement{ситуативное множество}

\scnheader{число}
\scnsubdividing{невычисленное число;вычисленное число}

\scnheader{невычисленное число}
\scniselement{ситуативное множество}

\scnheader{вычисленное число}

\scnheader{понятие}
\scnsubdividing{основное понятие;неосновное понятие;понятие, переходящее из основного в неосновное;понятие, переходящее из неосновного в основное}

\scnheader{основное понятие}
\scnidtf{основное понятие для данной ostis-системы}
\scniselement{ситуативное множество}
\scnexplanation{К \textbf{\textit{основным понятиям}} относятся те понятия, которые активно используются в системе и могут быть ключевыми элементами sc-агентов. К \textbf{\textit{основным понятиям}} относятся также все неопределяемые понятия.}

\scnheader{неосновное понятие}
\scnidtf{дополнительное понятие}
\scnidtf{вспомогательное понятие}
\scnidtf{неосновное понятие для данной ostis-системы}
\scniselement{ситуативное множество}
\scnexplanation{Каждое \textbf{\textit{неосновное понятие}} должно быть строго определено на основе \textit{основных понятий}. Такие \textbf{\textit{неосновные понятия}} используются только для понимания и правильного восприятия вводимой информации, в том числе, для выравнивания онтологий. Ключевым элементом \textit{sc-агентов} \textbf{\textit{неосновные понятия}} быть не могут.}
\scntext{правило идентификации экземпляров}{В случае, когда некоторое понятие полностью перешло из \textit{основных понятий} в неосновные, то есть стало \textbf{\textit{неосновным понятием}}, и соответствующее ему \textit{основное понятие} (через которое оно определяется) в рамках некоторого внешнего языка имеет одинаковый с ним основной идентификатор, то к идентификатору \textbf{\textit{неосновного понятия}} спереди добавляется знак \#. Если при этом соответствуюшее \textit{основное понятие} имеет в идентификаторе знак \$, добавленный в процессе перехода, то этот знак удаляется. Если указанные понятия имеют разные основные идентификаторы в рамках этого внешнего языка, то никаких дополнительных средств идентификации не используется.

Например:\\
\textit{\#трансляция sc-текста}\\
\textit{\#scp-программа}}

\scnheader{понятие, переходящее из основного в неосновное}
\scniselement{ситуативное множество}

\scnheader{понятие, переходящее из неосновного в основное}
\scniselement{ситуативное множество}
\scntext{правило идентификации экземпляров}{В случае, когда текущее \textit{основное понятие} и соответствующее ему \textbf{\textit{понятие, переходящее из неосновного в основное}} в рамках некоторого внешнего языка имеют одинаковый основной идентификатор, то к идентификатору понятия, переходящего из неосновного в основное спереди добавляется знак \$. Если указанные понятия имеют разные основные идентификаторы в рамках этого внешнего языка, то никаких дополнительных средств идентификации не используется.

Например:\\
\textit{\$трансляция sc-текста}\\
\textit{\$scp-программа}}

\scnheader{специфицированная сущность}
\scnsubdividing{недостаточно специфицированная сущность;достаточно специфицированная сущность;средне специфицированная сущность}

\scnheader{достаточно специфицированная сущность}
\scnexplanation{К \textbf{\textit{достаточно специфицированным сущностям}} предъявляются следующие требования:
\begin{scnitemize}
    \item если сущность не является понятием, то для нее должны быть указаны
    \begin{scnitemizeii}
    \item различные варианты обозначающих ее внешних знаков;
    \item классы, которым она принадлежит;
    \item связки, которыми она связана с другими сущностями (с указанием соответствующего отношения);
    \item значения параметров, которыми она обладает;
    \item те разделы базы знаний, в которых указанная сущность является ключевой;
    \item предметные области, в которые данная сущность входит.
    \end{scnitemizeii}
    \item если специфицированная сущность является понятием, то для нее должны быть указаны:
    \begin{scnitemizeii}
    \item различные варианты внешних обозначений этого понятия;
    \item предметные области, в которых это понятие исследуется;
    \item определение понятия;
    \item пояснения
    \item разделы базы знаний, в которых это понятие является ключевым;
    \item описание примера -- пример экземпляра понятия.
    \end{scnitemizeii}
\end{scnitemize}}

\scnheader{структура}
\scnsubdividing{сформированная структура;несформированная структура}
\scnsubdividing{недостаточно сформированная структура;достаточно сформированная структура;структура, имеющая средний уровень сформированности}

\scnheader{файл}
\scnsubdividing{недостаточно сформированный внутренний файл;достаточно сформированный внутренний файл;внутренний файл, имеющий средний уровень сформированности}

\scnheader{событие в sc-памяти}
\scnsuperset{событие}

\scnheader{элементарное событие в sc-памяти}
\scnsubset{событие в sc-памяти}
\scnexplanation{Под \textbf{\textit{элементарным событием в sc-памяти}} понимается такое \textit{событие}, в результате выполнения которого изменяется состояние только одного \textit{sc-элемента}.}
\scnsubdividing{событие добавления sc-дуги, выходящей из заданного sc-элемента
;событие добавления sc-дуги, входящей в заданный sc-элемент;событие добавления sc-ребра, инцидентного заданному sc-элементу;событие удаления sc-дуги, выходящей из заданного sc-элемента;событие удаления sc-дуги, входящей в заданный sc-элемент;событие удаления sc-ребра, инцидентного заданному sc-элементу;событие удаления sc-элемента;событие изменения содержимого файла}

\scnheader{точечный процесс}
\scnidtf{атомарный процесс}
\scnidtf{условно мгновенный процесс}
\scnidtf{процесс, длительность которого в данном контексте считается несущественной (пренебрежимо малой)}

\scnheader{элементарный процесс}
\scnidtf{процесс, детализация которого на входящие в него подпроцессы в текущем контексте не производится}

\bigskip
\scnendstruct \scnendcurrentsectioncomment

\end{SCn}

\scsubsection[\scnmonographychapter{Глава 2.4. Формальные онтологии базовых классов сущностей - множеств, связей, отношений, параметров, величин, чисел, структур, темпоральных сущностей}]{Предметная область и онтология пространственных сущностей различных форм}
\label{sd_spatial_entities}

\scsubsection[\scnmonographychapter{Глава 2.4. Формальные онтологии базовых классов сущностей - множеств, связей, отношений, параметров, величин, чисел, структур, темпоральных сущностей}]{Предметная область и онтология материальных сущностей}
\label{sd_material_entities}

\scsubsection[\scnmonographychapter{Глава 2.3. Структура баз знаний интеллектуальных компьютерных систем нового поколения: иерархическая система предметных областей и онтологий. Онтологии верхнего уровня. Формализация понятий семантической окрестности, предметной области и онтологии в интеллектуальных компьютерных системах нового поколения}]{Предметная область и онтология семантических окрестностей}
\label{sd_sem_neigh}
\begin{SCn}

\scnsectionheader{\currentname}

\scnstartsubstruct

\scnheader{Предметная область семантических окрестностей}
\scniselement{предметная область}
\scnsdmainclasssingle{семантическая окрестность}
\scnsdclass{семантическая окрестность по инцидентным коннекторам;семантическая окрестность по выходящим дугам;семантическая окрестность по выходящим дугам принадлежности;семантическая окрестность по входящим дугам;семантическая окрестность по входящим дугам принадлежности;полная семантическая окрестность;базовая семантическая окрестность;специализированная семантическая окрестность;пояснение;примечание;правило идентификации экземпляров;терминологическая семантическая окрестность;теоретико-множественная семантическая окрестность;описание декомпозиции;логическая семантическая окрестность;спецификация типичного экземпляра;сравнительный анализ}

\scnheader{семантическая окрестность}
\scnidtf{sc-окрестность}
\scnidtf{семантическая окрестность, представленная в виде sc-текста}
\scnidtf{sc-текст, являющийся семантической окрестностью некоторого sc-элемента}
\scnidtf{спецификация заданной сущности, знак которой указывается как ключевой элемент этой спецификации}
\scnidtf{описание заданной сущности, знак которой указывается как ключевой элемент этой спецификации}
\scnsubset{знание}
\scnsuperset{семантическая окрестность по инцидентным коннекторам}
\scnsuperset{полная семантическая окрестность}
\scnsuperset{базовая семантическая окрестность}
\scnsuperset{специализированная семантическая окрестность}
\scnidtftext{пояснение}{\textit{знание}, являющееся спецификацией (описанием) некоторой \textit{сущности}, знак которой является \textit{ключевым знаком\scnrolesign} указанного \textit{знания}. Заметим, что каждая \textit{семантическая окрестность} в отличие от \textit{знаний} других видов имеет только один \textit{ключевой знак\scnrolesign} (ключевой элемент\scnrolesign, знак описываемой сущности\scnrolesign). Заметим также, что многообразие видов семантических окрестностей свидетельствует о многообразии семантических видов описаний различных сущностей.}
\scnnote{Понятие \textit{семантической окрестности}, как и любой другой \uline{семантически} выделяемый класс \textit{знаний}, абсолютно не зависит от \textit{языка представления знаний}. Этим \textit{языком} может быть не только \textit{SC-код} или другой \textit{формальный язык представления знаний} или даже \textit{естественный язык}, тексты которых в \textit{памяти ostis-системы} представляются в виде \textit{файлов}.}

\scnheader{семантическая окрестность по инцидентным коннекторам}
\scnsuperset{семантическая окрестность по выходящим дугам}
\scnsuperset{семантическая окрестность по входящим дугам}
\scnidtftext{пояснение}{вид \textit{семантической окрестности}, в которую входят все коннекторы, инцидентные заданному элементу, а также все элементы, инцидентные указанным коннекторам.}

\scnheader{семантическая окрестность по выходящим дугам}
\scnsuperset{семантическая окрестность по выходящим дугам принадлежности}
\scnidtftext{пояснение}{вид \textit{семантической окрестности}, в которую входят все дуги, выходящие из заданного sc-элемента и вторые компоненты этих дуг. Также указывается факт принадлежности этих дуг каким-либо отношениям.}

\scnheader{семантическая окрестность по выходящим дугам принадлежности}
\scnidtftext{пояснение}{вид \textit{семантической окрестности}, в которую входят все дуги принадлежности, выходящие из заданного \textit{sc-элемента}, а также их вторые компоненты. При необходимости может указывается факт \textit{принадлежности} этих дуг каким-либо \textit{ролевым отношениям}.}

\scnheader{семантическая окрестность по входящим дугам}
\scnsuperset{семантическая окрестность по входящим дугам принадлежности}
\scnidtftext{пояснение}{вид \textit{семантической окрестности}, в которую входят все дуги, входящие в заданный sc-элемент, а также их первые компоненты. Также указывается факт принадлежности этих дуг каким-либо отношениям.}

\scnheader{семантическая окрестность по входящим дугам принадлежности}
\scnidtftext{пояснение}{вид \textit{семантической окрестности}, в которую входят все дуги принадлежности, входящие в заданный sc-элемент, а также их первые компоненты. При необходимости может указывается факт принадлежности этих дуг каким-либо ролевым отношениям.}

\scnheader{полная семантическая окрестность}
\scnidtf{полная спецификация некоторой описываемой сущности}
\scnidtftext{пояснение}{вид \textit{семантической окрестности}, включающий описание всех связей описываемой сущности. 

Структура полной семантической окрестности определяется прежде всего семантической типологией описываемой сущности. 

Так, например, для \textit{понятия} в \textit{полную семантическую окрестность} необходимо включить следующую информацию (при наличии):
\begin{scnitemize}
    \item варианты идентификации на различных внешних языках (sc-идентификаторы);
    \item принадлежность некоторой \textit{предметной области} с указанием роли, выполняемой в рамках этой предметной области;
    \item теоретико-множественные связи заданного \textit{понятия} с другими \textit{sc-элементами};
    \item определение или пояснение;
    \item высказывания, описывающие свойства указанного \textit{понятия};
    \item задачи и их классы, в которых данное \textit{понятие} является ключевым;
    \item описание типичного примера использования указанного \textit{понятия};
    \item экземпляры описываемого \textit{понятия}.
\end{scnitemize}
Для понятия, являющегося отношением дополнительно указываются:
\begin{scnitemize}
    \item домены;
    \item область определения;
    \item схема отношения;
    \item классы отношений, которым принадлежит описываемое отношение.
\end{scnitemize}
}

\scnheader{базовая семантическая окрестность}
\scnidtf{минимально достаточная семантическая окрестность}
\scnidtf{минимальная спецификация описываемой сущности}
\scnidtf{сокращенная спецификация описываемой сущности}
\scnidtf{основная семантическая окрестность}
\scnidtftext{пояснение}{вид \textit{семантической окрестности}, содержащий минимальную (краткую) информацию об описываемой сущности

Структура базовой семантической окрестности определяется прежде всего семантической типологией описываемой сущности. 

Так, например, для \textit{понятия} в базовую семантическую окрестность необходимо включить следующую информацию (при наличии): 
\begin{scnitemize}
    \item варианты идентификации на различных внешних языках (sc-идентификаторы);
    \item принадлежность некоторой \textit{предметной области} с указанием роли, выполняемой в рамках этой предметной области;
    \item \textit{определение} или пояснение.
\end{scnitemize}
Для \textit{понятия}, являющегося \textit{отношением} дополнительно указываются:
\begin{scnitemize}
    \item \textit{домены};
    \item \textit{область определения};
    \item описание типичного примера связки указанного отношения (спецификация типичного экземпляра).
\end{scnitemize}
}

\scnheader{специализированная семантическая окрестность}
\scnsuperset{пояснение}
\scnsuperset{примечание}
\scnsuperset{правило идентификации экземпляров}
\scnsuperset{терминологическая семантическая окрестность}
\scnsuperset{теоретико-множественная семантическая окрестность}
\scnsuperset{логическая семантическая окрестность}
\scnsuperset{описание типичного экземпляра}
\scnsuperset{описание декомпозиции} 
\scnidtftext{пояснение}{вид \textit{семантической окрестности}, набор связей для которой уточняется отдельно для каждого типа такой окрестности.}

\scnheader{пояснение}
\scnidtf{sc-пояснение}
\scnidtftext{пояснение}{знак sc-текста, поясняющего описываемую сущность.}

\scnheader{примечание}
\scnidtf{sc-примечание}
\scnidtftext{пояснение}{знак sc-текста, являющегося примечанием к описываемой сущности. В примечании обычно описываются особые свойства и исключения из правил для описываемой сущности.}

\scnheader{правило идентификации экземпляров}
\scnidtf{правило идентификации экземпляров заданного класса}
\scnidtftext{пояснение}{sc-текст являющийся описанием правил построения идентификаторов элементов заданного класса.}

\scnheader{терминологическая семантическая окрестность}
\scnidtftext{пояснение}{\textit{семантическая окрестность}, описывающая внешнюю идентификацию указанной сущности, т.е. её sc-идентификаторы}

\scnheader{теоретико-множественная семантическая окрестность}
\scnidtftext{пояснение}{описание связи описываемого множества с другими множествами с помощью теоретико-множественных отношений}

\scnheader{описание декомпозиции}
\scnidtf{\textit{семантическая окрестность}, описывающая декомпозицию некоторой сущности}
\scnidtftext{пояснение}{\textit{семантическая окрестность}, описывающая декомпозицию некоторой сущности на её части}

\scnheader{логическая семантическая окрестность }
\scnidtftext{пояснение}{\textit{семантическая окрестность}, описывающая семейство высказываний, описывающих свойства данного \textit{понятия} или какого-либо конкретного экземпляра некоторого понятия}

\scnheader{спецификация типичного экземпляра}
\scnidtf{описание типичного экземпляра заданного класса}
\scnidtftext{пояснение}{sc-текст являющийся описанием типичного примера рассматриваемого класса.}

\scnheader{сравнительный анализ}
\scnidtftext{пояснение}{описание сравнения некоторой сущности с другими аналогичными сущностями}

\scnheader{сравнение}
\scnidtftext{пояснение}{описание сравнения (сходств и отличий) двух сущностей, которые заданы \textit{парой} (двухмощными множествами), которому принадлежат знаки обеих сравниваемых сущностей}

\scnheader{семантическая окрестность}
\scnnote{Всему классу \textit{семантических окрестностей} и всем подклассам этого \textit{класса}, а так же всем другим классам \textit{знаний}, ставятся в соответствие \textit{бинарные ориентированые отношения}, вторыми \textit{доменами} которых являются указанные \textit{классы} и объединение которых является \textit{обратным отношением} для \textit{отношения} \scnqq{быть ключевым знаком\scnrolesign}. Эти отношения не следует причислять к основным отношениям, т.к. они вместе с выделенными классами семантических окрестностей привносят дополнительную логическую эквивалентность в базу знаний.}

\scnheader{семантическая окрестность}
\scnrelfromlist{отношение, заданное на}{семантическая окрестность*
	;семантическая окрестность по инцидентным коннекторам*
	;полная семантическая окрестность*
	;базовая семантическая окрестность*
	;специализированная синтетическая окрестность*
	;и т.д.
}
\scnnote{Понятие семантической окрестности, дополненное уточнением таких понятий, как семантическое расстояние между знаками (семантическая близость знаков), радиус семантической окрестности, является перспективной основой для исследования свойств смыслового пространства.}

\bigskip
\scnendstruct \scnendcurrentsectioncomment

\end{SCn}

\scsubsection[\scnmonographychapter{Глава 2.3. Структура баз знаний интеллектуальных компьютерных систем нового поколения: иерархическая система предметных областей и онтологий. Онтологии верхнего уровня. Формализация понятий семантической окрестности, предметной области и онтологии в интеллектуальных компьютерных системах нового поколения}]{Предметная область и онтология предметных областей}
\label{sd_sd}
\begin{SCn}

\scnsectionheader{\currentname}

\scnstartsubstruct

\scnreltovector{конкатенация сегментов}{Что такое предметная область;Роли знаков, входящих в состав предметных областей;Типология предметных областей и отношения, заданных на множестве предметных областей;Что такое sc-язык}

\scnheader{Предметная область предметных областей}
\scnidtf{Предметная область, объектами исследования которой являются предметные области}
\scnexplanation{В состав \textbf{\textit{Предметной области предметных областей}} входят структурные спецификации всех \textit{предметных областей}, входящих в состав базы знаний \textit{ostis-системы}, в том числе, самой \textbf{\textit{Предметной области предметных областей}}. Таким образом, \textbf{\textit{Предметная область предметных областей}} является, во-первых, \textit{рефлексивным множеством}, во-вторых, рефлексивной предметной областью, то есть \textit{предметной областью}, одним из объектов исследования которой является она сама.}
\scniselement{рефлексивное множество}
\scnsdmainclasssingle{предметная область}

\scnsdclass{статическая предметная область;динамическая предметная область;понятие;sc-язык}

\scnsdrelation{понятие предметной области\scnrolesign ;исследуемое понятие\scnrolesign ;максимальный класс объектов исследования\scnrolesign ;немаксимальный класс объектов исследования\scnrolesign ;исследуемый класс первичных элементов\scnrolesign ;исследуемое отношение\scnrolesign ;класс исследуемых структур\scnrolesign ;понятие, исследуемое в дочерней предметной области\scnrolesign ;понятие, исследуемое в материнской предметной области\scnrolesign ;вспомогательное понятие\scnrolesign ;дочерняя предметная область*;дочерняя предметная область по классу первичных элементов*;дочерняя предметная область по исследуемым отношениям*;предметная область sc-языка*}

\scnsegmentheader{Что такое предметная область}

\scnstartsubstruct

\scnheader{предметная область}
\scnidtf{sc-модель предметной области}
\scnidtf{sc-текст предметной области}
\scnidtf{sc-граф предметной области}
\scnidtf{представление предметной области в \textit{SC-коде}}
\scnsubset{знание}
\scnsubset{бесконечное множество}
\scnexplanation{\textbf{\textit{предметная область}} -- это результат интеграции (объединения) частичных семантических окрестностей, описывающих все исследуемые сущности заданного класса и имеющих одинаковый (общий) предмет исследования (то есть один и тот же набор отношений, которым должны принадлежать связки, входящие в состав интегрируемых семантических окрестностей).


\textbf{\textit{предметная область}} -- \textit{структура}, в состав которой входят:
\begin{scnitemize}
\item \textnormal{основные исследуемые (описываемые) объекты -- первичные и вторичные;}
\item \textnormal{различные классы исследуемых объектов;}
\item \textnormal{различные связки, компонентами которых являются исследуемые объекты (как первичные, так и вторичные), а также, возможно, другие такие связки -- то есть связки (как и объекты исследования) могут иметь различный структурный уровень;}
\item \textnormal{различные классы указанных выше связок (то есть отношения);}
\item \textnormal{различные классы объектов, не являющихся ни объектами исследования, ни указанными выше связками, но являющихся компонентами этих связок.}
\end{scnitemize}


При этом все классы, объявленные исследуемыми понятиями, должны быть полностью представлены в рамках данной предметной области вместе со своими элементами, элементами элементов и т.д. вплоть до терминальных элементов.


Можно говорить о типологии \textbf{\textit{предметных областей}} по разным структурным признакам:
\begin{scnitemize}
    \item наличие метасвязей;
    \item наличие исследуемых структур, входящих в состав предметной области;
    \item наличие исследуемых (смежных, дополнительных) объектов, которых исследуются в других предметных областях;
\end{scnitemize}


Понятие \textbf{\textit{предметной области}} является важнейшим методологическим приемом, позволяющим выделить из всего многообразия исследуемого Мира только определенный класс исследуемых сущностей и только определенное семейство отношений, заданных на указанном классе. То есть осуществляется локализация, фокусирование внимания только на этом, абстрагируясь от всего остального исследуемого Мира.


Во всем многообразии \textbf{\textit{предметных областей}} особое место занимают
\begin{scnitemize}
    \item \textit{Предметная область предметных областей}, объектами исследования которой являются всевозможные \textbf{\textit{предметные области}}, а предметом исследования -- всевозможные \textit{ролевые отношения}, связывающие предметные области с их элементами, отношения, связывающие предметные области между собой, отношение, связывающее предметные области с их онтологиями
    \item \textit{Предметная область сущностей}, являющаяся предметной областью самого высокого уровня и задающая базовую семантическую типологию \textit{sc-элементов}(знаков, входящих в тексты \textit{SC-кода})
    \item Семейство \textbf{\textit{предметных областей}}, каждая из которых задает семантику и синтаксис некоторого \textit{sc-языка}, обеспечивающего представление онтологий соответствующего вида (например, \textit{теоретико-множественных онтологий}, \textit{логических онтологий}, \textit{терминологических онтологий}, \textit{онтологий задач и способов их решения} и т.д.)
    \item Семейство \textbf{\textit{предметных областей}} верхнего уровня, в которых классами объектов исследования являются весьма \scnqq{крупные} классы сущностей. К таким классам, в частности
    
    \begin{scnitemizeii}
        \item класс всевозможных \textit{материальных сущностей},
        \item класс всевозможных \textit{множеств},
        \item класс всевозможных \textit{связей},
        \item класс всевозможных \textit{отношений},
        \item класс всевозможных \textit{структур},
        \item класс всевозможных \textit{временных (временно существующих, непостоянных сущностей) сущностей},
        \item класс всевозможных \textit{действий} (акций),
        \item класс всевозможных \textit{параметров} (характеристик),
        \item класс \textit{знаний} всевозможного вида 
        \item и т.п.
    \end{scnitemizeii}
\end{scnitemize}


Каждой \textbf{\textit{предметной области}} можно поставить в соответствие:
\begin{scnitemize}
    \item семейство соответствующих ей \textit{онтологий} разного вида;
    \item некий язык (в нашем случае -- язык, построенный на основе \textit{SC-кода}), тексты которого представляют различные фрагменты соответствующей предметной области
\end{scnitemize}


Указанные языки будем называть \textit{sc-языками}. Их синтаксис и семантика полностью задается \textit{SС-кодом} и \textit{онтологией} соответствующей \textbf{\textit{предметной области}}. Очевидно, что в первую очередь нас должны интересовать те \textit{sc-языки}, которые соответствуют \textbf{\textit{предметным областям}}, имеющим общий (условно говоря, предметно независимый) характер. К таким предметным областям, в частности, относятся:
\begin{scnitemize}
    \item \textit{Предметная область множеств}, описывающая множества и различные связи между ними
    \item \textit{Предметная область отношений и соответствий}
    \item \textit{Предметная область структур} (в частности, графовых)
    \item \textit{Предметная область чисел и числовых структур}
    \item и т.д
\end{scnitemize}


Каждому типу знаний можно поставить в соответствие предметную область, которая является результатом интеграции всех знаний данного типа. Эти знания и становятся объектами исследования в рамках указанной предметной области.


Понятие \textbf{\textit{предметной области}} может рассматриваться как обобщение понятия алгебраической системы. При этом семантическая структура базы знаний может рассматриваться как иерархическая система различных \textbf{\textit{предметных областей}}.
}
\scnidtf{система связей некоторого множества объектов исследования, \uline{ключевыми} элементами которой являются:
	\begin{scnitemize}
	\item классы (точнее, знаки классов) объектов исследования (объектов, описываемых этой предметной областью);
	\item конкретные объекты исследования, обладающие особыми свойствами;
	\item классы связей, входящих в состав рассматриваемой системы -- отношения, заданные на множестве элементов рассматриваемой системы;
	\item параметры, заданные на множестве элементов рассматриваемой системы;
	\item классы структур, являющихся фрагментами рассматриваемой системы.
	\end{scnitemize}}
\scnidtf{структура, представляющая собой множество связей (точнее, знаков связей) и соответствующее множество компонентов этих связей, к числу которых относится:
	\begin{scnitemize}
	\item элементы (экземпляры) некоторых заданных классов \uline{объектов исследования} (первичных исследуемых сущностей);
	\item сами связи, входящие в состав указанной структуры;
	\item введенные классы объектов исследования;
	\item введенные отношения (классы связей);
	\item введенные параметры (классы классов эквивалентных сущностей);
	\item значения параметров (и, в частности, величины для измеряемых параметров);
	\item введенные структуры, являющиеся фрагментами (подструктурами) рассматриваемой структуры;
	\item введенные классы подструктур рассматтриваемой структуры.
	\end{scnitemize}}
\scnnote{Выделяемые в рамках \textit{базы знаний} интеллектуальной системы \textit{предметные области} и соответствующие им \textit{онтологии} -- это, своего рода, семантические страты, кластеры, позволяющие \scnqq{разложить} все хранимые в памяти \textit{знания} по \scnqq{семантическим полочкам} при наличии четких критериев, позволяющих \uline{однозначно} определить то, на какой \scnqq{полочке} должны находиться те или иные \textit{знания}}
\scnnote{Существуют предметные области, в которых основным исследуемым понятием является множество всевозможных связей между экземплярами понятий, исследуемых в других предметных областях. Так, например, можно ввести Предметную область треугольников, Предметную область окружностей, а также Предметную область связей между треугольниками и окружностями.}

\bigskip
\scnendstruct \scnendsegmentcomment{Что такое предметная область}

\scnsegmentheader{Роли знаков, входящих в состав предметной области}

\scnstartsubstruct

\scnheader{роль элемента предметной области}
\scnidtf{ролевое отношения, связывающее предметные области с их ключевыми знаками}
\scnidtf{роль ключевого элемента (знака ключевой сущностей) предметной области}
\scnidtf{роль ключевого знака предметной области}
\scnhaselement{класс объектов исследования\scnrolesign}
\scnhaselement{максимальный класс объектов исследования\scnrolesign}
\scnhaselement{ключевой объект исследования\scnrolesign}
\scnhaselement{понятие, используемое в предметной области\scnrolesign}
\scnhaselement{первичный исследуемый элемент предметной области\scnrolesign}
\scnhaselement{вторичный исследуемый элемент предметной области\scnrolesign}
\scnhaselement{неисследуемый элемент предметной области\scnrolesign}


\scnheader{класс объектов исследования\scnrolesign}
\scnidtf{быть классом \uline{первичных} (для данной предметной области) объектов исследования\scnrolesign}
\scnnote{Понятие \uline{первичного} объекта исследования для предметной области является понятием \uline{относительным} и абсолютно не зависит от типа и уровня сложности этого объекта. Само исследование (спецификация) таких первичных исследуемых объектов осуществляется:
	\begin{scnitemize}
	\item путем введения различных классов объектов исследования, которым эти объекты принадлежат;
	\item путем введения различных связок из первичных объектов исследования и различных классов таких связок (отношений), которым принадлежат введенные связки;
	\item путем введения таких классов первичных объектов исследования, которые являются значениями вводимых параметров;
	\item путем введения различных структур, состоящих из первичных объектов исследования, из связок таких объектов, из введенных отношений и классов первичных объектов, из введенных параметров и значений этих параметров, и путем введения различных классов таких структур;
	\item путем введения различных связок из вторичных объектов исследования (т.е. из связок и структур) и путем введения различных классов таких связок;
	\item и далее можно переходить к объектам исследования более высокого уровня сложности, к параметрам, элементами значений которых являются такие объекты, а также к структурам, элементами которых являются объекты такого уровня и, соответственно, к классам таких структур.
	\end{scnitemize}}

\scnrelfrom{второй домен}{класс}
\scnaddlevel{1}
	\scnsuperset{\scnmakesetlocal{множество
			;отношение\\
			\scnaddlevel{1}
			\scnsubset{множество}
			\scnaddlevel{-1}
			;параметр\\
			\scnaddlevel{1}
			\scnsubset{класс классов}
			\scnaddlevel{-1}
			;значение параметра\\
			\scnaddlevel{1}
			\scnsubset{класс}
			\scnaddlevel{-1}
			;структура\\
			\scnaddlevel{1}
			\scnsubset{множество}
			\scnaddlevel{-1}
			;темпоральная сущность
			;темпоральная сущность базы знаний ostis-системы
			;семантическая окрестность
			;предметная область
			;онтология
			;логическая формула
			;действие
			;задача
			;информационная конструкция
			;язык
			;sc-конструкция
			;кибернетическая система
			;интеллектуальная компьютерная система
			;знание
			;база знаний
			;решатель задач интеллектуальной компьютерной системы
			;интерфейс интеллектуальной компьютерной системы
			;компьютерная система, основанная на смысловом представлении информации
			;смысловое представление информации
			;многоагентная модель решения задач, основанная на смысловом представлении информации
			;логико-семантическая модель интерфейсов компьютерных систем, основанных на смысловом представлении информации
			;решатель задач ostis-системы
			;действие, выполняемое ostis-системой
			;задача, решаемая ostis-системой
			:план решения задачи, реализуемый ostis-системой
			;протокол решения задачи, реализованный ostis-системой
			;метод решения класса задач, реализуемый ostis-системой
			;sc-агент\\
			\scnaddlevel{1}
			\scnidtf{внутренний агент ostis-системы, осуществляющий выполнение некоторого вида действий в памяти ostis-системы}
			\scnsuperset{sc-агент обработки информации в памяти ostis-системы}
			\scnsuperset{sc-агент управления внешними действиями ostis-системы}
			\scnaddlevel{-1}
			;Базовый язык программирования ostis-систем\\
			\scnaddlevel{1}
			\scnidtf{Язык SCP}	
			\scnaddlevel{-1}
			;искусственная нейронная сеть
			;интерфейс ostis-системы
			;интерфейсное действие пользователя ostis-системы
			;sc-агент интерфейса ostis-системы
			;естественный язык
			;базовый интерпретатор логико-семантических моделей ostis-систем
			;базовый интерпретатор логико-семантических моделей ostis-систем, реализованный программно на современных компьютерах
			;семантический ассоциативный компьютер
			;обучение пользователей ostis-систем
			;ostis-система персональной адаптивной поддержки всех видов деятельности пользователя
			;ostis-система управления рецептурным производством
			;ostis-система, реализующая интеллектуальный портал научно-технических знаний}}
		\scnaddlevel{1}
\scnnote{Здесь приведено семейство тех \textit{классов объектов исследования}, для которых в текущей версии \textit{Стандарта OSTIS} представлены соответствующие \textit{предметные области}. Очевидно, что это семейство должно быть существенно расширено и включить в себя, например, такие \textit{классы} сущностей, как:
	\begin{scnitemize}
	\item материальная сущность
	\item вещество
	\item физическое поле
	\item персона
	\item пространственная сущность
	\item юридическое лицо
	\item предприятие
	\item географический объект
	\item и многие другие
	\end{scnitemize}}
\scnaddlevel{-2}
\scnnote{Особого внимания требуют те \textit{классы объектов исследования}, которые носят наиболее общий характер  которым соответствуют \textit{предметные области и онтологии} \uline{высокого уровня}. Здесь важна продуманная система декомпозиции всего множества окружающих нас сущностей на иерархическую систему \textit{классов объектов исследования}, которой соответствует иерархическая система \textit{предметных областей и онтологий}, определяющая направления \uline{наследования свойств} исследуемых объектов.}

\scnheader{максимальный класс объектов исследования\scnrolesign}
\scnidtf{класс объектов исследования, для которого \uline{в заданной} (!) предметной области отсутствует другой класс объектов исследования, который был бы его надмножеством\scnrolesign}
\scnnote{В некоторых предметных областях может быть \uline{несколько} максимальных классов объектов исследования}


\scnheader{ключевой объект исследования\scnrolesign}
\scnidtf{особый объект исследования\scnrolesign}
\scnidtf{быть знаком особого исследуемого объекта в рамках заданной предметной области\scnrolesign}
\scnidtf{объект исследования, обладающий особыми свойствами\scnrolesign}
\scnhaselementrole{пример}{$\langle$Предметная область чисел; Нуль$\rangle$}
	\scnaddlevel{1}
	\scnnote{Особыми свойствами Числа \textit{Нуль} являются:
		\begin{scnitemize}
		\item Результатом сложения Числа \textbf{\textit{Нуль}} с любым числом \textbf{\textit{x}} является число \textbf{\textit{x}};
		\item Результатом умножения Числа \textbf{\textit{Нуль}} на любое число является Число \textbf{\textit{Нуль}}
		\end{scnitemize}}
	\scnaddlevel{-1}
\scnhaselement{$\langle$Предметная область чисел; Единица$\rangle$}
\scnhaselement{$\langle$Предметная область чисел; Число Пи$\rangle$}
\scnhaselement{$\langle$Предметная область чисел; Число Е$\rangle$}

\scnheader{ключевой элемент предметной области\scnrolesign}
\scnidtf{входящий в состав предметной области знак ключевой сущности\scnrolesign}
\scnsubdividing{понятие, используемое в предметной области\scnrolesign
;ключевой объект исследования\scnrolesign \\
	\scnaddlevel{1}
	\scnidtf{знак ключевого объекта исследования\scnrolesign}
	\scnaddlevel{-1}}


\scnheader{понятие, используемое в предметной области\scnrolesign}
\scnidtf{понятие, используемое в заданной предметной области не в качестве одного из объектов исследования, а в качестве \uline{ключевого} понятия\scnrolesign}
\scnsubset{используемое понятие\scnrolesign}
	\scnaddlevel{1}
	\scnidtf{понятие, используемое в sc-знании\scnrolesign}
	\scnsubset{используемое понятие*}
		\scnaddlevel{1}
		\scnidtf{понятие, используемое в знании, которое может быть представлено не только в SC-коде*}
		\scnaddlevel{-1}
	\scnaddlevel{-1}
\scnnote{Уточнение характера использования понятия в предментной области осуществляется по трем признакам:
	\begin{scnitemize}
	\item семантический тип используемого понятия;
	\item полнота вхождения элементов понятия в данную предметную область;
	\item наличие первого упоминания понятия;
	\item наличие определения понятия или объявления его неопределяемостис подробным пояснением и примерами;
	\item наличие исследования понятия.	
	\end{scnitemize}}
\scnrelfrom{разбиение}{семантический тип используемого понятия}
	\scnaddlevel{1}
	\scneqtoset{класс объектов исследования\scnrolesign
;отношение, используемое в предметной области\scnrolesign
;параметр, используемый в предметной области\scnrolesign
;класс структур, используемый в предметной области\scnrolesign}
	\scnaddlevel{-1}
\scnrelfrom{разбиение}{полнота вхождения элементов понятия в данную предметную область}
	\scnaddlevel{1}
	\scneqtoset{используемое понятие, все элементы которого входят в данную предметную область\scnrolesign \\
	\scnaddlevel{1}
	\scnnote{Для каждого используемого отношения в предметную область здесь должны входить не только знаки связок, но и все связки целиком с их компонентами}
	\scnaddlevel{-1}
;используемое понятие, не все элементы которого входят в данную предметную область\scnrolesign}
	\scnaddlevel{-1}
\scnrelfrom{разбиение}{наличие первого упоминания понятия}
	\scnaddlevel{1}
	\scneqtoset{понятие, вводимое в данной предметной области\scnrolesign
;понятие, которое в данной предметной области используется, но не вводится\scnrolesign}
	\scnaddlevel{1}
	\scnnote{Будем считать, что понятие вводится в данной предметной области в том и только в том случае, если ни в одной предметной области более высокого уровня это понятие не используется. Т.е. речь идет о первом упоминании этого понятия в рамках последовательности предметных областей от родительских к дочерним}
	\scnaddlevel{-1}
	\scnaddlevel{-1}
\scnrelfrom{разбиение}{наличие определения понятия или объявления его неопределяемости с подробным пояснением и примерами}
	\scnaddlevel{1}
	\scneqtoset{понятие, которое в данной предметной области определено или объявлено как неопределяемое
;понятие, которое в данной предметной области не имеет ни определения, ни указания факта его неопределяемости}
	\scnaddlevel{-1}
\scnrelfrom{разбиение}{наличие исследования понятия}
	\scnaddlevel{1}
	\scneqtoset{понятие, исследуемое в данной предметной области\scnrolesign
;понятие, которое в данной предметной области испольуется, но не исследуется\scnrolesign}
	\scnaddlevel{-1}
\scnnote{Понятие, используемое в базе знаний, может быть введено (впервые упомянуто) в одной предметной области, определено в другой, а исследоваться -- в третьей}


\scnheader{первичный исследуемый элемент предметной области\scnrolesign}
\scnidtf{знак первичного объекта исследования в рамках заданной предметной области\scnrolesign}


\scnheader{вторичный исследуемый элемент предметной области\scnrolesign}
\scnidtf{знак вторичного объекта исследования в рамках предметной области\scnrolesign}
\scnsuperset{связка элементов предметной области\scnrolesign}
	\scnaddlevel{1}
\scnsuperset{связка первичных элементов предметной области\scnrolesign}
\scnsuperset{метасвязка элементов предметной области\scnrolesign}
	\scnaddlevel{1}
\scnsuperset{метасвязка, в число компонентов которой входят связки элементов предметной области\scnrolesign}
\scnsuperset{метасвязка, в число компонентов которой входят классы элементов предметной области\scnrolesign}
\scnsuperset{метасвязка, в число компонентов которой входят структуры элементов предметной области\scnrolesign}
	\scnaddlevel{-1}
	\scnaddlevel{-1}
\scnsuperset{класс элементов предметной области\scnrolesign}
		\scnaddlevel{1}
\scnsuperset{класс первичных элементов предметной области\scnrolesign}
\scnsuperset{класс связок элементов предметной области\scnrolesign}
\scnsuperset{класс классов элементов предметной области\scnrolesign}
\scnsuperset{класс структур элементов предметной области\scnrolesign}
	\scnaddlevel{-1}
\scnsuperset{структура элементов предметной области\scnrolesign}
		\scnaddlevel{1}
\scnsuperset{тривиальная структура первичных элементов предметной области\scnrolesign}
\scnsuperset{структура, в число подмножеств которой входят связки элементов предметной области вместе со своими компонентами\scnrolesign}
\scnsuperset{структура, в число подмножеств которой входят классы элементов предметной области вместе со своими знаками\scnrolesign}
\scnsuperset{структура, в число подмножеств которой входят другие структуры вместе со своими знаками\scnrolesign}
	\scnaddlevel{-1}


\scnheader{неисследуемый элемент предметной области\scnrolesign}
\scnidtf{вспомогательный элемент предметной области, исследуемый в другой (смежной) предметной области\scnrolesign}
\scnnote{С помощью неисследуемых элементов предметной области описываются и исследуются различные вида связи между элементами, исследуемыми в данной \textit{предметной области} с элементами, исследуемыми в других \textit{предметных областях}. При этом \textit{связки}, компонентами которых являются как исследуемые, так и неисследуемые элементы данной \textit{предметной области} считаются \uline{исследуемыми} связками этой \textit{предметной области}. Примерами неисследуемых элементов, напримр, геометрической \textit{предметной области} являются \textit{числа}, являющиеся \textit{значениями величин} таких \textit{параметров}, как \textit{расстояние}\scnsupergroupsign, \textit{длина}\scnsupergroupsign, \textit{площадь}\scnsupergroupsign, \textit{объем}\scnsupergroupsign, а также различные числовые \textit{отношения} (\textit{сложение}*, \textit{умножение}*, \textit{возведение в степень}*), теоретико-множественные \textit{отношения} (\textit{включение}*, \textit{объединение}*, \textit{пересечение}*, \textit{принадлежность}*)}

\newpage
\scnheader{понятие}
\scnidtf{концепт}
\scnidtf{класс сущностей, который входит в состав по крайней мере одной предметной области в качестве (в роли) ключевого исследуемого понятия}
\scnnote{Семейство всех введенных понятий -- это, своего рода, семантическая система координат, позволяющая специфицировать всевозможные сущности в смысловом пространстве.}
\scnidtf{класс сущностей, который по крайней мере в одной \textit{предметной области} \scnqq{объявлен} как \textit{понятие} (вводимое, исследуемое или вспомогательное)}
\scnnote{Каждому \textit{понятию} соответствует по крайней мере одна \textit{предметная область}, в которой это понятие является \textit{исследуемым понятием} и в которой рассматриваются основные характеристики этого \textit{понятия}. Если же в какой-либо \textit{предметной области} необходимо рассмотреть дополнительные связи этого \textit{понятия} с другими \textit{понятиями}, то оно объявляется как \textit{вспомогательное понятия}\scnrolesign .}
\scnidtf{Второй домен Отношения \textit{используемое понятие}*}
\scnrelto{второй домен}{используемое понятие*}
\scnidtf{класс сущностей (класс связок (в т.ч. отношение), класс классов (в т.ч. параметр), класс структур), который по крайней мере в одной \textit{предметной области} является \textit{используемым понятием}\scnrolesign}

\bigskip
\scnendstruct \scnendsegmentcomment{Роли знаков, входящих в состав предметной области}

\scnsegmentheader{Типология предметных областей и отношения, заданные на множестве предметных областей}
\scnstartsubstruct

\scnheader{предметная область}
\scnsubdividing{статическая предметная область\\
\scnaddlevel{1}
\scnidtf{стационарная предметная область}
\scnidtf{\textit{предметная область}, в которой связи между сущностями, входящими в ее состав, не зависят от времени (не меняются во времени), элементами \textbf{\textit{статической предметной области}} не могут быть \textit{временные сущности}}
\scnaddlevel{-1}
;квазистатическая предметная область\\
\scnaddlevel{1}
\scnidtf{\textit{предметная область}, решение задач в которой не требует учета темпоральных свойств объектов исследования} 
\scnaddlevel{-1}
;динамическая предметная область\\
\scnaddlevel{1}
\scnidtf{нестационарная предметная область} 
\scnidtf{\textit{предметная область}, которая описывает изменение состояния (в том числе внутренней структуры) объектов исследования и/или изменение конфигурации связей между объектами исследования} 
\scnidtf{\textit{предметная область}, в которой некоторые связи между сущностями, входящими в ее состав, меняются со временем (то есть носят ситуационный, нестационарный характер, другими словами, являются \textit{временными сущностями})} 
\scnaddlevel{-1}
}

\scnsubdividing{первичная предметная область\\
\scnaddlevel{1}
\scnidtf{\textit{предметная область}, объектами исследования которой являются \uline{внешние} сущности (обозначаемые первичными \textit{sc-элементами})}
\scnaddlevel{-1}
;вторичная предметная область\\
\scnaddlevel{1}
\scnidtf{метапредметная область} 
\scnidtf{\textit{предметная область}, объектами исследования которой являются \textit{sc-множества} (отношения, параметры, структуры, классы структур, знания, языки и др.)} 
\scnaddlevel{-1}
}

\scnnote{Во всем многообразии предметных областей \uline{особое} местро занимают:
\begin{scnitemize}
		\item \textbf{\textit{Предметная область предметных областей}}, объектами исследования которой являются всевозможные предметные области, а предметом исследования являются -- всевозможные ролевые отношения, связывающие предметные области с их элементами, отношения, связывающие предметные области между собой, отношение, связывающее предметные области с их онтологиями;
		\item \textbf{\textit{Предметная область сущностей}}, являющаяся предметной областью самого высокого уровня и задающая базовую семантическую типологию sc-элементов (знаков, входящих в тексты SC-кода);
		\item Семейство \textit{предметных областей}, каждая из которых задает семантику и синтаксис некоторого \textit{sc-языка}, обеспечивающего представление \textit{\uline{онтологий}} соответствующего вида (например, теоретико множественных онтологий терминологических онтологий);
		\item Семейство \textit{предметных областей} \uline{верхнего уровня}, в которых классами объектов исследования являются весьма \scnqq{крупные} классы сущностей. К таким классам, в частности, относятся: 
		\begin{scnitemizeii}
			\item класс всевозможных материальных сущностей,
			\item класс всевозможных множеств,
			\item класс всевозможных связей,
			\item класс всевозможных отношений,
			\item класс всевозможных структур,
			\item класс всевозможных темпоральных (нестационарных) сущностей,
			\item класс всевозможных действий (воздествий, акций),
			\item класс всевозможных параметров (характеристик),
			\item класс знаний всевозможного вида и т.п.;
		\end{scnitemizeii}
		\item Предметные области абстрактных пространств (в том числе предметные области метрических пространств). Примерами абстрактного пространства являются Евклидово пространство геометрических точек и фигур, пространство всевозможных множеств, числовое пространство, SC-пространство (унифицированное смысловое пространство знаков всевозможных сущностей).
\end{scnitemize}
}

\scnheader{отношение, заданное на множестве предметных областей}
\scnhaselement{\scnkeyword{дочерняя предметная область*}}
\scnaddlevel{1}
\scnidtf{частная предметная область*}
\scnidtf{быть частной предметной областью*}
\scnidtf{близлежащий потомок предметной области*}
\scnidtf{сужение предметной области по классу объектов исследования*}
\scnidtf{предметная область, детализирующая описание одного из классов объектов исследования другой (более общей) предметной области*}
\scnidtf{предметная область, объединение классов объектов исследования которой является подмножеством объединения классов объектов исследования заданной предметной области*}
\scniselement{бинарное отношение}
\scniselement{ориентированное отношение}
\scniselement{неролевое отношение}
\scnsuperset{частная предметная область по классу первичных элементов*}
\scnsuperset{частная предметная область по исследуемым отношениям*}
\scnexplanation{\textit{дочерняя предметная область*} -- бинарное ориентированное отношение, с помощью которого задается иерархия предметных областей путем перехода от менее детального к более детальному рассмотрению соответствующих исследуемых понятий.}
\scnnote{Для любой \textit{предметной области} все свойства ее \textit{объектов исследования} \uline{наследуются} всеми ее \textit{дочерними предметными областями*}.}
\scnaddlevel{-1}
\scnhaselement{\scnkeyword{интеграция предметных областей*}}
\scnaddlevel{1}
\scnidtf{Отношение, связывающее заданное семейство предметных областей с предметной областью, которая является результатом их интеграции (это не только теоретико-множественное объединение заданных предметных областей, но и уточнение ролей ключевых понятий в интегрированной предметной области, поскольку одно и то же понятие в интегрируемых предметных областях может иметь разные роли).}
\scnaddlevel{-1}
\scnhaselement{\scnkeyword{изоморфность предметных областей*}}
\scnhaselement{\scnkeyword{гомоморфность предметных областей*}}

\scnheader{расширение семейства исследуемых отношений*}
\scnexplanation{Переход от одной предметной области к предметной области с тем же максимальным классомобъектов исследования, но с расширенным семейством отношений и, возможно, с расширенным семейством явно выделенных классов объектов исследования (подклассов максимального класса).}

\scnheader{переход к рассмотрению внутренней структуры объектов исследования*}
\scnexplanation{Переход от рассмотрения внешних связей объектов исследования к рассмотрению их \scnqq{внутренней} структуры путем декомпозиции исследуемых объектов на части и путем включения в число исследуемых объектов тех, которые являются указанными частями.}

\scnheader{переход к рассмотрению структур из объектов исследования*}
\scnexplanation{Переход от описаниязаданного класса исследуемых объектов к описанию класса всевозможных множеств, элементами которых являются указанные объекты (например, переход от предметной области геометрических точек к предметной области геометрических фигур).}

\bigskip

\scnendstruct \scnendsegmentcomment{Типология предметных областей и отношения, заданные на множестве предметных областей}

\scnsegmentheader{Что такое sc-язык}
\scnstartsubstruct

\scnheader{sc-язык}
\scnidtf{максимальное множество текстов SC-кода, являющихся фрагментами соответствующей предметной области (точнее, ее sc-модели)}
\scnexplanation{\textbf{\textit{sc-язык}} --- это подъязык (подмножество) \textit{SC-кода}, ориентированный на представление \textit{sc-текстов}, являющихся фрагментами некоторой \textit{предметной области}. Таким образом, каждому \textbf{\textit{sc-языку}} взаимно однозначно соответствует некоторая \textit{предметная область} (точнее, sc-модель некоторой \textit{предметной области}).}

\scnheader{предметная область sc-языка*}
\scnidtf{предметная область заданного sc-языка*}
\scnidtf{быть предметной областью соответствующей заданному sc-языку*}
\scnidtf{sc-язык и соответствующая ему предметная область*}
\scniselement{бинарное отношение}
\scnexplanation{\textbf{\textit{предметная область sc-языка*}} - это бинарное ориентированное отношение, каждая связка которого связывает знак некоторого \textit{sc-языка} (первый компонент связки данного отношения) и знак соответствующей этому \textbf{\textit{sc-языку}} \textit{предметной области}.
}
\scnrelfrom{первый домен}{sc-язык}
\scnrelfrom{второй домен}{предметная область}

\scnheader{sc-язык}
\scnidtf{подъязык \textit{SC-кода}}
\scnexplanation{Множество языков, синтаксис каждого из которых полностью соответствует синтаксису \textit{SC-кода} (т.е. каждый из них является подъязыком SC-кода), а денотационная семантика каждого из них определяется интегральной (объединенной) онтологией  \textit{предметной области}, которая взаимно однозначно соответствует этому sc-языку.}
\scnnote{Каждой предметной области можно поставить в соответствие множество sc-текстов, которые являются фрагментами sc-моделей этой \textit{предметной области}, описывающими (специфицирующими) свойства объектов, исследуемых в указанной \textit{предметной области}. Указанное множество \textit{sc-текстов} будем называть \textit{sc-языком}, соответствующим указанной \textit{предметной области}. Очевидно, что \textit{предметных областей} и соответсвующих им \textit{sc-языков} существует неограниченное количество. Синтаксис и семантика \textit{sc-языков} полностью задается \textit{SC-кодом} и онтологией соответсвующей \textit{предметной области}. Очевидно, что в первую очередь нас должны интересовать те \textit{sc-языки}, которые соответствуют предметным областям, имеющим общий (условно говоря, предметно независимый) характер. К таким предметным областям, в частности, относятся:
\begin{scnitemize}
		\item \textit{Предметная область множеств}, описывающая множества и различные связи между ними
		\item \textit{Предметная область структур}
		\item \textit{Предметная область чисел и числовых структур}
		\item \textit{Предметная область отношений}
		\item \textit{Предметная область параметров, величин и шкал}
		\item \textit{Предметная область логических формул, высказываний и формальных теорий}
		\item и другие
\end{scnitemize}}

\scnnote{Особое место для всего семейства sc-языков занимают две предметные области и соответствующие им онтологии:
\begin{scnitemize}
		\item Предметная область SC-кода и онтология, определяющая его базовую денотационную семантику, задаваемую синтаксисом SC-кода
		\item Предметная область и онтология Базового универсального sc-языка
\end{scnitemize}}
\scnnote{Подчеркнем, что все sc-языки, кроме тех, которые задаются указанными двумя предметными областями и соответсвующими им онтологиями, являются \uline{специализированными} sc-языками.
}

\scnheader{Предметная область SC-кода и онтология, определяющая его базовую денотационную семантику, задаваемую синтаксисом SC-кода}
\scnexplanation{Указанная онтология SC-кода определяет денотационную семантику SC-кода путем семантической интерпретации
\uline{синтаксически} выделяемых классов sc-элементов. Т.е. данная денотационная семантика уточняет смысл \uline{только} тех классов sc-элементов, которые задаются \uline{синтаксически} путем \scnqq{приписывания} sc-элементам соответствующей \scnqq{синтаксической метки}.}

\scnheader{Предметная область и онтология Базового универсального sc-языка}
\scnexplanation{Указанная онтология Базового универсального sc-языка определяет денотационную семантику этого Базового универсального sc-языка и представляет собой онтологию \uline{верхнего уровня}, которая уточняет смысл понятий, лежащий в основе указанного sc-языка. В эту онтологию входит уточнение таких понятий, как материальная сущность, абстрактная сущность, число, пространство, множество, связь, класс, отношение, параметр, структура, класс структур, информационная конструкция, знание, статическая сущность, динамическая сущность, процесс, ситуация, действие, постоянная сущность, временная сущность, часть*, декомпозиция*, пространственная часть*, темпоральная часть* и т.д.}

\bigskip

\scnendstruct \scnendsegmentcomment{Что такое sc-язык}

\bigskip

\scnendstruct \scnendcurrentsectioncomment

\end{SCn}

\scsubsection[\scnmonographychapter{Глава 2.3. Структура баз знаний интеллектуальных компьютерных систем нового поколения: иерархическая система предметных областей и онтологий. Онтологии верхнего уровня. Формализация понятий семантической окрестности, предметной области и онтологии в интеллектуальных компьютерных системах нового поколения}]{Предметная область и онтология онтологий}
\label{sd_ontologies}
\begin{SCn}

\scnsectionheader{\currentname}

\scnstartsubstruct

\scnreltovector{конкатенация сегментов}{Что такое онтология;
	Типология онтологий предметной области;
	Понятие объединенной онтологии предметной области, понятие предметной области и онтологии;
	Отношения, заданные на множестве онтологий}

\scnheader{Предметная область \textit{онтологий}}
\scnidtf{Предметная область теории \textit{онтологий}}
\scnidtf{Предметная область, объектами исследования которой являются \textit{онтологии}}

\scniselement{предметная область}
\scnsdmainclasssingle{онтология}
\scnsdclass{объединенная онтология;структурная спецификация предметной области;теоретико-множественная онтология предметной области;логическая онтология предметной области;логическая иерархия понятий предметной области;логическая иерархия высказываний предметной области;терминологическая онтология предметной области}
\scnsdrelation{онтология*;используемые константы*;используемые утверждения*} 

\scnsegmentheader{Что такое онтология}
\scnstartsubstruct

\scnheader{онтология}
\scnidtf{sc-онтология}
\scnidtf{онтология, представленная в SC-коде}
\scnnote{Поскольку термин ``\textit{онтология}'' в SC-коде соответствует множеству всевозможных онтологий, представленных в SC-коде, то для формальных онтологий, представленных на других языках, необходимо использовать sc-идентификатор, содержащий явное указание этих языков, например, ``owl-онтология''.}
\scnidtf{sc-текст онтологии}
\scnidtf{sc-модель онтологии}
\scnidtf{семантическая спецификация \textit{знаний}}
\scnidtf{семантическая спецификация любого знания, имеющего достаточно сложную структуру, любого целостного фрагмента базы знаний -- предметной области, метода решения сложных задач некоторого класса, описания истории некоторого вида деятельности, описания области выполнения некоторого множества действий (области решения задач), языка  представления методов решения задач и т.д.}
\scnidtfexp{\uline{семантическая} \textit{спецификация} некоторого достаточно информативного ресурса (\textit{знания})}
\scnsubset{спецификация}
\scnnote{Если \textit{спецификация} может специфицировать (описывать) любую \textit{сущность}, то \textit{отнология} специфицирует только различные \textit{знания}. При этом наиболее важными объектами такой спецификации являются \textit{предметные области}}
\scnsubset{метазнание}
\scniselement{вид знаний}
\scnnote{\textit{онтологии} являются важнейшим \textit{видом знаний} (точнее, метазнаний), обеспечивающих семантическую систематизацию \textit{знаний}, хранимых в памяти \textit{интеллектуальных компьютерных систем} (в т.ч. \textit{ostis-систем}), и, соответственно, семантическую структуризацию \textit{баз знаний}}
\scnidtf{важнейший вид \textit{метазнаний}, входящих в состав базы знаний}
\scnidtf{спецификация (уточнение) системы \textit{понятий}, используемых в соответствующем (специфицируемом) \textit{знании}}
\scntext{эпиграф}{Определив точно значения слов, вы избавите человечество от половины заблуждений}
\scnaddlevel{1}
\scnrelfrom{автор}{Рене Декарт}
\scnaddlevel{-1}
\scnexplanation{\textit{онтология} включает в себя: 
\begin{scnitemize}
\item типологию специфицируемого \textit{знания};
\item связи специфицируемого \textit{знания} с другими \textit{знаниями};
\item спецификацию ключевых \textit{понятий}, используемых в специфицируемом \textit{знании}, а также ключевых экземпляров некоторых таких \textit{понятий}.
\end{scnitemize}}
\scnexplanation{Основная \textit{цель} построения \textit{онтологии} -- семантическое уточнение (пояснение, а в идеале -- определение) такого семейства \textit{знаков}, используемых в заданном \textit{знании}, которых достаточно для понимания смысла всего специфицируемого \textit{знания}. Как выясняется, количество \textit{знаков}, смысл которых определяет смысл всего специфицируемого \textit{знания}, \uline{не является большим}.}
\scnsubdividing{неформальная онтология
;формальная онтология
\scnaddlevel{1}
\scnidtf{онтология, представленная на формальном языке}
\scnrelto{ключевой знак}{\scncite{Loukashevich2011}}
\scnaddlevel{-1}
}
\scnheader{формальная онтология}
\scnidtf{формальное описание \uline{денотационной семантики} (семантической интерпретации) специфицируемого знания}
\scnnote{Очевидно, что при отсутствии достаточно полных формальных онтологий невозможно обеспечить семантическую совместимость (интегрируемость) различных знаний, хранимых в базе знаний, а также приобретаемых извне.}
\scnheader{онтология предметной области}
\scnnote{\textit{онтология} чаще всего трактуется как спецификация концептуализации (спецификация системы \textit{понятий}) заданной \textit{предметной области}. Здесь имеется в виду описание теоретико-множественных связей (прежде всего, классификации) используемых \textit{понятий}, а также описание различных закономерностей для сущностей, принадлежащих этим \textit{понятиям}. Тем не менее, важными видами спецификации \textit{предметной области} являются также: 
\begin{scnitemize}
\item описание связей специфицируемой \textit{предметной области} с другими \textit{предметными областями};
\item описание терминологии специфицируемой \textit{предметной области}.
\end{scnitemize}}
\scnnote{\textit{онтологию предметной области} можно трактовать, с одной стороны, как \textit{семантическую окрестность} соответствующей \textit{предметной области}, с другой стороны, как \textit{объединение} определённого вида \textit{семантических окрестностей} всех \textit{понятий}, используемых в рамках указанной \textit{предметной области}, а также, возможно, ключевых экземпляров указанных \textit{понятий}, если таковые экземпляры имеются}
\scnexplanation{Каждая конкретная онтология заданного вида представляет собой семантическую окрестность соответствующей (специфицируемой) предметной области.
Каждому \textit{виду онтологий} однозначно соответствует \textit{предметная область}, фрагментами которые являются конкретные \textit{онтологии} этого вида. Следовательно, каждому \textit{виду онтологий} соответствует свой специализированный sc-язык, обеспечивающий представление \textit{онтологий} этого вида.}
\scnidtf{описание \textit{денотационной семантики} языка, определяемого (задаваемого) соответствующей (специфицируемой) \textit{предметной области}} 
\scnidtf{информационная надстройка (метаинформация) над соответствующей (специфицируемой) \textit{предметной областью}, описывающая различные аспекты этой \textit{предметной области} как достаточно крупного, самодостаточного и семантически целостного фрагмента \textit{база знаний}} 
\scnidtf{метаинформация (метазнание) о некоторой \textit{предметной области}}
 \bigskip
\scnendstruct \scnendsegmentcomment{Что такое онтология}

\scnsegmentheader{Типология онтологий предметной области}
\scnstartsubstruct

\scnheader{онтология предметной области}
\scnsubdividing{частная онтология предметной области
\scnaddlevel{1}
\scnidtf{\textit{онтология}, представляющая спецификацию соответствующей \textit{предметной области} в том или ином аспекте}
\scnaddlevel{-1}
;объединённая онтология предметной области
\scnaddlevel{1}
\scnidtf{онтология \textit{предметной области}, являющаяся результатом объединения всех известных \textit{частных онтологий} этой предметной области}
\scnaddlevel{-1}}
\scnheader{частная онтология предметной области}
\scnnote{Каждая \textit{частная онтология} является фрагментом \textit{предметной области}, включающей в себя \uline{все}(!) частные онтологии, принадлежащие соответствующему \textit{виду онтологии}. При этом указанная \textit{предметная область}, в свою очередь, также имеет соответствующую ей \textit{онтологию}, которая уже является не метазнанием (как любая онтология), а метаметазнанием (спецификацией метазнания).}
\scnrelfrom{разбиение}{вид онтологий предметных областей}
\scnaddlevel{1}
\scneqtoset{структурная спецификация предметной области\\
\scnaddlevel{1}
\scnidtf{sc-окрестность (sc-спецификация) заданной предметной области в рамках \textit{Предметной области предметных областей}}
\scnidtf{схема предметной области}
\scnaddlevel{-1}
;теоретико-множественная онтология предметной области
\scnaddlevel{1}
\scnidtf{sc-спецификация заданной предметной области в рамках \textit{Предметной области множеств}}
\scnaddlevel{-1}
;логическая онтология предметной области 
\scnaddlevel{1}
\scnidtf{sc-текст формальной теории заданной предметной области}
\scnaddlevel{-1}
;терминологическая онтология предметной области}
\scnaddlevel{-1}

\scnheader{вид онтологий предметных областей}
\scnidtf{вид спецификаций \textit{предметных областей}}
\scnidtf{вид \textit{метазнаний}, описывающих соответствующие этому виду \textit{метазнаний} свойства \textit{предметных областей}}
\scnnote{Каждому виду \textit{онтологий предметных областей}, представленных в \textit{SC-коде} (здесь представленными в \textit{SC-коде} предполагаются не только сами \textit{онтологии}, но и специфицируемые ими \textit{предметные области}) ставится в соответствие \textit{sc-метаязык}, обеспечивающий представление \textit{метазнаний}, входящих в состав указанного \textit{вида онтологий}. Кроме того, обычное теоретико-множественное \textit{объединение} всех \textit{sc-текстов} указанного \textit{sc-метаязыка} означает построение \textit{предметной области} (предметной метаобласти), которая формально задаёт соответствующий \textit{вид онтологии} и которая будет иметь свою \textit{онтологию}. При этом каждую частную онтологию можно связать с той предметной областью, фрагментом которой эта онтология является.} 
\scnheader{логическая онтология предметной области}
\scnrelto{семейство подмножеств}{Предметная область логических формул, высказываний и формальных теорий}
\scnheader{теоретико-множественная онтология предметной области}
\scnrelto{семейство подмножеств}{Предметная область множеств}
\scnheader{структурная спецификация предметной области}
\scnrelto{семейство подмножеств}{Предметная область предметных областей}
\scnheader{структурная спецификация предметной области} 
\scnidtf{структурная онтология предметной области}
\scnidtf{ролевая структура ключевых элементов предметной области}
\scnidtf{схема ролей понятий предметной области и её связи со смежными предметными областями}
\scnidtf{схема предметной области}
\scnidtf{спецификация предметной области с точки зрения теории графов и теории \textit{алгебраических систем}}
\scnidtf{описание внутренней (ролевой) структуры \textit{предметной области}, а также её внешних связей с другими \textit{предметными областями}}
\scnidtf{описание ролей ключевых элементов предметной области (прежде всего, понятий -- концептов), а также \scnqq{место} специфицируемой предметной области в множестве себе подобных}
\scnidtf{\textit{семантическая окрестность} знака \textit{предметной области} в рамках самой этой \textit{предметной области}, включающая в себя все \textit{ключевые знаки}, входящие в состав \textit{предметной области} (ключевые понятия и ключевые объекты исследования предметной области) с указанием их ролей (свойств) в рамках этой \textit{предметной области} и \textit{семантическая окрестность} знака специфицируемой \textit{предметной области} в рамках \textit{Предметной области предметных областей}, включающая в себя связи специфицируемой \textit{предметной области} с другими семантически близкими ей \textit{предметными областями} (дочерними и родительскими, аналогичными в том или ином смысле (например, изоморфными), имеющими одинаковые \textit{классы объектов исследования} или одинаковые наборы \textit{исследуемых отношений})}
\scnheader{теоретико-множественная онтология предметной области}
\scnidtf{\textit{семантическая окрестность} специфицируемой \textit{предметной области} в рамках \textit{Предметной области множеств}, описывающая теоретико-множественные связи между \textit{понятиями} специфицируемой \textit{предметной области}, включая связи \textit{отношений} с их \textit{областями определения} и \textit{доменами}, связи используемых \textit{параметров} и классов структур их \textit{областями определения}}
\scnidtf{онтология описывающая:
\begin{scnitemize}
\item классификацию объектов исследования специфицируемой предметной области;
\item соотношение областей определения и доменов используемых отношений с выделенным классами объектов исследования, а также с выделенными классами вспомогательных (смежных) объектов, не являющихся объектами исследования в специфицируемой предметной области;
\item спецификацию используемых отношений и, в том числе, указание того, все ли связки этих отношений входят в состав специфицируемой предметной области
\end{scnitemize}}
\scnexplanation{теоретико-множественная онтология предметной области включает в себя:
\begin{scnitemize}
\item теоретико-множественные связи (в т.ч. таксономию) между всеми используемыми понятиями, входящими в состав специфицируемой предметной области;
\item теоретико-множественную спецификацию всех \textit{отношений}, входящих в состав специфицируемой предметной области (ориентированность, арность, область определения, домены и т.д.);
\item теоретико-множественную спецификацию всех параметров, используемых в предметной области (области определения параметров, шкалы, единицы измерения, точки отсчета);
\item теоретико-множественную спецификацию всех используемых классов структур
\end{scnitemize}}

\scnheader{логическая онтология предметной области}
\scnidtf{формальная теория заданной (специфицируемой) предметной области, описывающая с помощью переменных, кванторов, логических связок, формул различные свойства экземпляров понятий, используемых в специфицируемой предметной области}
\scnidtfexp{онтология предметной области, которая включает в себя:
\begin{scnitemize}
\item формальные определения всех понятий, которые в рамках специфицируемой предметной области являются определяемыми;
\item неформальные пояснения и некоторые формальные спецификации (как минимум, примеры) для всех понятий, которые в рамках специфицируемой предметной области являются неопределяемыми;
\item иерархическую систему понятий, в которой для каждого понятия, исследуемого в специфицируемой предметной области либо указывается факт неопределяемости этого понятия, либо указываются все понятия, на основе которых даётся определение данному понятию.
 В результате этого множество исследуемых понятий разбивается на ряд уровней: 
\scnaddlevel{1}
\begin{scnitemizeii}
\scnaddlevel{1}
\item неопределяемые понятия;
\item понятия 1-го уровня, определяемые только на основе неопределяемых понятий;
\item понятия 2-го уровня, определяемые на основе понятий, изменяющих 1-й уровень и ниже;
\item и т.д.
\scnaddlevel{-1}
\end{scnitemizeii}
\item формальную запись всех аксиом, т.е. высказываний, которые не требуют доказательств;
\item формальную запись высказываний, истинность которых требует обоснования (доказательства);
\item формальные тексты доказательства истинности высказываний, представляющие собой спецификацию последовательности шагов соответствующих рассуждений (шагов логического вывода, применения различных правил логического вывода);
\item иерархическую систему высказываний, в которой для каждого высказывания, истинного по отношению к специфицируемой предметной области, либо указывается аксиоматичность этого высказывания, либо перечисляются \uline{все} высказывания, на основе которых доказывается данное высказывание. В результате этого множество высказываний, истинных по отношению к специфицируемой предметной области, разбивается на ряд уровней:
\scnaddlevel{1}
\begin{scnitemizeii}
\item аксиомы;
\item высказывания 1-го уровня, доказываемые только на основе аксиом;
\item высказывания 2-го уровня, доказываемые на основе высказываний, находящихся на 1-м уровне и ниже.
\end{scnitemizeii}
\scnaddlevel{-1}
\item формальная запись гипотетических высказываний;
\item формальное описание логико-семантической типологии высказываний -- высказываний о существовании, о несуществовании, об однозначности, высказывания определяющего  типа (которые можно использовать в качестве определений соответствующих понятий);
\item формальное описание различного вида логико-семантических связей между высказываниями (например, между высказыванием и его обобщением);
\item формальное описание аналогии
\scnaddlevel{1}
\begin{scnitemizeii}
\item между определениями;
\item между высказываниями любого вида;
\item между доказательствами различных высказываний.
\end{scnitemizeii}
\scnaddlevel{-1}
\end{scnitemize}
}

\scnheader{терминологическая онтология предметной области}
\scnidtf{онтология, описывающая \uline{правила построения} терминов (sc-идентификаторов), соответствующих \mbox{sc-элементам}, принадлежащим специфицируемой предметной области, а также описывающая различного рода терминологические связи между используемыми терминами, характеризующие происхождение этих терминов}
\scnidtf{система терминов заданной предметной области} 
\scnidtf{тезаурус соответствующей предметной области}
\scnidtf{словарь соответствующей (специфицируемой) предметной области} 
\scnidtf{фрагмент глобальной \textit{Предметной области sc-идентификаторов} (внешних идентификаторов sc-элементов), обеспечивающий терминологическую спецификацию некоторой предметной области}

\bigskip
\scnendstruct \scnendsegmentcomment{Типология онтологии предметной области}

\newpage
\scnsegmentheader{Понятие объединённой онтологии предметной области, а также понятие предметной области и онтологии}
\scnstartsubstruct
\scnheader{объединённая онтология предметной области}
\scnidtf{объединение всех частных онтологий, соответствующих одной предметной области}  
\scnreltoset{обобщённое объединение}{структурная спецификация предметной области;теоретико-множественная онтология предметной области;логическая онтология предметной области;терминологическая онтология предметной области}
\scnheader{предметная область и онтология}
\scnidtf{интеграция некоторой \textit{предметной области}  c соответствующей ей \textit{\uline{объединённой} онтологией}}
\scnidtf{предметная область \& онтология}
\scnreltoset{обобщённое объединение}{предметная область;объединённая онтология предметной области}
\scnidtf{sc-текст, являющийся объединением некоторой предметной области, представленной в SC-коде, и объединённой онтологии этой предметной области, также представленной в SC-коде}
\scnidtf{интеграция предметной области и всех онтологий, специфицирующих эту предметную область}
\scnidtf{совокупность различных \textit{фактов} о структуре некоторой области деятельности некоторых \textit{субъектов}, а также различного вида \textit{знаний}, специфицирующих эту область деятельности}
\scnidtf{факты и знания о некоторой области деятельность}
\scnidtf{sc-модель предметной области и всевозможных онтологий, специфицирующих эту предметную область (и, в первую очередь, её ключевых понятий) в разных ракурсах}
\scnidtf{целостный с логико-семантической точки зрения фрагмент базы знаний ostis-системы, акцентирующий внимание на конкретном классе объектов исследования и на конкретном аспекте их рассмотрения}
\scnnote{Декомпозиция \textit{базы знаний} на предметные области вместе с соответствующими им обязательными онтологиями носит в известной мере условный характер. При этом надо помнить, что для исследования и формализации межпредметных (междисциплинарных) свойств и закономерностей необходимо строить иерархию \textit{предметных областей} т.е. переходить от \textit{предметных областей} к их объединениям.} 
\scnnote{\textit{предметные области и онтологии} являются 
основным видом \textit{разделов баз знаний}, обладающих высокой степенью их независимости друг от друга и четкими правилами их согласования, что обеспечивает их семантическую (понятную) совместимость в рамках всей \textit{базы знаний}.}
\scnidtf{основной вид семантических кластеров \textit{базы знаний ostis-системы}}
\scnnote{При формализации \textit{научных знаний} в самых различных областях в рамках \textit{базы знаний ostis-системы} (в том числе в рамках \textit{объединённой виртуальной базы знаний Экосистемы OSTIS}) \uline{каждой}(!) научной дисциплине будет соответствовать своя \textit{предметная область} и соответствующая ей \textit{объединённая онтология}. Следует при этом подчеркнуть, что наличие в Технологии OSTIS развитых метаязыковых средств создает хорошие условия для активизации междисциплинарных исследований, для конвергенции различных научных дисциплин.}
\scnnote{Все ролевые отношения, предметные области с ключевыми объектами исследования, а также с используемыми, вводимыми и исследуемыми понятиями, логичным образом применяются и в онтологиях, и в объединениях предметных областей с соответствующими им онтологиями, поскольку указанные ключевые объекты исследования и ключевые понятия входят и являются ключевыми не только для предметных областей, но и для соответствующих им онтологий}

\bigskip
\scnendstruct \scnendsegmentcomment{Понятие объединённой онтологии предметной области, а также понятие предметной области и онтологии}

\scnsegmentheader{Отношения, заданные на множестве онтологий}
\scnstartsubstruct

\scnheader{отношение, заданное на множестве онтологий}
\scnhaselement{онтология*}
\scnaddlevel{1}
\scnidtf{быть онтологией заданного информационного ресурса*}
\scnsuperset{онтология предметной области*}
\scnaddlevel{-1}
\scnhaselement{часть знания*}
\scnhaselement{декомпозиция знания*}
\scnaddlevel{1}
\scnsuperset{декомпозиция предметной области и онтологии*}
\scnaddlevel{1}
\scnidtf{декомпозиция предметной области и онтологии на предметную область и её частные онтология*}
\scnaddlevel{-2}
\scnhaselement{область выполнения действий*}

\scnheader{онтология*}
\scnsubset{спецификация*}
\scnaddlevel{1}
\scnidtf{быть спецификацией заданной сущности*}
\scnaddlevel{-1}
\scnidtf{быть спецификацией знания (информационного ресурса)*} 
\scnidtf{метазнание*} 

\scnheader{онтология предметной области*}
\scnsubset{онтология*}
\scnidtf{быть онтологией заданной предметной области*}
\scnidtf{Бинарное ориентированное отношение, здорово связывает некоторую предметную область с онтологией, специфицирующей эту предметную область*}
\scnsuperset{объединённая онтология предметной области*}
\scnsuperset{структурная спецификация предметной области*}
\scnsuperset{теоретико множественная онтология предметной области*}
\scnsuperset{логическая онтология предметной области*}
\scnsuperset{терминологическая онтология предметной области*}
\scnnote{Отношение ``\textit{онтология предметной области*}'' и его подмножества не являются обязательными, поскольку конкретный \textit{вид онтологии} и вид сущности, специфицируемой данной онтологией, однозначно задаются классами, которым принадлежат эта онтология и эта сущность. Таким образом, указанные отношения вводятся исключительно для дидактических целей и не рекомендуются к использованию, поскольку увеличивают число случаев логической эквивалентности sc-текстов.}

\scnheader{часть знания*}
\scnsubset{часть*}
\scnidtf{\textit{\uline{знание}}, являющееся фрагментом (\textit{частью}) другого заданного \textit{знания}*} 
\scnnote{С помощью данного \textit{отношения}, также различных подмножеств отношения ``быть \textit{онтологией}*'' можно переходить от \textit{первичных знаний} к \textit{метазнаниям}, от \textit{метазнаний} к \textit{метаметазнаниям} и так далее.}

\scnheader{область выполнения действий*}
\scnidtf{быть формальной моделью области выполнения заданной системы действий (заданного сложного действия или заданной деятельности)*}
\scnnote{\textit{предметные области} и онтологии (предметные области, интегрированные с их \textit{объединёнными онтологиями}) являются часто используемыми и весьма удобными формальными моделями, обеспечивающими качественную семантическую систематизацию областей выполнения различного вида действий, осуществляемых кибернетическими системами как в собственной памяти (путём непосредственной обработки соответствующих предметных областей и их онтологий), так и в своей внешней среде. \\
Многим \textit{предметным областям и онтологиям} будут ставиться в соответствие следующие \textit{локальные предметные области действий и задач}, выполняемых на основе этих \textit{предметных областей и онтологий}: 
\begin{scnitemize}
\item история эксплуатации соответствующей предметной области и онтологии при выполнении действий и задач в рамках собственной памяти;
\item история эволюции соответствующей предметной области и онтологии;
\item история выполнения действий и задач во внешней среде на основе соответствующей предметной области и онтологии.
\end{scnitemize}}}

\bigskip
\scnendstruct \scnendsegmentcomment{Отношения, заданные на множестве онтологий}

\bigskip
\scnendstruct \scnendcurrentsectioncomment

\end{SCn}

\scsubsection[\scneditors{Василевская А.П.;Зотов Н.В.;Орлов М.К.}\protect\scnmonographychapter{Глава 2.5. Смысловое представление логических формул и высказываний в различного вида логиках}]{Предметная область и онтология логических формул, высказываний и логических sc-языков}
\label{sd_logics}
\begin{SCn}

\scnsectionheader{\currentname}

\scnstartsubstruct

\scnheader{Предметная область логических формул, высказываний и формальных теорий}
\scniselement{предметная область}
\scnsdmainclasssingle{формальная теория}
\scnsdclass{высказывание;атомарное высказывание;неатомарное высказывание;фактографическое высказывание;логическая формула;атомарная логическая формула;неатомарная логическая формула;утверждение;определение;общезначимая логическая формула;противоречивая логическая формула;нейтральная логическая формула;выполнимая логическая формула;невыполнимая логическая формула;тавтология;квантор;формула существования;число значений переменной;кратность существования;единственное существование;логическая формула и единственность;открытая логическая формула;замкнутая логическая формула}
\scnsdrelation{предметная область\scnrolesign;аксиома\scnrolesign;теорема\scnrolesign;подформула*;логическая связка*;импликация*;если\scnrolesign;то\scnrolesign;эквиваленция*;конъюнкция*;дизъюнкция*;строгая дизъюнкция*;отрицание*;всеобщность*;неатомарное существование*;связываемые переменные\scnrolesign}

\scnheader{формальная теория}
\scnexplanation{\textbf{\textit{формальная теория}} — это множество высказываний, которые считаются истинными в рамках данной \textbf{\textit{формальной теории}}.}
\scnaddlevel{1}
\scnexplanation{Высказывания могут быть как фактографическими, так и логическими формулами. Некоторые высказывания считаются аксиомами, а другие доказываются на основе других высказываний в рамках этой же \textbf{\textit{формальной теории}}.}
\scnaddlevel{-1}
\scnexplanation{Каждая формальная теория интерпретируется (т.е. ее высказывания являются истинными) на какой-либо \textit{предметной области}, которая является максимальным из \textit{фактографических высказываний} (их \textit{объединением*}),  входящих в состав этой \textbf{\textit{формальной теории}}.}
\scnexplanation{Каждой \textbf{\textit{формальной теории}} соответствует одна \textit{предметная область}, которая входит в нее под атрибутом \textit{предметная область\scnrolesign}.}
\scnexplanation{Каждая \textbf{\textit{формальная теория}} может рассматриваться как конъюнктивное высказывание, априори истинное (с чьей-то точки зрения) при интерпретации на соответствующей \textit{предметной области}.}
\scnexplanation{Каждая \textbf{\textit{формальная теория}} задаётся алфавитом, формулами, аксиомами, правилами вывода.}
\scnaddlevel{1}
\scnrelfrom{источник}{\scncite{Serhievskaya2004}}
\scnaddlevel{-1}

\scnheader{предметная область\scnrolesign}
\scniselement{ролевое отношение}
\scnexplanation{\textbf{\textit{предметная область\scnrolesign}} -- это \textit{ролевое отношение}, связывающее \textit{формальную теорию} с \textit{предметной областью}, на которой данная \textit{формальная теория} интерпретируется (в рамках которой истинны \textit{высказывания}, входящие в состав этой \textit{формальной теории}).}
\scnexplanation{\textit{Предметная область} является максимальным фактографическим высказыванием \textit{формальной теории}, которая интерпретируется на данной \textit{предметной области}.}
\scnrelfrom{смотрите}{\nameref{sd_sd}}

\scnheader{аксиома\scnrolesign}
\scniselement{ролевое отношение}
\scnexplanation{\textbf{\textit{аксиома\scnrolesign}} -- это \textit{ролевое отношение}, связывающее \textit{формальную теорию} с \textit{высказыванием}, истинность которого не  требует доказательства в рамках этой \textit{формальной теории}.}

\scnheader{теорема\scnrolesign}
\scniselement{ролевое отношение}
\scnexplanation{\textbf{\textit{теорема\scnrolesign}} -- это \textit{ролевое отношение}, связывающее \textit{формальную теорию} с \textit{высказыванием}, истинность которого доказывается в рамках этой \textit{формальной теории}.}

\scnheader{высказывание}
\scnsubdividing{атомарное высказывание;неатомарное высказывание}
\scnsubdividing{фактографическое высказывание;логическая формула}
\scnexplanation{Под \textbf{\textit{высказыванием}} понимается некоторая \textit{структура} (в которую входят \textit{sc-константы} из некоторой предметной области и/или \textit{sc-переменные}) или \textit{логическая связка}, которая может трактоваться как истинная или ложная в рамках какой-либо \textit{предметной области}.}
\scnnote{Истинность \textbf{\textit{высказывания}} задается путем указания принадлежности знака этого высказывания \textit{формальной теории}, соответствующей данной \textit{предметной области}. Ложность высказывания задается путем указания принадлежности знака \textit{отрицания*} этого высказывания данной \textit{формальной теории}.}
\scnaddlevel{1}
\scnnote{Явно указанная непринадлежность \textbf{\textit{высказывания}} \textit{формальной теории} может говорить как о его ложности в рамках данной теории (если это указано рассмотренным выше образом), так и о том, что данное  \textbf{\textit{высказывание}} вообще не рассматривается в данной \textit{формальной теории} (например, использует понятия, не принадлежащие данной \textit{предметной области}).}
\scnnote{Одно и то же \textbf{\textit{высказывание}} может быть истинно в рамках одной \textit{формальной теории} и ложно в рамках другой.}
\scnaddlevel{-1}

\scnheader{высказывание формальной теории\scnrolesign}
\scniselement{неосновное понятие}
\scnsubdividing{истинное высказывание\scnrolesign\\
	\scnaddlevel{1}
		\scnidtf{высказывание, истинное в рамках данной формальной теории\scnrolesign}
		\scnidtf{высказывание, знак которого принадлежит данной формальной теории\scnrolesign}
	\scnaddlevel{-1}
	;ложное высказывание\scnrolesign\\
	\scnaddlevel{1}
		\scnidtf{высказывание, ложное в рамках данной формальной теории\scnrolesign}
		\scnidtf{высказывание, знак отрицания которого принадлежит данной формальной теории\scnrolesign}
	\scnaddlevel{-1}
	;нечеткое высказывание\scnrolesign\\
	\scnaddlevel{1}
		\scnidtf{гипотетическое высказывание\scnrolesign}
		\scnidtf{высказывание, возможно истинное или ложное в рамках данной формальной теории\scnrolesign}
		\scnidtf{высказывание, истинное или ложное в рамках данной формальной теории с некоторой вероятностью\scnrolesign}
	\scnaddlevel{-1}
	;бессмысленное высказывание\scnrolesign\\
	\scnaddlevel{1}
		\scnidtf{высказывание, бессмысленное в рамках данной формальной теории\scnrolesign}
		\scnidtf{высказывание, не рассматриваемое в рамках данной формальной теории\scnrolesign}
		\scnexplanation{Высказывание является бессмысленным в рамках заданной формальной теории, если в какое-либо \textit{атомарное высказывание} в его составе (или в само это высказывание, если оно является атомарным) входит какая-либо \textit{sc-константа}, не являющаяся элементом предметной области, описываемой указанной \textit{формальной теорией}.}
	\scnaddlevel{-1}}

\scnheader{атомарное высказывание}
\scnsubset{структура}
\scnsubdividing{атомарное фактографическое высказывание;атомарная логическая формула}
\scndefinition{\textbf{\textit{атомарное высказывание}} -- это \textit{высказывание}, которое содержит хотя бы один \textit{sc-элемент}, не являющийся знаком другого \textit{высказывания}.}
\scnheader{неатомарное высказывание}
\scndefinition{\textbf{\textit{неатомарное высказывание}} -- это \textit{высказывание}, в состав которого входят только знаки других \textit{высказываний}.}
\scnnote{Следует отметить, что мы не можем говорить об истинности либо ложности \textbf{\textit{неатомарного высказывания}} в рамках какой-либо \textit{формальной теории}, в случае, когда невозможно установить истинность либо ложность любого из его элементов в рамках этой же \textit{формальной теории}.}

\scnheader{фактографическое высказывание}
\scnsuperset{атомарное фактографическое высказывание}
\scnexplanation{Под \textit{фактографическим высказыванием} понимается:
\begin{scnitemize}
    \item \textit{атомарное высказывание}, в состав которого не входит ни одна \textit{sc-переменная};
    \item \textit{неатомарное высказывание}, все элементы которого также являются \textbf{\textit{фактографическими высказываниями}}.
\end{scnitemize}
}

\scnheader{логическая формула}
\scnexplanation{Под \textit{логической формулой} понимается:
\begin{scnitemize}
    \item \textit{атомарное высказывание}, в состав которого входит хотя бы одна \textit{sc-переменная};
    \item \textit{неатомарное высказывание}, хотя бы один элемент которого является \textbf{\textit{логической формулой}}.
\end{scnitemize}}
\scnsubdividing{атомарная логическая формула;неатомарная логическая формула}
\scnsubdividing{открытая логическая формула;замкнутая логическая формула}

\scnheader{атомарная логическая формула}
\scnidtf{обобщенная структура}
\scnidtf{атомарная формула существования}
\scnexplanation{Под \textbf{\textit{атомарной логической формулой}} понимается \textit{атомарное высказывание}, которое является \textit{логической формулой}.}
\scnexplanation{\textbf{\textit{Атомарная логическая формула}} -- это  логическая формула, которая не содержит логических связок.}
\scnnote{По умолчанию \textbf{\textit{атомарная логическая формула}} трактуется как \textit{высказывание} о существовании, то есть наличия в памяти значений, соответствующих всем \textit{sc-переменным}, входящим в состав данной формулы и не попадающих под действие какого-либо другого \textit{квантора} (указанного явно или по умолчанию). Таким образом, на все \textit{sc-переменные}, входящие в состав \textbf{\textit{атомарной логической формулы}} и не попадающие под действие какого-либо другого \textit{квантора}, неявно накладывается квантор \textit{существования*}.}
\scnaddlevel{1}
\scnrelfrom{основной sc-идентификатор}{\scnfilelong{Примечание про высказывание о существовании}}
\scnaddlevel{-1}

\scnheader{неатомарная логическая формула}
\scnsubdividing{общезначимая логическая формула;противоречивая логическая формула;нейтральная логическая формула}
\scnsubdividing{выполнимая логическая формула;невыполнимая логическая формула}
\scnsuperset{тавтология}
\scnexplanation{Под \textbf{\textit{неатомарной логической формулой}} понимается \textit{неатомарное высказывание}, которое является \textit{логической формулой}.}
\scnnote{Для того, чтобы рассмотреть типологию \textbf{\textit{неатомарных логических формул}}, будем говорить, что исследуется истинность самой \textbf{\textit{неатомарной логической формулы}} и всех ее \textit{подформул*} в рамках одной и той же \textit{формальной теории}, при этом не важно, какой именно. Также считается, что в рассматриваемой \textit{формальной теории} каждая \textit{подформула*} рассматриваемой \textbf{\textit{неатомарной логической формулы}} в рамках этой \textit{формальной теории} может однозначно трактоваться как либо истинная, либо ложная. В противном случае мы не можем говорить об истинности либо ложности исходной \textbf{\textit{неатомарной логической формулы}} в рамках этой \textit{формальной теории}.}
\scnrelfrom{описание примера}{Примеры неатомарных логических формул}

\scnheader{подформула*}
\scnidtf{частная формула*}
\scniselement{бинарное отношение}
\scniselement{ориентированное отношение}
\scniselement{транзитивное отношение}
\scndefinition{Будем называть \textbf{\textit{подформулой*}} \textit{неатомарной логической формулы} \textbf{\textit{fi}} любую \textit{логическую формулу} \textbf{\textit{fj}}, являющуюся элементом исходной формулы \textbf{\textit{fi}}, а также любую \textbf{\textit{подформулу*}} формулы \textbf{\textit{fj}}.}
\scnrelfrom{описание примера}{
\scnfilescg{figures/sd_logical_formulas/subformula.png}}
\scnaddlevel{1}
\scniselement{sc.g-текст}
\scnaddlevel{-1}

\scnheader{утверждение}
\scnidtf{текст логической формулы}
\scndefinition{\textbf{\textit{утверждение}} -- это \textit{семантическая окрестность} некоторой \textit{логической формулы}, в которую входит полный текст этой \textit{логической формулы}, а также факт принадлежности этой \textit{логической формулы} некоторой \textit{формальной теории}.}
\scnexplanation{Знак \textit{логической формулы}, семантическая окрестность которой представляет собой утверждение, является \textit{главным ключевым sc-элементом\scnrolesign} в рамках этого \textbf{\textit{утверждения}}. Знаки понятий соответствующей \textit{предметной области}, которые входят в состав какой-либо \textit{подформулы*} указанной \textit{логической формулы}, будут \textit{ключевыми sc-элементами\scnrolesign} в рамках этого \textbf{\textit{утверждения}}.

Полный текст некоторой \textit{логической формулы} включает в себя:
\begin{scnitemize}
    \item знак самой этой \textit{логической формулы};
    \item знаки всех ее \textit{подформул*};
    \item элементы всех \textit{логических формул}, знаки которых попали в данную структуру;
    \item все пары принадлежности, связывающие \textit{логические формулы}, знаки которых попали в данную структуру, с их компонентами.
\end{scnitemize}
Таким образом, факт принадлежности (истинности) логической формулы нескольким \textit{формальным теориям} будет порождать новое утверждение для каждой такой \textit{формальной теории}. Текст \textbf{\textit{утверждения}} входит в состав \textit{логической онтологии}, соответствующей \textit{предметной области}, на которой интерпретируется \textit{главный ключевой sc-элемент\scnrolesign} данного утверждения.}
\scntext{правило идентификации экземпляров}{\textbf{\textit{утверждения}} в рамках \textit{Русского языка} именуются по следующим правилам:
\begin{scnitemize}
    \item в начале идентификатора пишется сокращение \textbf{Утв.};
    \item далее в круглых скобках через точку с запятой перечисляются основные идентификаторы \textit{ключевых \mbox{sc-элементов}\scnrolesign} данного \textbf{\textit{утверждения}}. Порядок определяется в каждом конкретном случае в зависимости от того, свойства каких из этих \textit{понятий} описывает данное \textbf{\textit{утверждение}} в большей или меньшей степени.
\end{scnitemize}
}
\scnaddlevel{1}
\scntext{описание примера}{\textit{Утв. (параллельность*; секущая*)}}
\scnnote{Могут быть исключения для \textbf{\textit{утверждений}}, названия которых закрепились исторически, например, \textit{Теорема Пифагора}, \textit{Аксиома о прямой и точке}.}
\scnaddlevel{-1}
\scnrelfrom{описание примера}{\scnfilescg{figures/sd_logical_formulas/statement.png}}
\scnaddlevel{1}
\scnnote{Утверждение показывает, что соответствующие углы при пересечении параллельных прямых секущей равны.}
\scniselement{sc.g-текст}
\scnaddlevel{-1}

\scnheader{определение}
\scnidtf{текст определения}
\scnsubset{утверждение}
\scndefinition{\textbf{\textit{определение}} -- это \textit{утверждение}, \textit{главным ключевым sc-элементом\scnrolesign} которого является связка \textit{эквиваленции*}, однозначно определяющая некоторое понятие на основе других понятий.}
\scnnote{Каждое определение имеет ровно один \textit{ключевой sc-элемент\scnrolesign} (не считая \textit{главного ключевого sc-элемента\scnrolesign}).}
\scnnote{Для одного и того же понятия в рамках одной \textit{формальной теории} может существовать несколько \textit{утверждений об эквиваленции*}, однозначно задающих некоторое понятие на основе других, однако только одно такое \textit{утверждение} в рамках этой \textit{формальной теории} может быть отмечено как \textbf{\textit{определение}}. Остальные \textit{утверждения об эквиваленции*} могут трактоваться как \textit{пояснения} данного понятия.}
\scntext{правило идентификации экземпляров}{\textbf{\textit{определения}} в рамках \textit{Русского языка} именуются по следующим правилам:
\begin{scnitemize}
    \item в начале идентификатора пишется сокращение \textbf{Опр.};
    \item далее в круглых скобках через точку с запятой записывается основной идентификатор  \textit{ключевого sc-элемента\scnrolesign} данного \textbf{\textit{определения}}.
\end{scnitemize}
}
\scnaddlevel{1}
\scntext{описание примера}{\textit{Опр. (ромб)}}
\scnaddlevel{-1}
\scnrelfrom{описание примера}{\scnfilescg{figures/sd_logical_formulas/definition.png}}
\scnaddlevel{1}
\scnnote{Определение показывает, что ромб — это четырёхугольник, у которого все стороны равны.}
\scniselement{sc.g-текст}
\scnaddlevel{-1}

\scnheader{общезначимая логическая формула}
\scnidtf{тождественно истинная логическая формула}
\scnsubset{выполнимая логическая формула}
\scnsubset{тавтология}
\scndefinition{\textbf{\textit{общезначимая логическая формула}} -- это \textit{логическая формула}, для которой не существует \textit{формальной теории}, в рамках которой она была бы ложной с учетом истинности и ложности всех ее \textit{подформул*} в рамках этой же \textit{формальной теории}.}
\scnrelfrom{описание примера}{
\scnfilescg{figures/sd_logical_formulas/valid_formula.png}}
\scnaddlevel{1}
\scnrelfrom{основной sc-идентификатор}{\scnfilelong{закон тождества}}
\scniselement{sc.g-текст}
\scnaddlevel{-1}

\scnheader{противоречивая логическая формула}
\scnidtf{тождественно ложная логическая формула}
\scnsubset{невыполнимая логическая формула}
\scnsubset{тавтология}
\scndefinition{\textbf{\textit{противоречивая логическая формула}} -- это \textit{логическая формула}, для которой не существует \textit{формальной теории}, в рамках которой она была бы истинной с учетом истинности и ложности всех ее \textit{подформул*} в рамках этой же \textit{формальной теории}.}
\scnrelfrom{описание примера}{
\scnfilescg{figures/sd_logical_formulas/contradiction_formula.png}}
\scnaddlevel{1}
\scnrelfrom{основной sc-идентификатор}{\scnfilelong{закон противоречия}}
\scniselement{sc.g-текст}
\scnaddlevel{-1}

\scnheader{нейтральная логическая формула}
\scnsubset{выполнимая логическая формула}
\scndefinition{\textbf{\textit{нейтральная логическая формула}} -- это \textit{логическая формула}, для которой существует хотя бы одна \textit{формальная теория}, в рамках которой эта формула ложна, и хотя бы одна \textit{формальная теория}, в рамках которой эта формула истинна.}
\scnrelfrom{описание примера}{
\scnfilescg{figures/sd_logical_formulas/neutral_formula.png}}
\scnaddlevel{1}
\scnnote{В евклидовой геометрии в плоскости через точку, не лежащую на данной прямой, можно провести одну и только одну прямую, параллельную данной. В геометрии Лобачевского данный постулат является ложным.}
\scnnote{В сферической геометрии все прямые пересекаются.}
\scniselement{sc.g-текст}
\scnaddlevel{-1}

\scnheader{непротиворечивая логическая формула}
\scnidtf{выполнимая логическая формула}
\scndefinition{\textbf{\textit{непротиворечивая логическая формула}} -- это \textit{логическая формула}, для которой существует хотя бы одна \textit{формальная теория}, в рамках которой эта формула истинна.}
\scnreltoset{объединение}{нейтральная логическая формула;общезначимая логическая формула}

\scnheader{необщезначимая логическая формула}
\scnidtf{невыполнимая логическая формула}
\scndefinition{\textbf{\textit{необщезначимая логическая формула}} -- это \textit{логическая формула}, для которой существует хотя бы одна \textit{формальная теория}, в рамках которой эта формула ложна.}
\scnreltoset{объединение}{нейтральная логическая формула;противоречивая логическая формула}
 
\scnheader{тавтология}
\scndefinition{\textbf{\textit{тавтология}} -- это \textit{логическая формула}, которая является либо только истинной, либо только ложной в рамках всех \textit{формальных теорий}, в которых можно установить ее истинность или ложность.}
\scnexplanation{\textbf{\textit{тавтология}} -- это такая \textit{логическая формула}, которая является либо \textit{общезначимой логической формулой}, либо \textit{противоречивой логической формулой}.}

\scnheader{логическая связка*}
\scnidtf{неатомарная логическая формула}
\scnidtf{логический оператор*}
\scnidtf{пропозициональная связка*}
\scniselement{класс связок разной мощности}
\scnrelto{семейство подмножеств}{неатомарное высказывание}
\scndefinition{\textbf{\textit{логическая связка*}} -- это отношение (класс связок), связками которого являются \textit{высказывания}.}
\scnexplanation{\textbf{\textit{логическая связка*}} -- это \textit{отношение}, областью определения которого является множество \textit{высказываний}, при этом само это отношение и некоторые его подмножества могут быть \textit{классами связок разной мощности}.}

\scnheader{конъюнкция*}
\scnidtf{логическое и*}
\scnidtf{логическое умножение*}
\scnsubset{логическая связка*}
\scniselement{неориентированное отношение}
\scniselement{класс связок разной мощности}
\scndefinition{\textbf{\textit{конъюнкция*}} -- это множество конъюнктивных \textit{высказываний}, каждое из которых истинно в рамках некоторой \textit{формальной теории} только в том случае, когда все его компоненты истинны в рамках этой же \textit{формальной теории}.}
\scnnote{\textbf{\textit{конъюнкция*}} атомарных формул может быть заменена на атомарную формулу, полученную путём объединения исходных атомарных формул.}
\scnaddlevel{1}
\scnrelfrom{описание примера}{
\scnfilescg{figures/sd_logical_formulas/conjunction_triangles.png}}
\scnaddlevel{1}
\scnexplanation{Данные конструкции эквивалентны по принципу $\exists x T(x) \land \exists x PT(x) \ \Longrightarrow \ \exists x (T(x) \land PT(x))$}
\scnaddlevel{1}
\scnexplanation{Следует помнить про \textbf{\textit{Примечание про высказывание о существовании}}.}
\scnaddlevel{-1}
\scniselement{sc.g-текст}
\scnaddlevel{-1}
\scnaddlevel{-1}
\scnrelfrom{описание примера}{
\scnfilescg{figures/sd_logical_formulas/conjunction.png}}
\scnaddlevel{1}
\scniselement{sc.g-текст}
\scnaddlevel{-1}

\scnheader{дизъюнкция*}
\scnidtf{логическое или*}
\scnidtf{логическое сложение*}
\scnidtf{включающее или*}
\scnsubset{логическая связка*}
\scniselement{неориентированное отношение}
\scniselement{класс связок разной мощности}
\scndefinition{\textbf{\textit{дизъюнкция*}} -- это множество дизъюнктивных \textit{высказываний}, каждое из которых истинно в рамках некоторой \textit{формальной теории} только в том случае, когда хотя бы один его компонент является истинным в рамках этой же \textit{формальной теории}.}
\scnrelfrom{описание примера}{
\scnfilescg{figures/sd_logical_formulas/disjunction.png}}
\scnaddlevel{1}
\scniselement{sc.g-текст}
\scnaddlevel{-1}

\scnheader{отрицание*}
\scnsubset{логическая связка*}
\scnsubset{синглетон}
\scndefinition{\textbf{\textit{отрицание*}} -- это множество \textit{высказываний} об отрицании, каждое из которых истинно в рамках некоторой \textit{формальной теории} только в том случае, когда его единственный элемент является ложным в рамках этой же \textit{формальной теории}.}
\scnrelfrom{описание примера}{
\scnfilescg{figures/sd_logical_formulas/negation.png}}
\scnaddlevel{1}
\scniselement{sc.g-текст}
\scnaddlevel{-1}

\scnheader{строгая дизъюнкция*}
\scnidtf{сложение по модулю 2*}
\scnidtf{исключающее или*}
\scnidtf{альтернатива*}
\scnsubset{логическая связка*}
\scniselement{неориентированное отношение}
\scniselement{класс связок разной мощности}
\scndefinition{\textbf{\textit{строгая дизъюнкция*}} -- это множество строго дизъюнктивных \textit{высказываний}, каждое из которых истинно в рамках некоторой \textit{формальной теории} только в том случае, когда ровно один его компонент является истинным в рамках этой же \textit{формальной теории}.}
\scnrelfrom{описание примера}{
\scnfilescg{figures/sd_logical_formulas/strictDisjunction.png}}
\scnaddlevel{1}
\scniselement{sc.g-текст}
\scnaddlevel{-1}
\scnrelfrom{описание примера}{
\scnfilescg{figures/sd_logical_formulas/strict_disjunction_triangle.png}}
\scnaddlevel{1}
\scnexplanation{Данная неатомарная логическая формула содержит следующую информацию: для любых переменных \_triangle если \_triangle является треугольником, то \_triangle является или тупоугольным треугольником, или остроугольным треугольником, или прямоугольным треугольником.}
\scniselement{sc.g-текст}
\scnaddlevel{-1}
\scnnote{\textbf{\textit{строгая дизъюнкция*}} может быть представлена как \textit{дизъюнкция} \textit{конъюнкции} \textit{отрицания} первой логической формулы и второй логической формулы и \textit{конъюнкции} первой логической формулы и \textit{отрицания} второй логической формулы. Также она может быть представлена и ввиде \textit{конъюнкции} \textit{дизъюнкций} двух логических формул и их \textit{отрицаний}.}
\scnaddlevel{1}
\scnrelfrom{описание примера}{	\scnfilescg{figures/sd_logical_formulas/strict_disjunction_representation.png}}
\scnaddlevel{1}
\scniselement{sc.g-текст}
\scnaddlevel{-1}
\scnaddlevel{-1}

\scnheader{импликация*}
\scnidtf{логическое следование*}
\scnsubset{логическая связка*}
\scniselement{бинарное отношение}
\scniselement{ориентированное отношение}
\scndefinition{\textbf{\textit{импликация*}} -- это множество импликативных \textit{неатомарных высказываний}, каждое из которых состоит из посылки (первый компонент \textit{высказывания}) и следствия (второй компонент \textit{высказывания}).}
\scnnote{Каждое импликативное \textit{высказывание} ложно в рамках некоторой \textit{формальной теории} в том случае, когда его посылка истинна, а заключение ложно в рамках этой же \textit{формальной теории}. В других случаях такое \textit{высказывание} истинно.}
\scnnote{По умолчанию на все переменные, входящие в обе части высказывания об \textbf{\textit{имликации*}} (или хотя бы одну из \textit{подформул*} каждой части) неявно накладывается квантор \textit{всеобщности*}, при условии, что эти переменные не связаны другим \textit{квантором}, указанным явно.}
\scnrelfrom{описание примера}{
\scnfilescg{figures/sd_logical_formulas/implication.png}}
\scnaddlevel{1}
\scniselement{sc.g-текст}
\scnaddlevel{-1}
\scnnote{\textbf{\textit{импликация*}} может быть представлена как \textit{дизъюнкция} \textit{отрицания} первой логической формулы и второй логической формулы или же как \textit{отрицание} \textit{конъюнкции} первой логической формулы и \textit{отрицания} второй логической формулы.}
\scnaddlevel{1}
\scnrelfrom{описание примера}{
\scnfilescg{figures/sd_logical_formulas/implication_representation.png}}
\scnaddlevel{1}
\scniselement{sc.g-текст}
\scnaddlevel{-1}
\scnaddlevel{-1}
\scnrelfrom{описание примера}{
\scnfilescg{figures/sd_logical_formulas/implication_triangle.png}}
\scnaddlevel{1}
\scnexplanation{Данная неатомарная логическая формула содержит следующую информацию: для любых переменных \_triangle и \_angle если \_triangle является прямоугольным треугольником, то синус его внутреннего угла \_angle равен единице.}
\scniselement{sc.g-текст}
\scnaddlevel{-1}

\scnheader{если\scnrolesign}
\scnidtf{посылка\scnrolesign}
\scnsubset{1\scnrolesign}
\scniselement{ролевое отношение}
\scndefinition{\textbf{\textit{если\scnrolesign}} -- это \textit{ролевое отношение}, используемое в связках \textit{импликации*} для указания посылки.}

\scnheader{то\scnrolesign}
\scnidtf{следствие\scnrolesign}
\scnsubset{2\scnrolesign}
\scniselement{ролевое отношение}
\scndefinition{\textbf{\textit{то\scnrolesign}} -- это \textit{ролевое отношение}, используемое в связках \textit{импликации*} для указания следствия.}

\scnheader{эквиваленция*}
\scnidtf{эквивалентность*}
\scnsubset{логическая связка*}
\scniselement{бинарное отношение}
\scniselement{неориентированное отношение}
\scndefinition{\textbf{\textit{эквиваленция*}} -- это множество \textit{высказываний} об эквивалентности, каждое из которых истинно в рамках некоторой \textit{формальной теории} только в тех случаях, когда оба его компонента одновременно либо истинны в рамках этой же \textit{формальной теории}, либо ложны.}
\scnnote{По умолчанию на все переменные, входящие в обе части высказывания об \textbf{\textit{эквиваленции*}} (или хотя бы одну из \textit{подформул*} каждой части) неявно накладывается квантор \textit{всеобщности*}, при условии, что эти переменные не связаны другим \textit{квантором}, указанным явно.}
\scnrelfrom{описание примера}{
\scnfilescg{figures/sd_logical_formulas/equivalent.png}}
\scnaddlevel{1}
\scniselement{sc.g-текст}
\scnaddlevel{-1}
\scnnote{\textbf{\textit{эквиваленция*}} двух логических формул может быть представлена как \textit{дизъюнкция} \textit{конъюнкции} этих двух логическх формул и \textit{конъюнкции} \textit{отрицаний} этих двух логических формул.}
\scnaddlevel{1}
\scnrelfrom{описание примера}{
\scnfilescg{figures/sd_logical_formulas/equivalence_representation.png}}
\scnaddlevel{1}
\scniselement{sc.g-текст} 
\scnaddlevel{-1}
\scnaddlevel{-1}

\scnheader{квантор}
\scnsubset{логическая связка*}
\scndefinition{\textbf{\textit{квантор}} — это \textit{отношение}, каждая связка которой задает истинность множества \textit{логических формул}, входящих в ее состав, при выполнении дополнительных условий, связанных с некоторыми из переменных, входящих в состав этих \textit{логических формул}.}
\scnnote{Будем говорить, что переменные связаны \textbf{\textit{квантором}} или попадают под область действия данного \textbf{\textit{квантора}} (имея в виду конкретную связку конкретного \textbf{\textit{квантора}}).}
\scnnote{В состав каждой связки каждого \textbf{\textit{квантора}} входит \textit{атомарная формула}, являющаяся \textit{тривиальной структурой}, в которой перечислены переменные, связанные данным \textbf{\textit{квантором}}.}

\scnheader{всеобщность*}
\scnidtf{квантор всеобщности*}
\scnidtf{квантор общности*}
\scniselement{квантор}
\scniselement{ориентированное отношение}
\scniselement{класс связок разной мощности}
\scndefinition{\textbf{\textit{всеобщность}} -- это \textit{квантор}, для каждой связки которого, истинной в рамках некоторой \textit{формальной теории}, выполняется следующее утверждение: все формулы, входящие в состав этой связки истинны в рамках этой же \textit{формальной теории} при всех (любых) возможных значениях всех элементов множества \textit{связываемых переменных\scnrolesign} входящего в эту связку.}
\scnnote{Каждая связка \textit{квантора} \textbf{\textit{всеобщность*}} может быть представлена как \textit{конъюнкция*} (потенциально бесконечная) исходных \textit{логических формул}, входящих в состав этой связки, в каждой из которых все \textit{связанные переменные\scnrolesign} заменены на их возможные значения.}
\scnnote{Квантор \textbf{\textit{всеобщности*}} зачастую обозначается \scnqq{$\forall$} \ и читается как \scnqq{для всех}, \scnqq{для каждого}, \scnqq{для любого} или \scnqq{все}, \scnqq{каждый}, \scnqq{любой}.}

\scnrelfrom{описание примера}{
\scnfilescg{figures/sd_logical_formulas/universality.png}}
\scnaddlevel{1}
\scniselement{sc.g-текст}
\scnaddlevel{-1}

\scnheader{формула существования}
\scnidtf{существование*}
\scnsubdividing{атомарная логическая формула;неатомарное существование*}

\scnheader{неатомарное существование*}
\scnidtf{квантор неатомарного существования*}
\scniselement{квантор}
\scniselement{ориентированное отношение}
\scniselement{класс связок разной мощности}
\scndefinition{\textbf{\textit{неатомарное существование*}} -- это \textit{квантор}, для каждой связки которого, истинной в рамках некоторой \textit{формальной теории}, выполняется следующее утверждение: существуют значения всех элементов множества \textit{связываемых переменных\scnrolesign} входящего в эту связку, такие, что все формулы, входящие в состав этой связки истинны в рамках этой же \textit{формальной теории}.}
\scnnote{Каждая связка \textit{квантора} \textbf{\textit{неатомарное существование*}} может быть представлена как \textit{дизъюнкция*} (потенциально бесконечная) исходных \textit{логических формул}, входящих в состав этой связки, в каждой из которых все \textit{связанные переменные\scnrolesign} заменены на их возможные значения.}
\scnnote{квантор \textbf{\textit{существования*}} зачастую обозначается \scnqq{$\exists$} \ и читается как \scnqq{существует}, \scnqq{для некоторого}, \scnqq{найдется}.}

\scnrelfrom{описание примера}{
\scnfilescg{figures/sd_logical_formulas/non_atomicExistence.png}}
\scnaddlevel{1}
\scniselement{sc.g-текст}
\scnaddlevel{-1}

\scnheader{число значений переменной}
\scniselement{параметр}
\scnexplanation{Каждый элемент \textit{параметра} \textbf{\textit{число значений переменной}} представляет собой класс ориентированных пар, первым компонентом которых является знак \textit{логической формулы}, вторым -- \textit{sc-переменная}, имеющая в рамках данной \textit{логической формулы} ограниченное известное число значений, при которых данная формула является истинной в рамках соответствующей \textit{формальной теории}.}
\scnnote{Отметим, что в случае \textit{атомарной логической формулы} каждая такая связка связывает знак формулы и знак принадлежащей ей \textit{sc-переменной}, т.е. является, по сути, частным случаем пары принадлежности. В случае \textit{неатомарной логической формулы} указанная \textit{sc-переменная} может принадлежать любой из \textit{подформул*} исходной формулы.}
\scnnote{\textit{измерением*} значения параметра \textbf{\textit{число значений переменной}} является некоторое \textit{число}, задающее количество значений \textit{sc-переменных} в рамках \textit{логической формулы}.}

\scnheader{кратность существования}
\scniselement{параметр}
\scnrelfrom{область определения параметра}{формула существования}
\scnhaselement{единственное существование}
\scnexplanation{Каждый элемент \textit{параметра} \textbf{\textit{кратность существования}} представляет собой класс логических \textit{формул существования}, для которых  при интерпретации на соответствующей \textit{предметной области} существует ограниченное общее для всех таких формул число комбинаций значений переменных, при которых указанные формулы являются истинными в рамках соответствующей \textit{формальной теории}.}
\scnaddlevel{1}
\scnnote{\textit{измерением*} каждого значения \textbf{\textit{кратности существования}} является некоторое \textit{число}, задающее количество таких комбинаций.}
\scnaddlevel{-1}

\scnheader{единственное существование}
\scnidtf{однократное существование}
\scnidtf{формула существования и единственности}
\scnnote{\textbf{\textit{единственное существование}} зачастую обозначается \scnqq{$\exists!$} \ и читается как \scnqq{существует и единственный}.}


\scnheader{логическая формула и единственность}
\scnsubset{логическая формула}
\scnsubset{единственное существование}
\scnexplanation{Каждый элемент множества \textbf{\textit{логическая формула и единственность}} представляет собой \textit{логическую формулу} (\textit{атомарную} или \textit{неатомарную}), для которой дополнительно уточняется, что при ее интерпретации на некоторой предметной области существует только один набор значений переменных, входящих в эту формулу (или ее \textit{подформулы*}), при котором указанная логическая формула истинна в рамках \textit{формальной теории}, в которую входит данная \textit{предметная область}.}

\scnheader{связываемые переменные\scnrolesign}
\scniselement{ролевое отношение}
\scndefinition{\textbf{\textit{связываемые переменные\scnrolesign}} -- это \textit{ролевое отношение}, которое связывает связку конкретного \textit{квантора} с множеством переменных, которые связаны этим квантором.}
%\scnrelfrom{описание примера}{
%\scnfilescg{figures/sd_logical_formulas/bindVariables.png}}

\scnheader{открытая логическая формула}
\scndefinition{\textbf{\textit{открытая логическая формула}} -- это \textit{логическая формула}, в рамках которой (и всех ее \textit{подформул*}) существует хотя бы одна переменная, не связанная никаким \textit{квантором}.}

\scnheader{замкнутая логическая формула}
\scndefinition{\textbf{\textit{замкнутая логическая формула}} -- это \textit{логическая формула}, в рамках которой (и всех ее \textit{подформул*}) не существует переменных, не связанных каким-либо \textit{квантором}.}

\scnheader{Примеры неатомарных логических формул}
\scneqtoset{\scgfileitem{figures/sd_logical_formulas/example_line_segment_sum.png}\\
\scnaddlevel{1}
\scnrelfrom{описание примера}{
\scnfilescg{figures/sd_logical_formulas/example_line_segment_sum_note.png}}
\scnexplanation{AB+BC=AC}
\scniselement{sc.g-текст}
\scnaddlevel{-1}
;
\scgfileitem{figures/sd_logical_formulas/example_line_segment_diff.png}\\
\scnaddlevel{1}
\scnrelfrom{описание примера}{
\scnfilescg{figures/sd_logical_formulas/example_line_segment_diff_note.png}}
\scnexplanation{AB-AC=CB}
\scniselement{sc.g-текст}
\scnaddlevel{-1}
}

\bigskip
\scnendstruct \scnendcurrentsectioncomment

\end{SCn}

\scsubsection[\scneditors{Садовский М.Е.;Никифоров С.А.}\protect\scnmonographychapter{Глава 2.6. Языковые средства формального описания синтаксиса и денотационной семантики различных языков в интеллектуальных компьютерных системах нового поколения}\protect\scnmonographychapter{Глава 4.1. Структура интерфейсов интеллектуальных компьютерных систем нового поколения}]{Предметная область и онтология файлов, внешних информационных конструкций и внешних языков ostis-систем}
\label{sd_file_internal_inform_struct}

\scsubsubsection[\scneditors{Никифоров С.А.;Бобёр  Е.С.}\protect\scnmonographychapter{Глава 2.6. Языковые средства формального описания синтаксиса и денотационной семантики различных языков в интеллектуальных компьютерных системах нового поколения}]{Предметная область и онтология естественных языков}
\label{sd_natural_languages}
\input{Contents/chapter2/sd_ostis_sys_models/sd_kb/sd_natural_languages.tex}

\scparagraph[\scneditors{Никифоров С.А.;Бобёр  Е.С.}\protect\scnmonographychapter{Глава 2.6. Языковые средства формального описания синтаксиса и денотационной семантики различных языков в интеллектуальных компьютерных системах нового поколения}]{Предметная область и онтология синтаксиса естественных языков}
\label{sd_syntax_natural_lang}

\scparagraph[\scneditors{Никифоров С.А.;Бобёр  Е.С.}\protect\scnmonographychapter{Глава 2.6. Языковые средства формального описания синтаксиса и денотационной семантики различных языков в интеллектуальных компьютерных системах нового поколения}]{Предметная область и онтология денотационной семантики естественных языков}
\label{sd_sem_natural_lang}

\scsubsection[\scneditors{Никифоров С.А.;Шункевич Д.В.}\protect\scnmonographychapter{Глава 3.1. Формализация понятий действия, задачи, метода, средства, навыка и технологии}]{Глобальная предметная область и онтология, описывающая воздействия, действия, методы, средства и технологии}
\label{sd_actions}
\input{Contents/chapter2/sd_ostis_sys_models/sd_kb/sd_actions.tex}

\scsubsubsection[\scnmonographychapter{Глава 3.1. Формализация понятий действия, задачи, метода, средства, навыка и технологии}]{Предметная область и онтология локальных предметных областей и онтологий действий}
\label{local_sd_actions}

\scsubsubsection[\scnidtf{Типология неавтоматизированных (\scnqq{вручную} выполняемых) и автоматически выполняемых \textit{действий}, направленных на управление процессами выполнения различных \textit{сложных действий}, а также система понятий, используемая для \textit{управления сложными действиями}}]{Предметная область и онтология действий по управлению деятельностью многоагентных систем}
\label{local_sd_project_management}

\scsectionfamily{Часть 3 Стандарта OSTIS. Многоагентные решатели задач интеллектуальных компьютерных систем нового поколения}
\label{part_solvers}

\scsection[\scneditor{Шункевич Д.В.}\protect\scnmonographychapter{Глава 3.2. Ситуационное управление обработкой знаний в интеллектуальных компьютерных системах нового поколения}]{Предметная область и онтология решателей задач ostis-систем}
\label{sd_ps}
\begin{SCn}

\scnsectionheader{Предметная область и онтология решателей задач компьютерных систем}

\scnstartsubstruct

\scnheader{Предметная область решателей задач современных интеллектуальных компьютерных систем}
\scnsdmainclasssingle{решатель задач компьютерных систем}
\scnsdclass{решатель задач компьютерных систем;гибридный решатель задач;объединенный решатель задач;многоагентная система}

\scnheader{решатель задач компьютерных систем}
\scnexplanation{Одним из ключевых компонентов интеллектуальной системы, обеспечивающим возможность решать широкий круг задач, является решатель задач. Особенностью решателей задач интеллектуальных систем по сравнению с другими современными программными системами является необходимость решать задачи в условиях, когда сведения, необходимые для решения задачи, не локализованы явно в базе знаний интеллектуальной системы и должны быть найдены в процессе решения задачи на основании каких-либо критериев}

\scnsuperset{объединенный решатель задач}
\scnaddlevel{1}
\scnrelfromlist{требования}{
	\scnfileitem{обеспечение основной функциональности системы (решение явно сформулированных задач по требованию)};
	\scnfileitem{обеспечение корректности и оптимизация работы системы (перманентно на протяжении жизненного цикла системы)};
	\scnfileitem{обеспечение автоматизации развития интеллектуальной системы}
}
\scnaddlevel{-1}
\scnsuperset{гибридный решатель задач}

\scntext{проблемы разработки}{Несмотря на то что в настоящее время существует большое число моделей решения задач, многие из которых реализованы и успешно используются на практике в различных системах, остается актуальной проблема низкой согласованности принципов, лежащих в основе реализации таких моделей, и отсутствия единой унифицированной основы для реализации и интеграции различных моделей, что приводит к тому, что:
\begin{scnitemize}
\item затруднена возможность одновременного использования различных моделей решения задач в рамках одной системы при решении одной и той же комплексной задачи; практически невозможно комбинировать различные модели с целью решения задачи, для которой априори отсутствует алгоритм ее
решения;
\item практически невозможно использовать технические решения, реализованные в одной системе, в других системах, т. е. возможности использования компонентного подхода при построении решателей задач сильно ограничены. Как следствие, велико количество дублирований аналогичных решений в разных системах;
\item фактически отсутствуют комплексные методики и средства построения решателей задач, которые бы обеспечивали возможность проектирования, реализации и отладки решателей различного вида.
\end{scnitemize}

Следствиями указанных проблем являются:
\begin{scnitemize}
\item высокая трудоемкость разработки каждого решателя, увеличение сроков их разработки, а значит, и увеличение затрат на разработку и поддержку соответствующих интеллектуальных систем;
\item высокая трудоемкость внесения изменений в уже разработанные решатели, т. е. отсутствует или сильно затруднена возможность дополнения уже разработанного решателя новыми компонентами и внесения изменений в уже существующие компоненты в процессе эксплуатации системы. Таким образом, высока трудоемкость поддержки разработанных решателей;
\item высокий уровень профессиональных требований к разработчикам решателей задач, что обусловлено, в частности:
\begin{scnitemizeii}
\item высокой сложностью существующих формализмов в области решения задач, рассчитанных на их интерпретацию компьютерной системой, а не человеком;
\item отсутствием возможности рассматривать разрабатываемые решатели на разных уровнях детализации, выделения на каждом уровне достаточно независимых компонентов, что затрудняет процесс проектирования, тестирования и отладки таких решателей, а также снижает эффективность попыток объединения разработчиков решателей в коллективы по причине увеличения накладных расходов на согласование их деятельности;
\item низким уровнем информационной поддержки разработчиков и автоматизации их деятельности.
\end{scnitemizeii}
\end{scnitemize}
}


\scnheader{гибридный решатель задач}
\scnexplanation{Расширение областей применения интеллектуальных систем требует от таких систем возможности решения комплексных задач, решение каждой из которых предполагает совместное использование целого ряда различных моделей представления знаний и различных моделей решения задач. Кроме того, решение комплексных задач предполагает использование общих информационных ресурсов (в предельном случае -- всей базы знаний интеллектуальной системы) различными компонентами решателя, ориентированными на решение различных подзадач. Поскольку решатель комплексных задач осуществляет интеграцию различных моделей решения задач, будем называть его \textbf{\textit{гибридным решателем задач}}.}
\scnrelfromset{требования}{
	\scnfileitem{обеспечение решения задач из оговоренного класса за оговоренное время, при этом результат решения задачи должен удовлетворять некоторым известным требованиям. Более детально рассмотрим некоторые положения, уточняющие сформулированное требование:
	\begin{scnitemize}
		\item для явно сформулированных задач система всегда должна давать какой-либо ответ за оговоренное время, при этом ответ может быть отрицательным (система не смогла решить поставленную задачу), возможно, с объяснением причин, по которым решение в текущий момент оказалось невозможным. Одним из факторов безуспешности решения является выход за рамки установленного промежутка времени;
		\item если явно сформулированная задача решена, то все информационные процессы, направленные на ее решение, должны быть уничтожены. Особенно актуальным данное требование становится в ситуации, когда для решения одной и той же задачи параллельно используются сразу несколько подходов и заранее неизвестно, какой из них приведет к результату раньше других;
		\item после решения задачи вся временная информация, сгенерированная в процессе решения этой задачи и имеющая ценность только в контексте решения указанной задачи, должна быть удалена из памяти.
	\end{scnitemize}};
	\scnfileitem{обеспечение возможности согласованного использования различных моделей решения задач при решении одной и той же комплексной задачи в случае необходимости};
	\scnfileitem{решатель должен быть легко модифицируемым, т. е. трудоемкость внесения изменений в уже разработанный решатель должна быть минимальна. Путями повышения модифицируемости решателя являются обеспечение локальности вносимых изменений, в том числе -- за счет стратификации решателя на независимые уровни и обеспечение максимальной независимости компонентов решателя друг от друга, а также наличие готовых компонентов, которые могут быть встроены в решатель при необходимости. При этом внесение изменений должно осуществляться непосредственно в процессе эксплуатации системы};
	\scnfileitem{для того чтобы интеллектуальная система имела возможность анализировать и оптимизировать имеющийся решатель задач, интегрировать в его состав новые компоненты (в том числе самостоятельно), оценивать важность тех или иных компонентов и применимость их для решения той или	иной задачи, спецификация решателя должна быть описана языком, понятным системе, например, при помощи тех же средств, что и обрабатываемые знания. Возможность интеллектуальной системы анализировать (верифицировать, корректировать, оптимизировать) собственные компоненты будем называть рефлексивностью}
}

\scnheader{многоагентная система}
\scntext{достоинства}{
\begin{scnitemize}
\item автономность (независимость) агентов в рамках такой системы, что позволяет локализовать изменения, вносимые в решатель при его эволюции, и снизить соответствующие трудозатраты;
\item децентрализация обработки, т.е. отсутствие единого контролирующего центра, что также позволяет локализовать вносимые в решатель изменения;
\item возможность параллельной работы разных информационных процессов, соответствующих как одному агенту, так и разным агентам, как следствие, -- возможность распределенного решения задач;
\item активность агентов и многоагентной системы в целом, дающая возможность при общении с такой системой не указывать явно способ решения поставленной задачи, а формулировать задачу в декларативном ключе.
\end{scnitemize}
}
\scntext{структура}{В общем случае для построения некоторой конкретной многоагентной системы необходимо уточнить следующие ее компоненты:
\begin{scnitemize}
\item модель собственно агента, входящего в состав такой системы, включая классификацию таких агентов и набор понятий, характеризующих каждый агент в рамках системы. В настоящее время наиболее популярной является модель BDI (belief-desire-intention), в рамках которой предполагается описывать на соответствующих языках \scnqq{убеждения}, \scnqq{желания} и \scnqq{намерения} каждого агента системы;
\item модель среды, в рамках которой находятся агенты, на события в которой они реагируют и в рамках которой могут осуществлять некоторые преобразования. Обзор разновидностей сред для многоагентных систем приводится в работе \scncite{Weyns2007};
\item модель коммуникации агентов, в рамках которой уточняется язык взаимодействия агентов (структура и классификация сообщений) и способ передачи сообщений между агентами. В настоящее время существует ряд стандартов, описывающих языки взаимодействия агентов, например, KQML \scncite{Finin1994} и ACL \scncite{ACL};
\item модель координации агентов, регламентирующая принципы их деятельности, в том числе, механизмы решения возможных конфликтов. В настоящее время основное число работ в области многоагентных систем направлено именно на разработку механизмов координации агентов, в числе которых выделение агентов более высокого уровня (метаагентов) \scncite{Hartung2008}, различные социально-психологические модели \scncite{Vasconcelos2009,Rumbell2012}, поведение на основе онтологий \scncite{Gorodetsky2015} и другие.
\end{scnitemize}}
\scntext{проблемы разработки}{
\begin{scnitemize}
\item жесткая ориентация большинства средств построения многоагентных систем на модель BDI (Belief-Desire-Intention) приводит к существенным накладным расходам, связанным с необходимостью выражения конкретной практической задачи в системе понятий BDI. В то же время, ориентация на модель BDI неявно провоцирует искусственное разделение языков, описывающих собственно компоненты BDI и знания агента о внешней среде, что приводит к отсутствию унификации представления и, соответственно, дополнительным накладным расходам;
\item большинство современных средств построения многоагентных систем ориентированы на представление знаний агента при помощи узкоспециализированных языков, зачастую не предназначенных для представления знаний в широком смысле. Речь при этом идет как о знаниях агента о себе самом (например, в соответствии с моделью BDI), так и знаниях о внешней среде. В некоторых подходах вначале строится онтология, для создания которой, однако, часто используются средства с низкой выразительной способностью, не предназначенные для построения онтологий \scncite{Evertsz2004,JADE2017}. В конечном итоге такой подход приводит к сильной ограниченности возможностей разработанных многоагентных систем и их несовместимости;
\item абсолютное большинство современных средств построения многоагентных систем предполагает, что взаимодействие агентов осуществляется путем обмена сообщениями непосредственно от агента к агенту. Такой подход обладает существенным недостатком, связанным с тем, что в этом случае каждый агент системы должен иметь достаточно полную информацию о других агентах в системе, что приводит к дополнительным затратам ресурсов, кроме того добавление или удаление одного или нескольких агентов приводит к необходимости оповещения об этом других агентов. Данная проблема решается путем организации общения агентов по принципу <<доски объявлений>> \scncite{Jagannathan1989}, предполагающему, что сообщения помещаются в некоторую общую для всех агентов область, при этом каждый агент в общем случае может не знать, какому из агентов адресовано сообщение и от какого из агентов получено то или иное сообщение. Однако, данный подход не исключает проблему, связанную с необходимостью разработки специализированного языка взаимодействия агентов, который в общем случае не связан с языком, на котором описываются знания агента о решаемых задачах и окружающей среде;
\item многие средства построения многоагентных систем построены таким образом, что логический уровень взаимодействия агентов жестко привязан к физическому уровню реализации многоагентной системы. Например, при передаче сообщений от агента к агенту разработчику многоагентной системы необходимо помимо семантически значимой информации указывать ip-адрес компьютера, на котором расположен агент-получатель, кодировку, с помощью которой закодирован текст сообщения и другую техническую информацию, обусловленную исключительно особенностями текущей реализации средств;
\item в большинстве подходов среда, с которой взаимодействуют агенты, уточняется отдельно разработчиком для каждой многоагентной системы, что с одной стороны, расширяет возможности применения соответствующих средств, но с другой стороны приводит к существенным накладным расходам и несовместимости таких многоагентных систем. Кроме того, в ряде случаев разработчик также обязан учитывать особенности технической реализации средств разработки в плане их стыковки с предполагаемой средой, в роли которой может выступать, например, локальная или глобальная сеть.
\end{scnitemize}}

\scnendstruct

\end{SCn}

\scsubsection[\scneditor{Шункевич Д.В.}\protect\scnmonographychapter{Глава 3.2. Ситуационное управление обработкой знаний в интеллектуальных компьютерных системах нового поколения}]{Предметная область и онтология действий, задач, планов, протоколов и методов, реализуемых ostis-системой, а также внутренних агентов, выполняющих эти действия}
\label{sd_agents}
\begin{SCn}

\scnsectionheader{\currentname}

\scnstartsubstruct

\scnheader{Предметная область и онтология действий,  задач, планов, протоколов и методов, реализуемых ostis-системой в ее памяти, а также внутренних агентов, выполняющих эти действия}
\scniselement{предметная область}
\scnsdmainclass{действие в sc-памяти;абстрактный sc-агент;sc-агент}
\scnsdclass{абстрактный sc-агент, не реализуемый на Языке SCP;абстрактный sc-агент, реализуемый на Языке SCP;Абстрактный программный sc-агент;неатомарный абстрактный sc-агент;атомарный абстрактный sc-агент;платформенно-независимый абстрактный sc-агент;платформенно-зависимый абстрактный sc-агент;внутренний абстрактный sc-агент;эффекторный абстрактный sc-агент;рецепторный абстрактный sc-агент;абстрактный sc-агент, не реализуемый на Языке SCP;абстрактный sc-агент, реализуемый на Языке SCP;
абстрактный sc-агент интерпретации scp-программ;абстрактный программный sc-агент;
абстрактный программный sc-агент, реализуемый на Языке SCP;абстрактный sc-метаагент;sc-агент;активный sc-агент;описание поведения sc-агента;тип блокировки;полная блокировка;блокировка на любое изменение;блокировка на удаление}
\scnsdrelation{декомпозиция абстрактного sc-агента*;ключевые sc-элементы sc-агента*;программа sc-агента*;первичное условие инициирования*;условие инициирования и результат*;блокировка*}

\scnheader{обработка информации в ostis-системах}
\scnrelfromlist{принципы, лежащие в основе}{
	\scnfileitem{В основе решателя задач каждой \textit{ostis-системы} лежит многоагентная система, агенты которой взаимодействуют между собой \uline{только}(!) через общую для них \textit{sc-память} посредством спецификации в этой памяти выполняемых ими \textit{действий в sc-памяти}. При этом пользователи \textit{ostis-системы} также считаются агентами этой системы. Кроме того, \textit{sc-агенты} делятся на внутренние, рецепторные и эффекторные. Взаимодействие между агентами через общую \textit{sc-память} сводится к следующим видам действий:
	\begin{scnenumerate}
		\item К использованию общедоступной для соответствующей группы sc-агентов части хранимой базы знаний;
		\item К формированию (генерации) новых фрагментов базы знаний и/или к корректировке (редактированию) каких-либо фрагментов доступной части базы знаний;
		\item К интеграции (погружению) новых и/или обновленных фрагментов в состав доступной части базы знаний;
	\end{scnenumerate}\\
	Подчеркнем, что sc-агенты не общаются между собой напрямую путем отправки сообщений, как это делается в большинстве современных подходов к построению многоагентных систем. Кроме того, sc-агенты имеют доступ к общей для них базе знаний за счет чего гарантируется семантическая совместимость (взаимопонимание) между агентами, включая и пользователей ostis-систем};
	\scnfileitem{Пользователь \textit{ostis-системы} не может сам непосредственно выполнить какое-либо действие в \mbox{sc-памяти}, но он может средствами пользовательского интерфейса инициировать построение (генерацию, формирование в \textit{sc-памяти}) \textit{sc-текста}, являющегося спецификацией \textit{действия в \mbox{sc-памяти}}, выполняемого либо одним \textit{атомарным sc-агентом} за один акт, либо одним \textit{атомарным sc-агентом} за несколько актов, либо коллективом \textit{sc-агентов} (\textit{неатомарным sc-агентом}). В спецификации каждого такого \textit{действия в sc-памяти}, инициированного пользователем, этот пользователь указывается как заказчик этого действия. Таким образом, пользователь \textit{ostis-системы} дает поручения (задания, команды) \textit{sc-агентам} этой системы на выполнение различных специфицируемых им действий в \textit{sc-памяти}.};
	\scnfileitem{Каждый \textit{sc-агент}, выполняя некоторое \textit{действие в sc-памяти}, должен \scnqq{помнить}, что \textit{sc-память}, над которой он работает, является общим ресурсом не только для него, но и для всех остальных \textit{\mbox{sc-агентов}}, работающих над этой же \textit{sc-памятью}. Поэтому \textit{sc-агент} должен соблюдать определенную этику поведения в коллективе таких \textit{sc-агентов}, которая должна минимизировать помехи, которые он создает другим \textit{sc-агентам}.};
	\scnfileitem{Деятельность каждого агента \textit{ostis-системы} дискретна и представляет собой множество элементарных действий (актов). При этом при выполнении каждого акта агент может устанавливать блокировки нескольких типов на фрагменты базы знаний. Указанные блокировки позволяют запретить другим агентам изменять указанный фрагмент базы знаний или вообще сделать его \scnqq{невидимым} для других агентов. Блокировки устанавливаются самим агентом при выполнении соответствующего акта и снимаются им же на последнем этапе выполнения этого акта или раньше, если это возможно.};
	\scnfileitem{Если некий \textit{sc-агент} выполняет некоторое \textit{действие в sc-памяти}, то он на время выполнения этого действия может:
	\begin{scnenumerate}
		\item Запретить другим \textit{sc-агентам} изменять состояние некоторых sc-элементов, хранимых в \textit{sc-памяти} -- удалять их, изменять тип;
		\item Запретить другим \textit{sc-агентам} добавлять или удалять элементы некоторых множеств, обозначаемых соответствующими \textit{sc-узлами};
		\item Запретить другим \textit{sc-агентам} доступ на просмотр некоторых \textit{sc-элементов}, то есть эти \textit{\mbox{sc-элементы}} становятся полностью \scnqq{невидимыми} (полностью заблокированными) для других \textit{sc-агентов} но только на время выполнения соответствующего действия.
	\end{scnenumerate}\\
	Указанные блокировки должны быть полностью сняты до завершения выполнения соответствующего действия. Подчеркнем, что число \textit{sc-элементов}, блокируемых на время выполнения некоторого действия, в основном входят атомарные и неатомарные связки, и не должны входить \textit{sc-узлы}, обозначающие бесконечные классы каких-либо сущностей, и, тем более, sc-узлы, обозначающие различные понятия (ключевые классы различных предметных областей).\\
	Этичное (неэгоистичное) поведение \textit{sc-агента}, касающееся блокировки \textit{sc-элементов} (то есть ограничения к ним доступа другим \textit{sc-агентам}) предполагает соблюдение следующих правил:
	\begin{scnenumerate}
		\item Не следует блокировать больше \textit{sc-элементов}, чем это необходимо для решения задачи;
		\item Как только для какого-либо \textit{sc-элемента} необходимость его блокировки отпадает до завершения выполнения соответствующего действия, этот \textit{sc-элемент} желательно сразу деблокировать (снять блокировку);
	\end{scnenumerate}\\
	Для того, чтобы \textit{sc-агент} имел возможность работы с каким-либо произвольным \textit{sc-элементом}, он должен либо убедиться в том, что этот \textit{sc-элемент} не входит во фрагмент базы знаний, входящий в \textit{полную блокировку}, либо убедиться в том, что эта блокировка не установлена самим этим агентом.\\
	Особой группой полностью заблокированных \textit{sc-элементов} (на время выполнения действия \textit{\mbox{sc-агентом}}) являются вспомогательные \textit{sc-элементы} (\scnqq{леса}), создаваемые только на время выполнения этого действия. Эти sc-элементы в конце выполнения действия должны не деблокироваться, а удаляться.};
	\scnfileitem{Если \textit{действие в sc-памяти}, выполняемое \textit{sc-агентом}, завершилось (т.е. стало прошлой сущностью), то \textit{sc-агент} оформляет результат этого \textit{действия}, указывая (1) удаленные \textit{sc-элементы} и (2) сгенерированные sc-элементы. Это необходимо, если по каким-либо причинам придется сделать откат этого \textit{действия}, т.е возвратиться к состоянию базы знаний до выполнения указанного \textit{действия}.}
}

\scnsegmentheader{Понятие действия в sc-памяти}
\scnstartsubstruct

\scnheader{действие в sc-памяти}
\scnidtf{внутреннее действие ostis-системы}
\scnidtf{действие, выполняемое в sc-памяти}
\scnidtf{действие, выполняемое в абстрактной унифицированной семантической памяти}
\scnidtf{действие, выполняемое машиной обработки знаний ostis-системы}
\scnidtf{действие, выполняемое агентом или коллективом агентов ostis-системы}
\scnidtf{информационный процесс над базой знаний, хранимой в sc-памяти}
\scnidtf{процесс решения информационной задачи в sc-памяти}
\scnsubset{процесс в sc-памяти}
\scnexplanation{Каждое \textbf{\textit{действие в sc-памяти}} обозначает некоторое преобразование, выполняемое некоторым \textit{sc-агентом} (или коллективом \textit{sc-агентов}) и ориентированное на преобразование \textit{sc-памяти}. Спецификация действия после его выполнения может быть включена в протокол решения некоторой задачи. 

Преобразование состояния базы знаний включает, в том числе и информационный поиск, предполагающий (1) локализацию в базе знаний ответа на запрос, явное выделение структуры ответа и (2) трансляцию ответа на некоторый внешний язык.

Во множество \textbf{\textit{действий в sc-памяти}} входят знаки действий самого различного рода, семантика каждого из которых зависит от конкретного контекста, т.е. ориентации действия на какие-либо конкретные объекты и принадлежности действия какому-либо конкретному классу действий.

Следует четко отличать:
\begin{scnitemize}
\item Каждое конкретное \textbf{\textit{действие в sc-памяти}}, представляющее собой некоторый переходный процесс, переводящий sc-память из одного состояния в другое;
\item Каждый тип \textbf{\textit{действий в sc-памяти}}, представляющий собой некоторый класс однотипных (в том или ином смысле) действий;
\item sc-узел, обозначающий некоторое конкретное \textbf{\textit{действие в sc-памяти}};
\item sc-узел, обозначающий структуру, которая является описанием, спецификацией, заданием, постановкой соответствующего действия.
\end{scnitemize}
}
\scnsuperset{действие в sc-памяти, инициируемое вопросом}
\scnsuperset{действие редактирования базы знаний ostis-системы}
\scnsuperset{действие установки режима ostis-системы}
\scnsuperset{действие редактирования файла, хранимого в sc-памяти}
\scnsuperset{действие интерпретации программы, хранимой в sc-памяти}

\scnheader{действие в sc-памяти, инициируемое вопросом}
\scnidtf{действие, направленное на формирование ответа на поставленный вопрос}
\scnsuperset{действие. cформировать заданный файл}
\scnsuperset{действие. cформировать заданную структуру}
\scnaddlevel{1}
	\scnsuperset{действие. верифицировать заданную структуру}
	\scnaddlevel{1}
		\scnsuperset{действие. установить истинность или ложность указываемого логического высказывания}
		\scnsuperset{действие. установить корректность или некорректность указываемой структуры}
		\scnsuperset{действие. сформировать структуру, описывающую некорректности, имеющиеся в указываемой структуре}
	\scnaddlevel{-1}
	\scnsuperset{действие. уточнить тип заданного sc-элемента}
	\scnaddlevel{1}
		\scnsuperset{действие. установить позитивность/негативность указываемой sc-дуги принадлежности или непринадлежности}
	\scnaddlevel{-1}
	\scnsuperset{действие. сформировать семантическую окрестность}
	\scnaddlevel{1}
		\scnsuperset{действие. сформировать полную семантическую окрестность указываемой сущности}
		\scnsuperset{действие. сформировать базовую семантическую окрестность указываемой сущности}
		\scnsuperset{действие. сформировать частную семантическую окрестность указываемой сущности}
	\scnaddlevel{-1}
	\scnsuperset{действие. сформировать структуру, описывающую связи между указываемыми сущностями}
	\scnaddlevel{1}
		\scnsuperset{действие. сформировать структуру, описывающую сходства указываемых сущностей}
		\scnsuperset{действие. сформировать структуру, описывающую различия указываемых сущностей}
	\scnaddlevel{-1}
	\scnsuperset{действие. сформировать структуру, описывающую способ решения указываемой задачи}
	\scnsuperset{действие. сформировать план генерации ответа на указанный вопрос}
	\scnsuperset{действие. сформировать протокол выполнения указываемого действия}
	\scnsuperset{действие. сформировать обоснование корректности указываемого решения}
	\scnsuperset{действие. верифицировать обоснование корректности указываемого решения}
	\scnsuperset{действие, направленное на установление темпоральных характеристик указываемой сущности}
	\scnsuperset{действие, направленное на установление пространственных характеристик указываемой сущности}
\scnaddlevel{-1}

\scnheader{действие редактирования базы знаний}
\scnsuperset{действие. изменить направление указанной sc-дуги}
\scnsuperset{действие. исправить ошибки в заданной структуре}
\scnsuperset{действие. преобразовать указанную структуру в соответствии с указанным правилом}
\scnsuperset{действие. отождествить два указанных sc-элемента}
\scnsuperset{действие. включить множество}
\scnaddlevel{1}
	\scnidtf{сделать все элементы множества \textbf{\textit{Si}} явно принадлежащими множеству \textbf{\textit{Sj}}, то есть сгенерировать соответствующие sc-дуги принадлежности}
\scnaddlevel{-1}
\scnsuperset{действие генерации sc-элементов}
\scnaddlevel{1}
	\scnsuperset{действие генерации, одним из аргументов которого является некоторая обобщенная структура}
	\scnaddlevel{1}
		\scnsuperset{действие. сгенерировать структуру, изоморфную указываемому образцу}
	\scnaddlevel{-1}
	\scnsuperset{действие. сгенерировать sc-элемент указанного типа}
	\scnaddlevel{1}
		\scnsuperset{действие. сгенерировать sc-коннектор указанного типа}
		\scnsuperset{действие. сгенерировать sc-узел указанного типа}
	\scnaddlevel{-1}
	\scnsuperset{действие. сгенерировать файл с заданным содержимым}
	\scnsuperset{действие. установить указанный файл в качестве основного идентификатора указанного sc-элемента для указанного внешнего языка}
\scnaddlevel{-1}
\scnsuperset{действие. обновить понятия}
\scnaddlevel{1}
	\scnidtf{действие. заменить неосновные понятия на их определения через основные понятия}
\scnaddlevel{-1}
\scnsuperset{действие. интегрировать информационную конструкцию в текущее состояние базы знаний}
\scnaddlevel{1}
	\scnsuperset{действие. интегрировать содержимое указанного файла в текущее состояние базы знаний}
	\scnaddlevel{1}
		\scnsuperset{действие. протранслировать содержимое указанного файла в sc-память}
	\scnaddlevel{-1}
	\scnsuperset{действие. интегрировать указанную структуру в текущее состояние базы знаний}
\scnaddlevel{-1}
\scnsuperset{действие. дополнить описание прошлого состояния ostis-системы}
\scnaddlevel{1}
	\scnsuperset{действие. дополнить структуру, описывающую историю эволюции ostis-системы}
	\scnsuperset{действие. дополнить структуру, описывающую историю эксплуатации ostis-системы}
\scnaddlevel{-1}

\scnsuperset{действие удаления sc-элементов}
\scnaddlevel{1}
    \scnsuperset{действие. удалить указанные sc-элементы}
   	\scnaddlevel{1}
	\scnsuperset{действие. удалить sc-элементы, входящие в состав указанной структуры и не являющиеся ключевыми узлами каких-либо sc-агентов}
	\scnaddlevel{-1}
\scnresetlevel

\scnheader{действие. отождествить два указанных sc-элемента}
\scnidtf{действие. совместить два указанных sc-элемента}
\scnidtf{действие. склеить два указанных sc-элемента}
\scnsubdividing{действие. физически отождествить два указанных sc-элемента;действие. логически отождествить два указанных sc-элемента}

\scnheader{действие. отождествить два указанных sc-элемента}
\scnexplanation{Каждое \textbf{\textit{действие. отождествить два указанных sc-элемента}} может быть выполнено как \textit{действие. физически отождествить два указанных sc-элемента} или \textit{действие. логически отождествить два указанных sc-элемента}. В случае логического отождествления в протоколе деятельности агентов сохраняется само действие с его спецификацией, включающей обязательное указание того, какие элементы были сгенерированы, а какие удалены. В случае физического отождествления протокол действия не сохраняется.}

\scnheader{действие. обновить понятия}
\scnidtf{действие. заменить некоторое множество понятий на другое множество понятий}
\scnexplanation{Каждое \textbf{\textit{действие. обновить понятия}} обозначает переход от какой-то группы понятий, использовавшихся ранее, к другой группе понятий, которые будут использоваться вместо первых, и станут \textit{основными понятиями}.
В общем случае \textbf{\textit{действие. обновить понятия}} состоит из следующих этапов:

\begin{scnitemize}
    \item Определить заменяемые понятия на основе заменяющих;
    \item Внести соответствующие изменения в программы sc-агентов, ключевыми узлами которых являются обновляемые понятия;
    \item Заменить все конструкции в базе знаний, содержащие заменяемые понятия, в соответствии с определениями этих понятий через заменяющие их понятия;
    \item При необходимости,\textit{ sc-элементы}, обозначающие замененные таким образом понятия, могут быть полностью выведены из текущего состояния базы знаний.
\end{scnitemize}

Первым аргументом (входящим в знак \textit{действия} под атрибутом \textit{1’}) \textbf{\textit{действия. обновить понятия}} является знак множества \textit{sc-узлов}, обозначающих заменяемые понятия, вторым (входящим в знак \textit{действия} под атрибутом \textit{2’}) - знак множества \textit{sc-узлов}, обозначающих заменяющие понятия. В общем случае любое или оба этих множества могут быть \textit{синглетонами}.}

\scnheader{действие. удалить указанные sc-элементы}
\scnsubdividing{действие. физически удалить указанные sc-элементы;действие. логически удалить указанные sc-элементы}
\scnexplanation{Каждое \textbf{\textit{действие. удалить указанные sc-элементы}} может быть выполнено как \textit{действие. физически удалить указанные sc-элементы} или \textit{действие. логически удалить указанные sc-элементы}. В случае логического удаления в протоколе деятельности агентов сохраняется само действие с его спецификацией, включающей обязательное указание того, какие элементы были удалены, т.е. по сути, элементы просто исключаются из текущего состояния базы знаний. В случае физического удаления протокол действия не сохраняется.

В случае удаления какого-либо \textit{sc-элемента}, инцидентные ему \textit{связки}, в том числе \textit{sc-коннекторы}, также удаляются.}

\scnheader{действие. интегрировать указанную структуру в текущее состояние базы знаний}
\scnexplanation{Для того, чтобы выполнить \textbf{\textit{действие. интегрировать указанную структуру в текущее состояние базы знаний}}, необходимо склеить \textit{sc-элементы}, входящие в интегрируемую \textit{структуру} с синонимичными им \textit{sc-элементами}, входящими в текущее состояние базы знаний, заменить неиспользуемые (например, устаревшие) понятия, входящие в интегрируемую \textit{структуру}, на используемые (т.е. заменить неиспользуемые понятия на их определения через используемые), явно включить все элементы интегрируемой \textit{структуры} в число элементов утвержденной части базы знаний и явно включить все элементы интегрируемой \textit{структуры} в число элементов одного из атомарных разделов утвержденной части базы знаний.}

\scnheader{действие интерпретации программы, хранимой в sc-памяти}
\scnsuperset{действие интерпретации scp-программы}

\scnheader{задача, решаемая в sc-памяти}
\scnsubset{задача}
\scnidtf{спецификация действия, выполняемого в sc-памяти}
\scnidtf{структура, являющая таким описанием (постановкой, заданием) соответствующего действия в sc-памяти, которое обладает достаточной полнотой для выполения указанного действия}
\scnidtf{семантическая окрестность некоторого действия в sc-памяти, обеспечивающая достаточно полное задание этого действия}

\scnheader{класс действий}
\scnsuperset{класс действий в sc-памяти}
\scnaddlevel{1}
	\scnrelto{семейство подмножеств}{действие в sc-памяти}
\scnaddlevel{-1}
\scnsubdividing{класс логически атомарных действий\\
	\scnaddlevel{1}
		\scnidtf{класс автономных действий}
	\scnaddlevel{-1};	
	класс логически неатомарных действий\\
	\scnaddlevel{1}
	\scnidtf{класс неавтономных действий}
	\scnaddlevel{-1}}

\scnheader{класс логически атомарных действий}
\scnexplanation{Каждое \textit{действие}, принадлежащее некоторому конкретному \textit{классу логически атомарных действий}, обладает двумя необходимыми свойствами:
\begin{scnitemize}
	\item выполнение действия не зависит от того, является ли указанное действие частью декомпозиции более общего действия. При выполнении данного действия также не должен учитываться тот факт, что данное действие предшествует каким-либо другим действиям или следует за ними (что явно указывается при помощи отношения \textit{последовательность действий*});
	\item указанное действие должно представлять собой логически целостный акт преобразования, например, в семантической памяти. Такое действие по сути является транзакцией, т. е. результатом такого преобразования становится новое состояние преобразуемой системы, а выполняемое действие должно быть либо выполнено полностью, либо не выполнено совсем, частичное выполнение не допускается. 
\end{scnitemize}

В то же время логическая атомарность не запрещает декомпозировать выполняемое действие на более частные, каждое из которых, в свою очередь, также будет являться логически атомарным.}
\scnsuperset{класс логически атомарных действий в sc-памяти}
\scnaddlevel{1}
	\scnexplanation{На логически атомарные действия предлагается делить всю деятельность, направленную на решение каких-либо задач ostis-системой. Соответственно \textit{решатель задач ostis-системы} предлагается делить на компоненты, соответствующие таким \textit{классам логически атомарных действий в sc-памяти}, что является основой для обеспечения его \textit{модифицируемости}.}
\scnaddlevel{-1}

\bigskip
\scnendstruct \scnendsegmentcomment{Понятие действия в sc-памяти}

\scnsegmentheader{Понятие sc-агента и абстрактного sc-агента}

\scnstartsubstruct

\scnheader{sc-агент}
\scnidtf{единственный вид \textit{субъектов}, выполняющих преобразования в \textit{\textit{sc-памяти}}}
\scnidtf{\textit{субъект}, способный выполнять \textit{действия в sc-памяти}, принадлежащие некоторому определенному \textit{классу логически атомарных действий}}
\scnexplanation{Логическая атомарность выполняемых sc-агентом действий предполагает, что каждый sc-агент реагирует на соответствующий ему класс ситуаций и/или событий, происходящих в sc-памяти, и осуществляет определенное преобразование sc-текста, находящегося в семантической окрестности обрабатываемой ситуации и/или события. При этом каждый sc-агент в общем случае не имеет информацию о том, какие еще sc-агенты в данный момент присутствуют в системе и осуществляет взаимодействие в другими sc-агентами исключительно посредством формирования некоторых конструкций (как правило – спецификаций действий) в общей sc-памяти. Таким сообщением может быть, например, вопрос, адресованный другим sc-агентам в системе (заранее не известно, каким конкретно), или ответ на поставленный другими sc-агентами вопрос (заранее не известно, каким конкретно). Таким образом, каждый sc-агент в каждый момент времени контролирует только фрагмент базы знаний в контексте решаемой данным агентом задачи, состояние всей остальной базы знаний в общем случае непредсказуемо для sc-агента.}

\scnheader{абстрактный sc-агент}
\scnnote{Поскольку предполагается, что копии одного и того же \textit{sc-агента} или функционально эквивалентные \textit{sc-агенты} могут работать в разных ostis-системах, будучи при этом физически разными sc-агентами, то целесообразно рассматривать свойства и классификацию не sc-агентов, а классов функционально эквивалентных sc-агентов, которые будем называть \textit{абстрактными sc-агентами}.}
\scnexplanation{Под \textbf{\textit{абстрактным sc-агентом}} понимается некоторый класс функционально эквивалентных \textit{sc-агентов}, разные экземпляры (т.е. представители) которого могут быть реализованы по-разному.

Каждый \textbf{\textit{абстрактный sc-агент}} имеет соответствующую ему спецификацию. В спецификацию каждого \textbf{\textit{абстрактного sc-агента}} входит:
\begin{scnitemize}
    \item указание ключевых \textit{sc-элементов} этого \textit{sc-агента}, т.е. тех \textit{sc-элементов}, хранимых в \textit{sc-памяти}, которые для данного \textit{sc-агента} являются «точками опоры»;
    \item формальное описание условий инициирования данного \textit{sc-агента}, т.е. тех \textit{ситуация} в \textit{sc-памяти}, которые инициируют деятельность данного \textit{sc-агента};
    \item формальное описание первичного условия инициирования данного \textit{sc-агента}, т.е. такой ситуации в \textit{sc-памяти}, которая побуждает \textit{sc-агента} перейти в активное состояние и начать проверку наличия своего полного условия инициирования (для \textit{внутренних абстрактных sc-агентов});
    \item строгое, полное, однозначно понимаемое описание деятельности данного \textit{sc-агента}, оформленное при помощи каких-либо понятных, общепринятых средств, не требующих специального изучения, например на естественном языке.
    \item описание результатов выполнения данного \textit{sc-агента}.
\end{scnitemize}
}
\scnsubdividing{неатомарный абстрактный sc-агент;атомарный абстрактный sc-агент}
\scnsubdividing{внутренний абстрактный sc-агент;эффекторный абстрактный sc-агент;рецепторный абстрактный sc-агент}
\scnsubdividing{абстрактный sc-агент, не реализуемый на Языке SCP;абстрактный sc-агент, реализуемый на Языке SCP}
\scnsubdividing{абстрактный sc-агент интерпретации scp-программ;абстрактный программный sc-агент;абстрактный sc-метаагент}
\scnsubdividing{платформенно-зависимый абстрактный sc-агент\\
\scnaddlevel{1}
\scnsuperset{абстрактный sc-агент, не реализуемый на Языке SCP}
\scnaddlevel{-1}
;платформенно-независимый абстрактный sc-агент}

\scnheader{абстрактный sc-агент, не реализуемый на Языке SCP}
\scnidtf{абстрактный sc-агент, который не может быть реализован на платформенно-независимом уровне}
\scnsubdividing{эффекторный абстрактный sc-агент;рецепторный абстрактный sc-агент
;абстрактный sc-агент интерпретации scp-программ}

\scnheader{абстрактный sc-агент, реализуемый на Языке SCP}
\scnidtf{абстрактный sc-агент, который может быть реализован на платформенно-независимом уровне}
\scnsubdividing{абстрактный sc-метаагент;абстрактный программный sc-агент, реализуемый на Языке SCP}

\scnheader{абстрактный программный sc-агент}
\scnsubdividing{эффекторный абстрактный sc-агент;рецепторный абстрактный sc-агент
;абстрактный программный sc-агент, реализуемый на Языке SCP}

\scnheader{неатомарный абстрактный sc-агент}
\scnexplanation{Под \textbf{\textit{неатомарным абстрактным sc-агентом}} понимается \textit{абстрактный sc-агент}, который декомпозируется на коллектив более простых \textit{абстрактных sc-агентов}, каждый из которых в свою очередь может быть как \textit{атомарным абстрактным sc-агентом}, так и \textbf{\textit{неатомарным абстрактным sc-агентом}}. При этом в каком либо варианте \textit{декомпозиции абстрактного sc-агента*} дочерний \textbf{\textit{неатомарный абстрактный sc-агент}} может стать \textit{атомарным абстрактным sc-агентом}, и реализовываться соответствующим образом.}

\scnheader{атомарный абстрактный sc-агент}
\scnexplanation{Под \textbf{\textit{атомарным абстрактным sc-агентом}} понимается \textit{абстрактный sc-агент}, для которого уточняется платформа его реализации, т.е. существует соответствующая связка отношения \textit{программа sc-агента*}.}
\scnsubdividing{платформенно-независимый абстрактный sc-агент;платформенно-зависимый абстрактный sc-агент}

\scnheader{платформенно-независимый абстрактный sc-агент}
\scnexplanation{К \textbf{\textit{платформенно-независимым абстрактным \mbox{sc-агентам}}} относят \textit{атомарные абстрактные sc-агенты}, реализованные на базовом языке программирования Технологии OSTIS, т.е. на \textit{Языке SCP}.

При описании \textbf{\textit{платформенно-независимых абстрактных sc-агентов}} под платформенной независимостью понимается платформенная независимость с точки зрения Технологии OSTIS, т.е реализация на специализированном языке программирования, ориентированном на обработку семантических сетей (\textit{Языке SCP}), поскольку \textit{атомарные sc-агенты}, реализованные на указанном языке могут свободно переноситься с одной платформы интерпретации \textit{sc-моделей} на другую. При этом языки программирования, традиционно считающиеся платформенно-независимыми в данном случае не могут считаться таковыми.

Существуют \textit{sc-агенты}, которые принципиально не могут быть реализованы на платформенно-независимом уровне, например, собственно \textit{sc-агенты} интерпретации \textit{sc-моделей} или рецепторные и эффекторные \textit{sc-агенты}, обеспечивающие взаимодействие с внешней средой.}

\scnheader{платформенно-зависимый абстрактный sc-агент}
\scnexplanation{К \textbf{\textit{платформенно-зависимым абстрактным sc-агентам}} относят \textit{атомарные абстрактные sc-агенты}, реализованные ниже уровня sc-моделей, т.е. не на \textit{Языке SCP}, а на каком-либо другом языке описания программ.

Существуют \textit{sc-агенты}, которые принципиально должны быть реализованы на платформенно-зависимом уровне, например, собственно \textit{sc-агенты} интерпретации \textit{sc-моделей} или рецепторные и эффекторные \textit{sc-агенты}, обеспечивающие взаимодействие с внешней средой.}

\scnheader{внутренний абстрактный sc-агент}
\scnexplanation{Каждый \textbf{\textit{внутренний абстрактный sc-агент}} обозначает класс \textit{sc-агентов}, которые реагируют на события в \textit{sc-памяти} и осуществляют преобразования исключительно в рамках этой же \textit{sc-памяти}.}

\scnheader{эффекторный абстрактный sc-агент}
\scnexplanation{Каждый \textbf{\textit{эффекторный абстрактный sc-агент}} обозначает класс \textit{sc-агентов}, которые реагируют на события в \textit{sc-памяти} и осуществляют преобразования во внешней относительно данной \textit{ostis-системы} среде.}

\scnheader{рецепторный абстрактный sc-агент}
\scnexplanation{Каждый \textbf{\textit{рецепторный абстрактный sc-агент}} обозначает класс \textit{sc-агентов}, которые реагируют на события во внешней относительно данной \textit{ostis-системы} среде и осуществляют преобразования в памяти данной системы.}

\scnheader{абстрактный sc-агент, не реализуемый на Языке SCP}
\scnexplanation{Каждый \textbf{\textit{абстрактный sc-агент, не реализуемый на Языке SCP}} должен быть реализован на уровне платформы интерпретации sc-моделей, в том числе, аппаратной. К таким \textit{абстрактным sc-агентам} относятся абстрактные sc-агенты интерпретации scp-программ, а также эффекторные и рецепторные абстрактные sc-агенты.}

\scnheader{абстрактный sc-агент, реализуемый на Языке SCP}
\scnexplanation{Каждый \textbf{\textit{абстрактный sc-агент, реализуемый на Языке SCP}} может быть реализован на Языке SCP, то есть платформенно-независимом уровне, но при необходимости, может реализовываться и на уровне платформы, например, с целью повышения производительности.}

\scnheader{абстрактный sc-агент интерпретации scp-программ}
\scnexplanation{К \textbf{\textit{абстрактным sc-агентам интерпретации scp-программ}} относятся не реализуемые на платформенно-независимом уровне \textit{абстрактные sc-агенты}, обеспечивающие интерпретацию \textit{scp-программ} и \textit{\mbox{scp-метапрограмм}}, в том числе создание \textit{scp-процессов}, собственно интерпретацию \textit{scp-операторов}, а также другие вспомогательные действия. По сути, агенты данного класса обеспечивают работу sc-агентов более высоких уровней (программных sc-агентов и sc-метаагентов), реализованных на Языке SCP, в частности, обеспечивают соблюдение указанными агентами общих принципов синхронизации.}

\scnheader{абстрактный программный sc-агент}
\scnexplanation{К \textbf{\textit{абстрактным программным sc-агентам}} относятся все \textit{абстрактные sc-агенты}, обеспечивающие основной функционал системы, то есть ее возможность решать те или иные задачи. Агенты данного класса должны работать в соответствии с общими принципами синхронизации деятельности субъектов в sc-памяти.}

\scnheader{абстрактный sc-метаагент}
\scnexplanation{Задачей \textbf{\textit{абстрактных sc-метаагентов}} является координация деятельности \textit{абстрактных программных sc-агентов}, в частности, решение проблемы взаимоблокировок. Агенты данного класса могут быть реализованы на Языке SCP, однако для синхронизации их деятельности используются другие принципы, соответственно, для реализации таких агентов требуется Язык SCP другого уровня, типология операторов которого полностью аналогична типологии scp-операторов, однако эти операторы имеют другую операционную семантику, учитывающую отличия в принципах синхронизации (работы с \textit{блокировками*}). Программы такого языка будем называть \textit{scp-метапрограммами}, соответствующие им \mbox{\textit{процессы в sc-памяти} – \textit{scp-метапроцессами}}, операторы – \textit{scp-метаоператорами}.}

\scnheader{декомпозиция абстрактного sc-агента*}
\scniselement{отношение декомпозиции}
\scnexplanation{Отношение \textbf{\textit{декомпозиции абстрактного sc-агента*}} трактует \textit{неатомарные абстрактные sc-агенты} как коллективы более простых \textit{абстрактных sc-агентов}, взаимодействующих через \textit{sc-память}.

Другими словами, \textbf{\textit{декомпозиция абстрактного sc-агента*}} на \textit{абстрактные sc-агенты} более низкого уровня уточняет один из возможных подходов к реализации этого \textit{абстрактного sc-агента} путем построения коллектива более простых \textit{абстрактных sc-агентов}.}

\scnheader{sc-агент}
\scnidtf{агент над sc-памятью}
\scnsubset{субъект}
\scnrelfrom{семейство подмножеств}{абстрактный sc-агент}
\scnexplanation{Под \textbf{\textit{sc-агентом}} понимается конкретный экземпляр (с теоретико-множественной точки зрения - элемент) некоторого \textit{атомарного абстрактного sc-агента}, работающий в какой-либо конкретной интеллектуальной системе.

Таким образом, каждый \textit{sc-агент} - это субъект, способный выполнять некоторый класс однотипных действий либо только над \textit{sc-памятью}, либо над sc-памятью и внешней средой (для эффекторных \textit{sc-агентов}). Каждое такое действие инициируется либо состоянием или ситуацией в sc-памяти, либо состоянием или ситуацией во внешней среде (для рецепторных sc-агентов-датчиков),  соответствующей условию инициирования \textit{атомарного абстрактного sc-агента}, экземпляром которого является заданный \textit{sc-агент}. В данном случае можно провести аналогию между принципами объектно-ориентированного программирования, рассматривая \textit{атомарный абстрактный sc-агент} как класс, а конкретный \textit{sc-агент} – как экземпляр, конкретную имплементацию этого класса.

Взаимодействие \textit{sc-агентов} осуществляется только через \textit{sc-память}. Как следствие, результатом работы любого \textit{sc-агента} является некоторое изменение состояния \textit{sc-памяти}, т.е. удаление либо генерация каких-либо \textit{sc-элементов}.

В общем случае один \textit{sc-агент} может явно передать управление другому \textit{sc-агенту}, если этот \textit{sc-агент} априори известен. Для этого каждый \textit{sc-агент} в \textit{sc-памяти} имеет обозначающий его \textit{sc-узел}, с которым можно связать конкретную ситуацию в текущем состоянии базы знаний, которую инициируемый \textit{sc-агент} должен обработать.

Однако далеко не всегда легко определить того \textit{sc-агента}, который должен принять управление от заданного \textit{sc-агента}, в связи с чем описанная выше ситуация возникает крайне редко. Более того, иногда условие инициирования \textit{sc-агента} является результатом деятельности непредсказуемой группы \textit{sc-агентов}, равно как и одна и та же конструкция может являться условием инициирования целой группы \textit{sc-агентов}.

При этом общаются через \textit{sc-память} не \textit{программы sc-агентов*}, а сами описываемые данными программами \textit{sc-агенты}.

В процессе работы \textit{sc-агент} может сам для себя порождать вспомогательные \textit{sc-элементы}, которые сам же удаляет после завершения акта своей деятельности (это вспомогательные \textit{структуры}, которые используются в качестве \scnqq{информационных лесов} только в ходе выполнения соответствующего акта деятельности и после завершения этого акта удаляются).}

\scnheader{активный sc-агент}
\scnsubset{sc-агент}
\scnexplanation{Под \textbf{\textit{активным sc-агентом}} понимается \textit{sc-агент} ostis-системы, который реагирует на события, соответствующие его условию инициирования, и, как следствие, его \textit{первичному условию инициирования*}. Не входящие во множество \textbf{\textit{активных sc-агентов}} \textit{sc-агенты} не реагируют ни на какие события в \textit{sc-памяти}.}

\scnheader{ключевые sc-элементы sc-агента*}
\scnexplanation{Связки отношения \textbf{\textit{ключевые sc-элементы sc-агента*}} связывают между собой \textit{sc-узел}, обозначающий \textit{абстрактный sc-агент} и \textit{sc-узел}, обозначающий множество \textit{sc-элементов}, которые являются ключевыми для данного \textit{абстрактного sc-агента}, то данные \textit{sc-элементы} явно упоминаются в рамках программ, реализующих данный \textit{абстрактный sc-агент}.}

\scnheader{программа sc-агента*}
\scnexplanation{Связки отношения \textbf{\textit{программа sc-агента*}} связывают между собой \textit{sc-узел}, обозначающий \textit{атомарный абстрактный sc-агент} и \textit{sc-узел}, обозначающий множество программ, реализующих указанный \textit{атомарный абстрактный sc-агент}. В случае \textit{платформенно-независимого абстрактного sc-агента} каждая связка отношения \textit{программа sc-агента*} связывает \textit{sc-узел}, обозначающий указанный \textit{абстрактный sc-агент} с множеством \textit{scp-программ}, описывающих деятельность данного \textit{абстрактного sc-агента}. Данное множество содержит одну \textit{агентную scp-программу}, и произвольное количество (может быть, и ни одной) \textit{scp-программ}, которые необходимы для выполнения указанной \textit{агентной scp-программы}.

В случае \textit{платформенно-зависимого абстрактного sc-агента} каждая связка отношения \textit{программа \mbox{sc-агента*}} связывает \textit{sc-узел}, обозначающий указанный \textit{абстрактный sc-агент} с множеством файлов, содержащих исходные тексты программы на некотором внешнем языке программирования, реализующей деятельность данного \textit{абстрактного sc-агента}.}

\scnheader{первичное условие инициирования*}
\scnexplanation{Связки отношения \textbf{\textit{первичное условие инициирования*}} связывают между собой \textit{sc-узел}, обозначающий \textit{абстрактный sc-агент} и бинарную ориентированную пару, описывающую первичное условие инициирования данного \textit{абстрактного sc-агента}, т.е. такой спецификацию \textit{ситуации} в \textit{sc-памяти}, возникновение которой побуждает \textit{sc-агента} перейти в активное состояние и начать проверку наличия своего полного условия инициирования.

Первым компонентом данной ориентированной пары является знак некоторого класса \textit{элементарных событий в sc-памяти*}, например, \textit{событие добавления sc-дуги, выходящей из заданного sc-элемента*}.

Вторым компонентом данной ориентированной пары является произвольный в общем случае \textit{sc-элемент}, с которым непосредственно связан указанный тип события в \textit{sc-памяти}, т.е., например, \textit{sc-элемент}, из которого выходит либо в который входит генерируемая либо удаляемая \textit{sc-дуга}, либо \textit{файл}, содержимое которого было изменено.

После того, как в \textit{sc-памяти} происходит некоторое событие, активизируются все \textit{активные sc-агенты}, \textbf{\textit{первичное условие инициирования*}} которых соответствует произошедшему событию.}

\scnheader{условие инициирования и результат*}
\scnexplanation{Связки отношения \textbf{\textit{условие инициирования и результат*}} связывают между собой \textit{sc-узел}, обозначающий \textit{абстрактный sc-агент} и бинарную ориентированную пару, связывающую условие инициирования данного \textit{абстрактного sc-агента} и результаты выполнения данного экземпляров данного \textit{sc-агента} в какой-либо конкретной системе.

Указанную ориентированную пару можно рассматривать как логическую связку импликации, при этом на \textit{sc-переменные}, присутствующие в обеих частях связки, неявно накладывается квантор всеобщности, на \textit{sc-переменные}, присутствующие либо только в посылке, либо только в заключении неявно накладывается квантор существования.

Первым компонентом указанной ориентированной пары является логическая формула, описывающая условие инициирования описываемого \textit{абстрактного sc-агента}, то есть конструкции, наличие которой в \textit{sc-памяти} побуждает \textit{sc-агент} начать работу по изменению состояния \textit{sc-памяти}. Данная логическая формула может быть как атомарной, так и неатомарной, в которой допускается использование любых связок логического языка.

Вторым компонентом указанной ориентированной пары является логическая формула, описывающая возможные результаты выполнения описываемого абстрактного \textit{sc-агента}, то есть описание произведенных им изменений состояния \textit{sc-памяти}. Данная логическая формула может быть как атомарной, так и неатомарной, в которой допускается использование любых связок логического языка.}

\scnheader{описание поведения sc-агента}
\scnsubset{семантическая окрестность}
\scnexplanation{\textbf{\textit{описание поведения sc-агента}} представляет собой \textit{семантическую окрестность}, описывающую деятельность \textit{sc-агента} до какой-либо степени детализации, однако такое описание должно быть строгим, полным и однозначно понимаемым. Как любая другая \textit{семантическая окрестность}, \textbf{\textit{описание поведения sc-агента}} может быть протранслировано на какие-либо понятные, общепринятые средства, не требующие специального изучения, например на естественный язык.\\
Описываемый \textit{абстрактный sc-агент} входит в соответствующее \textbf{\textit{описание поведения sc-агента}} под атрибутом \textit{ключевой sc-элемент'}.}

\bigskip
\scnendstruct \scnendsegmentcomment{Понятие sc-агента и абстрактного sc-агента}

\scnsegmentheader{Принципы синхронизации деятельности sc-агентов}

\scnstartsubstruct

\scnheader{процесс в sc-памяти}
\scnnote{Понятия \textit{действие в sc-памяти}, и \textit{процесс в sc-памяти} (информационный процесс, выполняемый агентом в семантической памяти), являются синонимичными, поскольку все процессы, протекающие в sc-памяти, являюся осознанными и выполняются каким-либо sc-агентами. Тем не менее, когда идет речь о синхронизации выполнения каких-либо преобразований в памяти компьютерной системы, в литературе принято использовать именно термины ``процесс'', ``взаимодействие процессов'' \cite{Dijkstra1972,Hoare1989}, в связи с чем будем использовать этот термин при описании принципов синхронизации деятельности sc-агентов при выполнении ими параллельных процессов в sc-памяти.}
\scnsubdividing{процесс в sc-памяти, соответствующий платформенно-зависимому sc-агенту;scp-процесс}
\scnsubdividing{scp-процесс, не являющийся scp-метапроцессом;scp-метапроцесс}

\scnheader{процесс в sc-памяти, соответствующий платформенно-зависимому sc-агенту}
\scnsubdividing{процесс в sc-памяти, соответствующий платформенно-зависимому sc-агенту и не являющийся действием абстрактной scp-машины;действие абстрактной scp-машины\\
\scnaddlevel{1}
	\scnsuperset{действие интерпретации scp-программы}
\scnaddlevel{-1}
}

\scnheader{блокировка*}
\scniselement{бинарное отношение}
\scnexplanation{Для синхронизации выполнения \textit{процессов в sc-памяти} используется механизм блокировок. Отношение \textbf{\textit{блокировка*}} связывает знаки \textit{действий в sc-памяти} со знаками \textit{структур} (ситуативных), которые содержат элементы, заблокированные на время выполнения данного действия или на какую-то часть этого периода. Каждая такая \textit{структура} принадлежит какому-либо из \textit{типов блокировки}.

Первым компонентом связок отношения \textbf{\textit{блокировка*}} является знак \textit{действия в sc-памяти}, вторым – знак заблокированной \textit{структуры}.}
\scnrelfrom{описание примера}{\scnfilescg{figures/sd_agents/lock.png}}

\scnheader{тип блокировки}
\scnexplanation{Множество \textbf{\textit{тип блокировки}} содержит все возможные классы блокировок, т.е. \textit{структуры}, содержащие \textit{sc-элементы}, заблокированные каким-либо \textit{sc-агентом} на время выполнения им некоторого \textit{действия в sc-памяти}.}
\scnhaselement{полная блокировка}
\scnhaselement{блокировка на любое изменение}
\scnhaselement{блокировка на удаление}

\scnheader{полная блокировка}
\scnexplanation{Каждая \textit{структура}, принадлежащая множеству \textbf{\textit{полная блокировка}} содержит \textit{sc-элементы}, просмотр и изменение (удаление, добавление инцидентных \textit{sc-коннекторов}, удаление самих \textit{sc-элементов}, изменение содержимого в случае файла) которых запрещены всем \textit{sc-агентам}, кроме собственно \textit{sc-агента}, выполняющего соответствующее данной структуре \textit{действие в sc-памяти}, связанное с ней отношением \textit{блокировка*}.

Для того, чтобы исключить возможность реализации \textit{sc-агентов}, которые могут внести изменения в конструкции, описывающие блокировки других \textit{sc-агентов}, все элементы этих конструкций, в том числе, сам знак \textit{структуры}, содержащей заблокированные \textit{sc-элементы} (принадлежащей как множеству \textbf{\textit{полная блокировка}}, так и любому другому \textit{типу блокировки}) и связки отношения \textit{блокировка*}, связывающие эту \textit{структуру} и конкретное \textit{действие в sc-памяти}, добавляются в \textbf{\textit{полную блокировку}}, соответствующую данному \textit{действию в sc-памяти}. Таким образом, каждой \textbf{\textit{полной блокировке}} соответствует петля принадлежности, связывающая ее знак с самим собой.}

\scnheader{блокировка на любое изменение}
\scnexplanation{Каждая \textit{структура}, принадлежащая множеству \textbf{\textit{блокировка на любое изменение}} содержит \textit{sc-элементы}, изменение (физическое удаление, добавление инцидентных \textit{sc-коннекторов}, физическое удаление самих \textit{\mbox{sc-элементов}}, изменение содержимого в случае файла) которых запрещено всем \textit{sc-агентам}, кроме собственно \textit{sc-агента}, выполняющего соответствующее данной структуре \textit{действие в sc-памяти}, связанное с ней отношением \textit{блокировка*}. Однако не запрещен просмотр (чтение) этих \textit{sc-элементов} любым \textit{sc-агентом}.}

\scnheader{блокировка на удаление}
\scnexplanation{Каждая \textit{структура}, принадлежащая множеству \textbf{\textit{блокировка на удаление}} содержит \textit{sc-элементы}, удаление которых запрещено всем \textit{sc-агентам}, кроме собственно \textit{sc-агента}, выполняющего соответствующее данной структуре \textit{действие в sc-памяти}, связанное с ней отношением \textit{блокировка*}. Однако не запрещен просмотр (чтение) этих \textit{sc-элементов} любым \textit{sc-агентом}, добавление инцидентных sc-коннекторов.}

\scnheader{блокировка*}
\scnrelfromset{принципы работы}{
	\scnfileitem{в каждый момент времени одному процессу в sc-памяти может соответствовать только одна блокировка каждого типа};
	\scnfileitem{в каждый момент времени одному процессу в sc-памяти может соответствовать только одна блокировка, установленная на некоторый конкретный sc-элемент};
	\scnfileitem{при завершении выполнения любого процесса в sc-памяти все установленные им блокировки автоматически снимаются};
	\scnfileitem{для повышения эффективности работы системы в целом каждый процесс должен в каждый момент времени блокировать минимально необходимое множество sc-элементов, снимая блокировку с каждого sc-элемента сразу же, как это становится возможным (безопасным)};
	\scnfileitem{В случае когда в рамках \textit{процесса в sc-памяти} явно выделяются более частные подпроцессы (при помощи отношений \textit{темпоральная часть*, поддействие*, декомпозиция действия*} и т. д.), то каждый такой подпроцесс с точки зрения синхронизации выполнения рассматривается как самостоятельный процесс, которому в соответствие могут быть поставлены все необходимые блокировки.}
	\scnaddlevel{1}
	\scnrelfromlist{детализация}{
	\scnfileitem{все дочерние процессы в sc-памяти имеют доступ к блокировкам родительского процесса так же, как если бы это были блокировки соответствующие каждому из таких дочерних процессов};
	\scnfileitem{в свою очередь, родительский процесс не имеет какого-либо привилегированного доступа к sc-элементам, заблокированным дочерними процессами, и работает с ними так же, как любой другой процесс в sc-памяти. Исключение составляют sc-элементы, обозначающие сами дочерние процессы, поскольку родительский процесс должен иметь возможность управления дочерним, например, приостановки или прекращения их выполнения};
	\scnfileitem{все дочерние процессы по отношению друг к другу работают так же, как и по отношению к любым другим процессам};
	\scnfileitem{в случае, когда родительский процесс приостанавливает выполнение (становится \textit{отложенным действием}), \uline{все} его дочерние процессы также приостанавливают выполнение. В свою очередь, приостановка одного из дочерних процессов в общем случае не инициирует явно остановку всего родительского процесса и соответственно других дочерних.} 	
}
	\scnaddlevel{-1}
}

\scnheader{полная блокировка}
\scnrelfromset{принципы работы}{
	\scnfileitem{если sc-элемент, инцидентный некоторому sc-коннектору, попадает в какую-либо полную блокировку, то сам этот sc-коннектор по умолчанию также считается заблокированным этой же блокировкой. Обратное в общем случае неверно, т. к. часть sc-коннекторов, инцидентных некоторому sc-элементу, может быть полностью заблокирована, при этом сам этот элемент заблокирован не будет. Такая ситуация типична, например, для sc-узлов, обозначающих классы понятий};
	\scnfileitem{каждый процесс в sc-памяти может свободно изменять или удалять любые sc-элементы, попадающие в полную блокировку, соответствующую этому процессу.}
}
\scnaddlevel{1}
	\scnnote{Принципы работы с \textit{полными блокировками}, с одной стороны, наиболее просты, поскольку все процессы, кроме установившего такую блокировку, не имеют доступа к заблокированным \mbox{sc-элементам} и конфликты возникнуть не могут. С другой стороны, частое использование блокировок такого типа может привести к тому, что система не сможет использовать в полной мере имеющиеся у нее знания и давать неполные или даже некорректные ответы на поставленные вопросы.}
\scnaddlevel{-1}


\scnheader{блокировка на любое изменение}
\scnrelfromset{принципы работы}{
	\scnfileitem{на один и тот же sc-элемент в один момент времени может быть установлена только одна блокировка одного типа, но разные процессы могут одновременно установить на один и тот же элемент блокировки двух разных типов. Это касается случая, когда первый процесс установил на некоторый sc-элемент блокировку на удаление, а второй процесс затем устанавливает блокировку на любое изменение. В других случаях возникает конфликт блокировок};
	\scnfileitem{установка блокировки любого типа также считается изменением, таким образом, если на некоторый \mbox{sc-элемент} была установлена блокировка на любое изменение, то другой процесс не сможет установить на этот же sc-элемент блокировку любого типа, пока первый процесс не снимет свою};
	\scnfileitem{если блокировка на удаление устанавливается на некоторый sc-коннектор, то по умолчанию та же блокировка устанавливается на инцидентные этому sc-коннектору sc-элементы, поскольку удаление этих элементов приведет к удалению этого коннектора.}
}
\scnaddlevel{1}
	\scnrelto{принципы работы}{блокировка на удаление}
\scnaddlevel{-1}

\scnheader{процесс в sc-памяти}
\scnidtf{действие в sc-памяти}
\scnrelfrom{разбиение}{Классификация процессов в sc-памяти с точки зрения синхронизации их выполнения}
\scnaddlevel{1}
	\scneqtoset{действие поиска sc-элементов;действие генерации sc-элементов;действие удаления sc-элементов;действие установки блокировки некоторого типа на некоторый sc-элемент;действие снятия блокировки с некоторого sc-элемента}
\scnaddlevel{-1}

\scnheader{транзакция в sc-памяти}
\scnexplanation{В некоторых случаях для того, чтобы обеспечить синхронизацию, необходимо объединять несколько элементарных действий над sc-памятью в одно неделимое действие (\textit{транзакцию в sc-памяти}), для которого гарантируется, что ни один сторонний процесс не сможет прочитать или изменить участвующие в этом действии sc-элементы, пока действие не завершится. При этом, в отличие от ситуации с полной блокировкой, процесс, пытающийся получить доступ к таким элементам, не продолжает выполнение так, как если бы этих элементов просто не было в sc-памяти, а ожидает завершения транзакции, после чего может выполнять с данными элементами любые действия согласно общим принципам синхронизации процессов. Проблема обеспечения транзакций не может быть решена на уровне SC-кода и требует реализации таких неделимых действий на уровне \textit{платформы интерпретации sc-моделей}.}

\scnheader{действие поиска sc-элементов}
\scnexplanation{В случае осуществления поиска все найденные и сохраненные в рамках какого-либо процесса sc-элементы попадают в соответствующую данному процессу \textit{блокировку на любое изменение}. Таким образом, гарантируется целостность фрагмента базы знаний, с которым работает некоторый процесс в sc-памяти. При этом поиск и автоматическая установка такой блокировки должны быть реализованы как \textit{транзакция в sc-памяти}.

Такой подход также позволяет избежать ситуации, когда один процесс заблокировал некоторый sc-элемент на любое изменение, а второй процесс пытается сгенерировать или удалить \textit{sc-коннектор}, инцидентный данному \textit{sc-элементу}. В таком случае второй процесс должен будет предварительно найти и заблокировать указанный \textit{sc-элемент} на любое изменение, что вызовет конфликт блокировок (\textit{взаимоблокировку*}).}

\scnheader{действие генерации sc-элементов}
\scnexplanation{В случае генерации любого sc-элемента в рамках некоторого процесса он автоматически попадает в полную блокировку, соответствующую данному процессу. При этом генерация и автоматическая установка такой блокировки должны быть реализованы как \textit{транзакция в sc-памяти}. При необходимости сгенерированные элементы могут быть удалены (т. е. их временное существование вообще никак не отразится на деятельности других процессов) или разблокированы в случае, когда сгенерирована информация, которая может иметь некоторую ценность в дальнейшем.}

\scnheader{действие установки блокировки некоторого типа на некоторый sc-элемент}
\scnexplanation{В случае если какой-либо процесс пытается установить блокировку любого типа на какой-либо sc-элемент, уже заблокированный каким-либо другим процессом, то, с одной стороны, блокировка не может быть установлена, пока другой процесс не разблокирует указанный sc-элемент; с другой стороны, для того чтобы обеспечить возможность поиска и устранения \textit{взаимоблокировок}, необходимо явно указывать тот факт, что какой-либо процесс хочет получить доступ к какому-либо заблокированному другим процессом sc-элементу. Для того чтобы иметь возможность указать, какие процессы пытаются заблокировать уже заблокированный \textit{sc-элемент}, предлагается наряду с отношением \textit{блокировка*} использовать отношение \textit{планируемая блокировка*}, полностью аналогичное отношению \textit{блокировка*}.

Описанный механизм регулирует также и процессы поиска, поскольку поиск и сохранение некоторого sc-элемента предполагает установку \textit{блокировки на любое изменение}. Кроме того, следует учитывать, что на один sc-элемент \textit{блокировка на любое изменение} может быть установлена после \textit{блокировки на удаление}, соответствующей другому процессу. В этом случае использовать отношение \textit{планируемые блокировки*} нет необходимости.}
\scnnote{Действие проверки наличия на некотором sc-элементе блокировки и в зависимости от результата проверки, установки блокировки или планируемой блокировки (с указанием приоритета при необходимости) должно быть реализовано как транзакция.}

\scnheader{планируемая блокировка*}
\scnsubset{блокировка*}
\scnexplanation{Процесс, которому в соответствие поставлена \textit{планируемая блокировка*}, приостанавливает выполнение до тех пор, пока уже установленные блокировки не будут сняты, после чего \textit{планируемая блокировка*} становится реальной \textit{блокировкой*} и процесс продолжает выполнение в соответствии с общими правилами.}

\scnheader{приоритет блокировки*}
\scnrelfrom{область определения}{планируемая блокировка*}
\scnexplanation{В случае, когда на один и тот же sc-элемент планируют установить блокировку сразу несколько процессов, используется отношение \textit{приоритет блокировки*}, связывающее между собой пары отношения \textit{планируемая блокировка*}. Как правило, приоритет блокировки определяется тем, какой из процессов раньше попытался установить блокировку на рассматриваемый sc-элемент, хотя в общем случае приоритет может устанавливаться или меняться в зависимости от дополнительных критериев.}

\scnheader{действие удаления sc-элементов}
\scnnote{В случае попытки удаления некоторого sc-элемента некоторым процессом удаление может быть осуществлено только в случае, когда на данный sc-элемент не установлена (и не планируется) ни одна блокировка каким-либо другим процессом.
	
В других случаях необходимо обеспечить корректное завершение выполнения всех процессов, работающих с данным sc-элементом, и только потом удалить его физически.
	
Для реализации такой возможности каждому процессу в соответствие может быть поставлено множество удаляемых данным процессом sc-элементов.}
\scnnote{Действие проверки наличия блокировок или планируемых блокировок на удаляемый sc-элемент и собственно его удаление или добавление во множество удаляемых sc-элементов для соответствующего процесса должно быть реализовано как транзакция.}

\scnheader{удаляемые sc-элементы*}
\scnrelfrom{первый домен}{процесс в sc-памяти}
\scnexplanation{Sc-элементы, попавшие во множество удаляемых sc-элементов некоторого процесса в sc-памяти, доступны процессам, уже установившим (или планирующим установить) на эти sc-элементы блокировки ранее (до попытки его удаления), а для всех остальных процессов эти sc-элементы уже считаются удаленными. Процесс, пытающийся удалить sc-элемент, приостанавливает свое выполнение до того момента, пока все заблокировавшие и планирующие заблокировать данный sc-элемент процессы не разблокируют его. В общем случае один sc-элемент может входить во множества удаляемых элементов одновременно для нескольких процессов, в этом случае все такие процессы одновременно продолжат выполнение после снятия с этого sc-элемента всех блокировок. Если удаление пытается осуществить один из процессов, уже установивший на указанный sc-элемент блокировку, то алгоритм действий остается прежним -- sc-элемент добавляется во множество удаляемых данным процессом sc-элементов, и будет физически удален, как только все остальные процессы, установившие на данный sc-элемент блокировки, снимут их.}

\scnheader{действие снятия блокировки с некоторого sc-элемента}
\scnrelfromvector{алгоритм выполнения}{\scnfileitem{если на данный sc-элемент установлена одна или несколько \textit{планируемых блокировок*}, то первая из них по приоритету (или единственная) становится \textit{блокировкой*}, соответствующий ей процесс продолжает выполнение (становится настоящей сущностью); связка отношения приоритет выполнения, соответствовавшая удаленной связке отношения \textit{планируемая блокировка*} также удаляется, т. е. приоритет смещается на одну позицию};
\scnfileitem{если \textit{планируемых блокировок*}, установленных на данный sc-элемент, нет, но он попадает во множество удаляемых sc-элементов для одного или нескольких процессов, то рассматриваемый sc-элемент физически удаляется, а приостановленные до его удаления процессы продолжают свое выполнение (становится настоящими сущностями)};
\scnfileitem{если на данный sc-элемент не установлены планируемые блокировки и он не входит во множество удаляемых для какого-либо процесса, то блокировка просто снимается без каких-либо дополнительных изменений.}}

\scnheader{транзакция в sc-памяти}
\scnsubdividing{поиск некоторой конструкции в sc-памяти и автоматическая установка блокировки на любое изменение на найденные sc-элементы;генерация некоторого sc-элемента и автоматическая установка на него полной блокировки;проверка наличия на некотором sc-элементе блокировки и в зависимости от результата проверки установка блокировки или планируемой блокировки;проверка наличия блокировок или планируемых блокировок на удаляемый sc-элемент и собственно его удаление или добавление во множество удаляемых sc-элементов для соответствующего процесса;снятие блокировки с заданного sc-элемента и при необходимости установка первой по приоритету планируемой блокировки или удаление данного sc-элемента, если он входит во множество удаляемых sc-элементов для некоторого процесса;поиск подпроцессов процесса и добавление их во множество отложенных действий в случае добавления самого процесса в данное множество;поиск подпроцессов процесса и удаление их из множества отложенных действий в случае удаления самого процесса из данного множества}

\scnheader{абстрактный программный sc-агент}
\scnnote{При реализации \textit{абстрактных программных sc-агентов} на \textit{языке SCP}, соблюдение всех принципов синхронизации соответствующих этим sc-агентам процессов обеспечивается на уровне \textit{sc-агентов интерпретации scp-программ}, т. е. средствами \textit{платформы интерпретации sc-моделей}. При реализации \textit{абстрактных программных sc-агентов} на уровне платформы, соблюдение всех принципов синхронизации возлагается, во-первых, непосредственно на разработчика агентов, во-вторых, -- на разработчика платформы. Так, например, платформа может предоставлять доступ к хранимым в sc-памяти элементам через некоторый программный интерфейс, уже учитывающий принципы работы с блокировками, что избавит разработчика агентов от необходимости учитывать все эти принципы вручную.}
\scnrelfromset{принципы работы}{\scnfileitem{в результате появления в sc-памяти некоторой конструкции, удовлетворяющей условию инициирования какого-либо \textit{абстрактного sc-агента}, реализованного при помощи \textit{Языка SCP}, в \textit{sc-памяти} генерируется и инициируется \textit{scp-процесс}. В качестве шаблона для генерации используется \textit{агентная scp-программа}, соответствующая данному \textit{абстрактному sc-агенту}.};
\scnfileitem{каждый такой \textit{scp-процесс}, соответствующий некоторой \textit{агентной \mbox{scp-программе}}, может быть связан с набором структур, описывающих блокировки различных типов. Таким образом, синхронизация взаимодействия параллельно выполняемых \textit{scp-процесcов} осуществляется так же, как и в случае любых других \textit{действий в sc-памяти}.};
\scnfileitem{несмотря на то что каждый \textit{scp-оператор} представляет собой атомарное действие в sc-памяти, являющееся поддействием в рамках всего \textit{\mbox{scp-процесса}}, блокировки, соответствующие одному оператору, не вводятся, чтобы избежать громоздкости и избытка дополнительных системных конструкций, создаваемых при выполнении некоторого \textit{scp-процесса}. Вместо этого используются блокировки, общие для всего \textit{scp-процесса}. Таким образом, \textit{агенты интерпретации scp-программ} работают только с учетом блокировок, общих для всего интерпретируемого \textit{scp-процесса}.};
\scnfileitem{процессы, описывающие деятельность агентов интерпретации \textit{scp-программ}, как правило, не создаются, следовательно, и не вводятся соответствующие им блокировки. Поскольку такие агенты работают с уникальным scp-процессом и их число ограничено и известно, то использование блокировок для их синхронизации не требуется.};
\scnfileitem{в случае приостановки \textit{scp-процесса} (добавления его во множество \textit{отложенных действий}) в соответствии с общими правилами синхронизации все его дочерние процессы также должны быть	приостановлены. В связи с этим все \textit{scp-операторы}, которые в	этот момент являются \textit{настоящими сущностями}, становятся	\textit{отложенными действиями}.};
\scnfileitem{во избежание нежелательных изменений в самом теле \textit{scp-процесса}, вся конструкция, сгенерированная на основе некоторой \textit{scp-программы} (весь \textit{sc-текст}, описывающий декомпозицию \textit{scp-процесса} на \textit{scp-операторы}), должна быть добавлена в \textit{полную блокировку}, соответствующую данному \textit{scp-процессу}.};
\scnfileitem{при необходимости разблокировать или заблокировать некоторую конструкцию каким-либо типом блокировки используются соответствующие \textit{scp-операторы} класса \textit{scp-оператор управления блокировками}.};
\scnfileitem{после завершения выполнения некоторого scp-процесса его текст, как правило, удаляется из \textit{sc-памяти}, а все заблокированные конструкции освобождаются (разрушаются знаки структур, обозначавших блокировки).};
\scnfileitem{как правило, частный \textit{класс действий}, соответствующий конкретной \textit{scp-программе}, явно не вводится, а используется более общий класс \textit{scp-процесс}, за исключением тех случаев, когда введение	специального \textit{класса действий} необходимо по каким-либо другим соображениям.}}
	
\scnheader{блокировка*}
\scnnote{В общем случае весь механизм блокировок может описываться как на уровне SC-кода (для повышения уровня платформенной независимости), так и при необходимости может быть реализован на уровне \textit{платформы интерпретации sc-моделей}, например для повышения производительности. Для этого каждому выполняемому в sc-памяти процессу на нижнем уровне может быть поставлена в соответствие некая уникальная таблица, в каждый момент времени содержащая перечень заблокированных элементов с указанием типа блокировки.}

\scnrelfromvector{пример применения}{\scgfileitem{figures/sd_agents/plan_lock_1.png}\\
\scnaddlevel{1}
\scnexplanation{В данном примере \textit{Процесс1} непосредственно работает с sc-элементом \textit{\textbf{e1}},\textit{Процесс2} и \textit{Процесс3} планируют установить блокировку на любое изменение и блокировку на удаление соответственно, причем \textit{Процесс2} попытался установить свою блокировку раньше, чем \textit{Процесс3}, поэтому согласно направлению связки отношения \textit{приоритет блокировки*}, его блокировка будет установлена раньше. \textit{Процесс4} и \textit{Процесс5} ожидают снятия всех блокировок и планируемых блокировок, после чего \textit{\textbf{e1}} будет удален и \textit{Процесс1} и \textit{Процесс2} продолжат свое выполнение. Никакие другие планируемые блокировки установлены быть уже не могут, поскольку \textit{\textbf{e1}} попал во множество удаляемых sc-элементов как минимум одного процесса и, в соответствии с изложеннымивыше правилами, все остальные процессы кроме \textit{Процесс1}-\textit{Процесс5}, уже несмогут получить доступ к этому sc-элементу.		
Выполняемый процесс принадлежит множеству настоящая сущность, приостановленные – множеству отложенное действие.
}
\scnaddlevel{-1}
;\scgfileitem{figures/sd_agents/plan_lock_2.png}\\
\scnaddlevel{1}
\scnexplanation{После того как \textit{Процесс1} разблокировал sc-элемент \textit{\textbf{e1}}, этот элемент будет заблокирован \textit{Процессом2}, и \textit{Процесс2} продолжит выполнение. \textit{Планируемая блокировка*}, установленная \textit{Процессом2}, становится обычной \textit{блокировкой*}.}
\scnaddlevel{-1}
;\scgfileitem{figures/sd_agents/plan_lock_3.png}\\
\scnaddlevel{1}
\scnexplanation{После того как \textit{Процесс2} разблокировал sc-элемент \textit{\textbf{e1}}, этот элемент будет заблокирован \textit{Процессом3}, и \textit{Процесс3} продолжит выполнение.}
\scnaddlevel{-1}
;\scgfileitem{figures/sd_agents/plan_lock_4.png}\\
\scnaddlevel{1}
\scnexplanation{Когда все процессы снимут блокировки с sc-элемента \textit{\textbf{e1}}, он может быть физически удален и \textit{Процесс4} и \textit{Процесс5} продолжат выполнение.}
\scnaddlevel{-1}
}

\scnheader{взаимоблокировка*}
\scnexplanation{В зависимости от конкретных \textit{типов блокировок} установленных паралельно выполняемыми процессами на некоторые sc-элементы и того, какие конкретно действия с этими \textit{sc-элементами} предполагается выполнить далее в рамках выполнения этих процессов, возможны ситуации взаимоблокировки, когда каждый из указанных процессов будет ожидать снятия блокировки вторым процессом с нужного \textit{sc-элемента}, не снимая при этом установленной им самим блокировки с \textit{sc-элемента}, доступ к которому необходим второму процессу.
	
В случае когда хотя бы одна из блокировок является \textit{полной блокировкой}, ситуация взаимоблокировки возникнуть не может, поскольку \textit{sc-элементы}, попавшие в \textit{полную блокировку} некоторого \textit{scp-процесса}, не доступны другим \textit{scp-процессам} даже для чтения и, таким образом, остальные \textit{scp-процессы} будут работать так, как будто заблокированные \textit{sc-элементы} просто отсутствуют в текущем состоянии \textit{sc-памяти}.
	
В случаях, когда ни одна из установленных блокировок не является \textit{полной блокировкой}, возможно появление взаимоблокировок.}
\scnnote{Устранение \textit{взаимоблокировки} невозможно без вмешательства специализированного \textit{sc-метаагента}, который имеет право игнорировать блокировки, установленные другими процессами. 

В общем случае проблема конкретной взаимоблокировки может быть решена путем выполнения специализированным \textit{sc-метаагентом} следующих шагов:	
\begin{scnitemize}
	\item откат нескольких операций, выполненных одним из участвующих в взаимоблокировке процессов настолько шагов назад, насколько это необходимо для того, чтобы второй процесс получил доступ к необходимым \textit{sc-элементам} и смог продолжить выполнение;
	\item ожидание выполнения второго процесса вплоть до завершения или до снятия им всех блокировок с \textit{sc-элементов}, доступ к которым необходимо получить первому процессу;
	\item повторное выполнение в рамках первого процесса отмененных операций и продолжение его выполнения, но уже с учетом изменений в памяти, внесенных вторым процессом.		
\end{scnitemize}
}

\scnheader{sc-метаагент}
\scnexplanation{Для \textit{sc-метаагентов} все sc-элементы, в том числе описывающие блокировки, планируемые блокировки и т. д. полностью эквивалентны между собой с точки зрения доступа к ним, т. е. любой \textit{sc-метаагент} имеет доступ к любым sc-элементам, даже попавшим в полную блокировку для какого-либо другого процесса. Это необходимо для того, чтобы \textit{sc-метаагенты} смогли выявлять и устранять различные проблемы, например, описанную выше проблему взаимоблокировки.
	
Таким образом, проблема синхронизации деятельности \textit{sc-метаагентов} требует введения дополнительных правил.
	
Указанную проблему разделим на две более частные:
\begin{scnitemize}
	\item обеспечение синхронизации деятельности \textit{sc-метаагентов} между собой;
	\item обеспечение синхронизации деятельности \textit{sc-метаагентов} и \textit{программных sc-агентов}.		
\end{scnitemize}
	
Первую проблему предлагается решить за счет запрета параллельного выполнения \textit{sc-метаагентов}. Таким образом, в каждый момент времени в рамках одной \textit{ostis-системы} может существовать только один процесс, соответствующий \textit{sc-метаагенту} и являющийся \textit{настоящей сущностью}. 
	
Вторую проблему предлагается решить за счет введения дополнительных привилегий для \textit{sc-метаагентов} при обращении к какому-либо sc-элементу. Для этого достаточно одного правила: 

Если некоторый sc-элемент стал использоваться в рамках процесса, соответствующего \textit{sc-метаагенту} (например, стал элементом хотя бы одного scp-оператора, входящего в данный процесс), то все процессы, в блокировки соответствующие которым попадает указанный sc-элемент, становятся отложенными действиями (приостанавливают выполнение). Как только указанный sc-элемент перестает использоваться в рамках процесса, соответствующего \textit{sc-метаагенту}, все приостановленные по этой причине процессы продолжают выполнение.
	
Рассмотренные ограничения не ухудшают производительность ostis-системы существенно, поскольку \textit{sc-метаагенты} предназначены для решения достаточно узкого класса задач, которые, как показал опыт практической разработки прототипов различных \textit{ostis-систем}, возникают достаточно редко.}
\scnnote{Стоит отметить, что возможна ситуация, при которой выполнение некоторого процесса в sc-памяти прервано по причине возникновения какой-либо ошибки. В таком случае существует вероятность того, что блокировка, установленная данным процессом не будет снята до тех пор, пока этого не сделает sc-метаагент, обнаруживший подобную ситуацию. Однако указанная проблема на уровне sc-модели может быть решена лишь частично, для случаев, когда ошибка возникает при интерпретации scp-программы, отслеживается scp-интепретатором и в памяти формируется соответствующая конструкция, сообщающая о проблеме sc-метаагенту. Случаи, когда возникла ошибка на уровне scp-интерпретатора или sc-хранилища, должны рассматриваться на уровне платформы интерпретации sc-моделей.}

\bigskip
\scnendstruct \scnendsegmentcomment{Принципы синхронизации деятельности sc-агентов}

\bigskip
\scnendstruct \scnendcurrentsectioncomment

\end{SCn}


\scsubsection[\scneditor{Шункевич Д.В.}\protect\scnmonographychapter{Глава 3.2. Ситуационное управление обработкой знаний в интеллектуальных компьютерных системах нового поколения}]{Предметная область и онтология Базового языка программирования ostis-систем}
\label{sd_scp}
\input{Contents/chapter2/sd_ostis_sys_models/sd_ps/sd_scp}

\scsubsubsection[\scnmonographychapter{Глава 3.2. Ситуационное управление обработкой знаний в интеллектуальных компьютерных системах нового поколения}]{Предметная область и онтология синтаксиса Базового языка программирования ostis-систем}
\label{sd_scp_syntax}

\scsubsubsection[\scnmonographychapter{Глава 3.2. Ситуационное управление обработкой знаний в интеллектуальных компьютерных системах нового поколения}]{Предметная область и онтология денотационной семантики Базового языка программирования ostis-систем}
\label{sd_scp_denote_sem}
\input{Contents/chapter2/sd_ostis_sys_models/sd_ps/sd_scp_denote_sem}

\scsubsubsection[\scnmonographychapter{Глава 3.2. Ситуационное управление обработкой знаний в интеллектуальных компьютерных системах нового поколения}]{Предметная область и онтология операционной семантики Базового языка программирования ostis-систем}
\label{sd_scp_oper_sem}
\input{Contents/chapter2/sd_ostis_sys_models/sd_ps/sd_scp_oper_sem}

\scsubsection[\scneditor{Сердюков Р.Е.}\protect\scnmonographychapter{Глава 3.3. Семантическая теория программ в интеллектуальных компьютерных системах нового поколения}]{Предметная область и онтология программ и языков программирования для ostis-систем}
\label{sd_programs}

\scsubsubsection[\scneditor{Сердюков Р.Е.}\protect\scnmonographychapter{Глава 3.3. Семантическая теория программ в интеллектуальных компьютерных системах нового поколения}]{Предметная область и онтология интерпретации современных языков программирования в ostis-системах}
\label{sd_program_interpreting}

\scsubsection[\scneditors{Бутрин С.В.;Шункевич Д.В.;Зотов Н.В.;Орлов М.К.}\protect\scnmonographychapter{Глава 3.4. Язык запросов для интеллектуальных компьютерных систем нового поколения}]{Предметная область и онтология sc-языка вопросов}
\label{sd_sc_quest_lang}

\scsubsubsection[\scneditor{Бутрин С.В.}\protect\scnmonographychapter{Глава 3.4. Язык запросов для интеллектуальных компьютерных систем нового поколения}]{Предметная область и онтология синтаксиса sc-языка вопросов}
\label{sd_syntax_sc_quest_lang}

\scsubsubsection[\scneditor{Бутрин С.В.}\protect\scnmonographychapter{Глава 3.4. Язык запросов для интеллектуальных компьютерных систем нового поколения}]{Предметная область и онтология денотационной семантики sc-языка вопросов}
\label{sd_denot_sem_sc_quest_lang}

\scsubsubsection[\scneditor{Бутрин С.В.}\protect\scnmonographychapter{Глава 3.4. Язык запросов для интеллектуальных компьютерных систем нового поколения}]{Предметная область и онтология операционной семантики sc-языка вопросов}
\label{sd_operat_sem_sc_quest_lang}

\scsubsection[\scneditors{Василевская А.П.;Зотов Н.В.;Орлов М.К.}\protect\scnmonographychapter{Глава 3.5. Логические, продукционные и функциональные модели решения задач в интеллектуальных компьютерных системах нового поколения}]{Предметная область и онтология операционной семантики логических sc-языков}
\label{sd_operat_sem_sc_logical_lang}

\scsubsection[\scneditors{Зотов Н.В.;Орлов М.К.}\protect\scnmonographychapter{Глава 3.5. Логические, продукционные и функциональные модели решения задач в интеллектуальных компьютерных системах нового поколения}]{Предметная область и онтология sc-языков продукционного программирования}
\label{sd_sc_product_program_lang}

\scsubsubsection[\scneditors{Зотов Н.В.;Орлов М.К.}\protect\scnmonographychapter{Глава 3.5. Логические, продукционные и функциональные модели решения задач в интеллектуальных компьютерных системах нового поколения}]{Предметная область и онтология синтаксиса sc-языков продукционного программирования}
\label{sd_sc_product_program_lang_syntax}

\scsubsubsection[\scneditors{Зотов Н.В.;Орлов М.К.}\protect\scnmonographychapter{Глава 3.5. Логические, продукционные и функциональные модели решения задач в интеллектуальных компьютерных системах нового поколения}]{Предметная область и онтология денотационной семантики sc-языков продукционного программирования}
\label{sd_sc_product_program_lang_denot_sem}

\scsubsubsection[\scneditors{Зотов Н.В.;Орлов М.К.}\protect\scnmonographychapter{Глава 3.5. Логические, продукционные и функциональные модели решения задач в интеллектуальных компьютерных системах нового поколения}]{Предметная область и онтология операционной семантики sc-языков продукционного программирования}
\label{sd_sc_product_program_lang_oper_sem}

\scsubsection[\scneditors{Ковалев М.В.;Крощенко А.А.;Михно Е.В.}\protect\scnmonographychapter{Глава 3.6. Конвергенция и интеграция искусственных нейронных сетей с базами знаний в интеллектуальных компьютерных системах нового поколения}]{Предметная область и онтология sc-моделей искусственных нейронных сетей}
\label{sd_ann}
\input{Contents/chapter2/sd_ostis_sys_models/sd_ps/sd_ann}

\scsubsubsection[\scneditor{Ковалев М.В.}\protect\scnmonographychapter{Глава 3.6. Конвергенция и интеграция искусственных нейронных сетей с базами знаний в интеллектуальных компьютерных системах нового поколения}]{Предметная область и онтология синтаксиса sc-моделей искусственных нейронных сетей}
\label{sd_syntax_sc_model_ann}

\scsubsubsection[\scneditor{Ковалев М.В.}\protect\scnmonographychapter{Глава 3.6. Конвергенция и интеграция искусственных нейронных сетей с базами знаний в интеллектуальных компьютерных системах нового поколения}]{Предметная область и онтология денотационной семантики sc-моделей искусственных нейронных сетей}
\label{sd_denot_sem_sc_model_ann}

\scsubsubsection[\scneditor{Ковалев М.В.}\protect\scnmonographychapter{Глава 3.6. Конвергенция и интеграция искусственных нейронных сетей с базами знаний в интеллектуальных компьютерных системах нового поколения}]{Предметная область и онтология операционной семантики sc-моделей искусственных нейронных сетей}
\label{sd_oper_sem_sc_model_ann}

\scsectionfamily{Часть 4 Стандарта OSTIS. Онтологические модели интерфейсов интеллектуальных компьютерных систем нового поколения}
\label{part_interfaces}

\scsection[\scneditor{Садовский М.Е.}\protect\scnmonographychapter{Глава 4.1. Структура интерфейсов интеллектуальных компьютерных систем нового поколения}]{Предметная область и онтология интерфейсов ostis-систем}
\label{sd_interfaces}
\input{Contents/chapter2/sd_ostis_tools/sd_ui/sd_user_interfaces.tex}

\scsubsection[\scneditors{Садовский М.Е.;Жмырко А.В.}\protect\scnmonographychapter{Глава 4.1. Структура интерфейсов интеллектуальных компьютерных систем нового поколения}]{Предметная область и онтология интерфейсных действий пользователей ostis-системы}
\label{sd_user_interface_actions}
\input{Contents/chapter2/sd_ostis_tools/sd_ui/sd_user_interface_actions.tex}

\scsubsection[\scneditors{Садовский М.Е.;Никифоров С.А.;Захарьев В.А.}\protect\scnmonographychapter{Глава 4.1. Структура интерфейсов интеллектуальных компьютерных систем нового поколения}]{Предметная область и онтология сообщений, входящих в ostis-систему и выходящих из неё}
\label{sd_messages}

\scsubsection[\scneditors{Садовский М.Е.;Жмырко А.В.}\protect\scnmonographychapter{Глава 4.1. Структура интерфейсов интеллектуальных компьютерных систем нового поколения}]{Предметная область и онтология действий и внутренних агентов пользовательского интерфейса ostis-системы}
\label{sd_actions_and_internal_agent}

\scsubsection[\scneditor{Никифоров С.А.}\protect\scnmonographychapter{Глава 4.2. Естественно-языковые интерфейсы интеллектуальных компьютерных систем нового поколения}]{Предметная область и онтология естественно-языковых интерфейсов ostis-систем}
\label{sd_natural_lang_interface}

\scsubsubsection[\scneditors{Никифоров С.А.;Бобёр Е.С.;Захарьев В.А.}\protect\scnmonographychapter{Глава 4.2. Естественно-языковые интерфейсы интеллектуальных компьютерных систем нового поколения}]{Предметная область и онтология синтаксического анализа естественно-языковых сообщений, входящих в ostis-систему}
\label{sd_process_syntax_message_analysis}

\scsubsubsection[\scnmonographychapter{Глава 4.2. Естественно-языковые интерфейсы интеллектуальных компьютерных систем нового поколения}]{Предметная область и онтология понимания естественно-языковых сообщений, входящих в ostis-систему}
\label{sd_process_message_understanding}

\scsubsubsection[\scneditors{Никифоров С.А.;Бобёр Е.С.;Захарьев В.А.}\protect\scnmonographychapter{Глава 4.2. Естественно-языковые интерфейсы интеллектуальных компьютерных систем нового поколения}]{Предметная область и онтология синтеза естественно-языковых сообщений  ostis-системы}
\label{sd_process_message_synthesis}

\scsectionfamily{Часть 5 Стандарта OSTIS. Методы и средства проектирования интеллектуальных компьютерных систем нового поколения}
\label{part_methods_and_tools}

\scsection[\scneditor{Шункевич Д.В.}\protect\scnmonographychapter{Глава 5.1. Комплексная библиотека многократно используемых семантически совместимых компонентов интеллектуальных компьютерных систем нового поколения}]{Предметная область и онтология комплексной библиотеки многократно используемых семантически совместимых компонентов ostis-систем}
\label{sd_biblio_component}

\scsubsection[\scneditor{Банцевич К.А.}\protect\scnmonographychapter{Глава 5.1. Комплексная библиотека многократно используемых семантически совместимых компонентов интеллектуальных компьютерных систем нового поколения}]{Предметная область и онтология многократно используемых компонентов баз знаний ostis-систем}
\label{sd_know_base_component}

\scsubsection[\scneditor{Шункевич Д.В.}\protect\scnmonographychapter{Глава 5.3. Методика и средства компонентного проектирования решателей задач интеллектуальных компьютерных систем нового поколения}]{Предметная область и онтология многократно используемых компонентов решателей задач ostis-систем}
\label{sd_problem_solver_component}

\scsubsubsection[\scneditor{Шункевич Д.В.}\protect\scnmonographychapter{Глава 5.3. Методика и средства компонентного проектирования решателей задач интеллектуальных компьютерных систем нового поколения}]{Предметная область и онтология многократно используемых методов, хранимых в памяти ostis-систем и интерпретируемых их внутренними агентами}
\label{sd_method_agent}

\scsubsubsection[\scneditors{Шункевич Д.В.;Зотов Н.В.;Орлов М.К.}\protect\scnmonographychapter{Глава 5.3. Методика и средства компонентного проектирования решателей задач интеллектуальных компьютерных систем нового поколения}]{Предметная область и онтология многократно используемых внутренних агентов ostis-систем}
\label{sd_internal_agent}

\scsubsection[\scneditor{Садовский М.Е.}\protect\scnmonographychapter{Глава 5.4. Методика и средства компонентного проектирования интерфейсов интеллектуальных компьютерных систем нового поколения}]{Предметная область и онтология многократно используемых компонентов интерфейсов ostis-систем}
\label{sd_component_interface}

\scsubsection[\scneditor{Шункевич Д.В.}\protect\scnmonographychapter{Глава 5.1. Комплексная библиотека многократно используемых семантически совместимых компонентов интеллектуальных компьютерных систем нового поколения}]{Предметная область и онтология многократно используемых встраиваемых ostis-систем}
\label{sd_embed_sys}

\scsection[\scneditor{Шункевич Д.В.}]{Предметная область и онтология действий и методик проектирования ostis-систем}
\label{sd_actions_methodology_design}

\scsubsection[\scneditors{Банцевич К.А.;Бутрин С.В.}\protect\scnmonographychapter{Глава 5.2. Методика и средства проектирования и анализа качества баз знаний интеллектуальных компьютерных систем нового поколения}]{Предметная область и онтология действий и методик проектирования баз знаний ostis-систем}
\label{sd_actions_methodology_know_base_design}

\scsubsection[\scneditor{Шункевич Д.В.}\protect\scnmonographychapter{Глава 5.3. Методика и средства компонентного проектирования решателей задач интеллектуальных компьютерных систем нового поколения}]{Предметная область и онтология действий и методик проектирования решателей задач ostis-систем}
\label{sd_actions_methodology_problem_solver_design}

\scsubsection[\scneditor{Садовский М.Е.}\protect\scnmonographychapter{Глава 5.4. Методика и средства компонентного проектирования интерфейсов интеллектуальных компьютерных систем нового поколения}]{Предметная область и онтология действий и методик проектирования интерфейсов ostis-систем}
\label{sd_actions_methodology_interface_design}

\scsection{Предметная область и онтология действий и методик \uline{обучения} проектированию ostis-систем}
\label{sd_actions_methodology_learning_design}

\scsection[\scneditor{Шункевич Д.В.}]{Предметная область и онтология средств проектирования ostis-систем}
\label{sd_fund_design}

\scsubsection[\scneditor{Бутрин С.В.}\protect\scnmonographychapter{Глава 5.2. Методика и средства проектирования и анализа качества баз знаний интеллектуальных компьютерных систем нового поколения}]{Логико-семантическая модель комплекса встраиваемых ostis-систем автоматизации проектирования баз знаний ostis-систем}
\label{logical_model_embed_automation_design}

\scsubsubsection[\scneditors{Бутрин С.В.}\protect\scnmonographychapter{Глава 5.2. Методика и средства проектирования и анализа качества баз знаний интеллектуальных компьютерных систем нового поколения}]{Логико-семантическая модель ostis-системы редактирования, сборки и ввода исходных текстов различных компонентов проектируемой базы знаний в память ostis-системы}
\label{edit_assem_logical_model}

\scsubsubsection[\scneditor{Бутрин С.В.}\protect\scnmonographychapter{Глава 5.2. Методика и средства проектирования и анализа качества баз знаний интеллектуальных компьютерных систем нового поколения}]{Логико-семантическая модель ostis-системы редактирования проектируемой базы знаний ostis-системы на уровне её внутреннего представления}
\label{edit_tools_proj_logical_model}

\scsubsubsection[\scneditor{Бутрин С.В.}\protect\scnmonographychapter{Глава 5.2. Методика и средства проектирования и анализа качества баз знаний интеллектуальных компьютерных систем нового поколения}]{Логико-семантическая модель ostis-системы обнаружения и анализа ошибок и противоречий в базе знаний ostis-системы}
\label{detec_error_logical_model}

\scsubsubsection[\scneditor{Бутрин С.В.}\protect\scnmonographychapter{Глава 5.2. Методика и средства проектирования и анализа качества баз знаний интеллектуальных компьютерных систем нового поколения}]{Логико-семантическая модель ostis-системы обнаружения и анализа информационных дыр в базе знаний ostis-системы}
\label{detec_hole_logical_model}

\scsubsubsection[\scneditors{Бутрин С.В.;Банцевич К.А.}\protect\scnmonographychapter{Глава 5.2. Методика и средства проектирования и анализа качества баз знаний интеллектуальных компьютерных систем нового поколения}]{Логико-семантическая модель ostis-системы автоматизации управления взаимодействием разработчиков различных категорий в процессе проектирования базы знаний ostis-системы}
\label{author_logical_model}

\scsubsection[\scneditor{Шункевич Д.В.}\protect\scnmonographychapter{Глава 5.3. Методика и средства компонентного проектирования решателей задач интеллектуальных компьютерных систем нового поколения}]{Логико-семантическая модель комплекса ostis-систем автоматизации проектирования решателей задач ostis-систем}
\label{problem_solver_design_tools}

\scsubsubsection[\scneditor{Шункевич Д.В.}\protect\scnmonographychapter{Глава 5.3. Методика и средства компонентного проектирования решателей задач интеллектуальных компьютерных систем нового поколения}]{Логико-семантическая модель ostis-системы автоматизации проектирования программ Базового языка программирования ostis-систем}
\label{programs_design_tools}

\scsubsubsection[\scneditor{Шункевич Д.В.}\protect\scnmonographychapter{Глава 5.3. Методика и средства компонентного проектирования решателей задач интеллектуальных компьютерных систем нового поколения}]{Логико-семантическая модель ostis-системы автоматизации проектирования внутренних агентов ostis-систем, а также коллективов таких агентов}
\label{agents_design_tools}

\scsubsubsection[\scneditors{Ковалев М.В.;Крощенко А.А.;Михно Е.В.}\protect\scnmonographychapter{Глава 3.6. Конвергенция и интеграция искусственных нейронных сетей с базами знаний в интеллектуальных компьютерных системах нового поколения}]{Логико-семантическая модель ostis-системы автоматизации проектирования искусственных нейронных сетей, семантически совместимых с базами знаний ostis-систем}
\label{autom_design_neural_network_logical_model}

\scsubsection[\scneditors{Садовский М.Е.;Жмырко А.В.}\protect\scnmonographychapter{Глава 5.4. Методика и средства компонентного проектирования интерфейсов интеллектуальных компьютерных систем нового поколения}]{Логико-семантическая модель ostis-системы автоматизации проектирования интерфейсов ostis-систем}
\label{autom_design_interface_logical_model}

\scsubsection[\scnmonographychapter{Глава 7.2. Экосистема интеллектуальных компьютерных систем нового поколения (Экосистема OSTIS) и реализация рынка знаний на ее основе}]{Предметная область и онтология ostis-систем автоматизации проектирования различных классов ostis-систем}
\label{sd_autom_design_class}

\scsection{Предметная область и онтология ostis-систем обучения проектированию ostis-систем и их компонентов}
\label{sd_learn_design_component}