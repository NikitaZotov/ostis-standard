\begin{SCn}

\scnsectionheader{\currentname}

\scnstartsubstruct

\scnrelfrom{дочерний раздел}{\nameref{ims_ostis_model}}

\scnheader{Предметная область семантически совместимых интеллектуальных порталов научно-технических знаний}
\scniselement{предметная область}
\scnsdmainclasssingle{портал научных знаний}
%\scnsdclass{***}
%\scnsdrelation{***}

\scnheader{портал научных знаний}
\scnexplanation{Целями интеллектуального \textbf{\textit{портала научных знаний}} являются:

\begin{scnitemize}
    \item Ускорение погружения каждого человека в новые для него научные области при постоянном сохранении общей целостной картины Мира (образовательная цель);
    \item Фиксация в систематизированном виде новых научных результатов так, чтобы все основные связи новых результатов с известными были четко обозначены;
    \item Автоматизация координации работ по рецензированию новых результатов;
    \item Автоматизация анализа текущего состояния базы знаний.
\end{scnitemize}


Создание интеллектуальных \textbf{\textit{порталов научных знаний}}, обеспечивающих повышение темпов интеграции и согласования различных точек зрения, -- это способ существенного повышения темпов эволюции научно-технической деятельности.

Совместимые \textbf{\textit{порталы научных знаний}}, реализованные в виде \textit{ostis-систем}, входящих в \textit{Экосистему OSTIS}, являются основой новых принципов организации научной деятельности, в которой 
\begin{scnitemize}
    \item результатами этой деятельности являются не статьи, монографии, отчеты и другие научно-технические документы,  а фрагменты глобальной базы знаний, разработчиками которых являются свободно формируемые научные коллективы, состоящие из специалистов в соответствующих научных дисциплинах,
    \item с помощью \textbf{\textit{порталов научных знаний}} осуществляется 
    \begin{scnitemizeii}
        \item координация процесса рецензирования новой научно-технической информации, поступающей от научных работников в базы знаний этих порталов,
        \item  процесс согласования различных точек зрения ученых (в частности, введению и семантической корректировке понятий, а также введению и корректировке терминов, соответствующих различным сущностям).
    \end{scnitemizeii}
\end{scnitemize}

Реализация семейства семантически совместимых порталов научных знаний в виде совместимых \textit{\mbox{ostis-систем}}, входящих в состав \textit{Экосистемы OSTIS}, предполагает разработку иерархической системы семантически согласованных формальных онтологий, соответствующих различным научно-техническим дисциплинам, с четко заданным наследованием свойств описываемых сущностей и с четко заданными междисциплинарными связями, которые описываются связями между соответствующими формальными онтологиями и специфицируемыми ими предметными областями.

Реализация \textbf{\textit{порталов научных знаний}} в виде семейства семантически совместимых \textit{ostis-систем} означает также попытку преодолеть \scnqq{вавилонское столпотворение} многообразия научно-технических языков, не меняя сути научно-технических знаний, а сводя эти знания к единой универсальной форме смыслового представления знаний в памяти порталов научных знаний, т.е. к форме которая в достаточной степени понятна как \textit{ostis-системам}, так и любым потенциальным их пользователям.

Примером \textbf{\textit{портала научных знаний}}, построенного в виде \textit{ostis-системы} является \textit{Метасистема IMS.ostis}, содержащая все известные на текущий момент знания и навыки, входящие в состав \textit{Технологии OSTIS}.}

\bigskip
\scnendstruct \scnendcurrentsectioncomment

\end{SCn}