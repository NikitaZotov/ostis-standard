\begin{SCn}

\scsuperchapter

\scnsectionheader{Стандарт OSTIS}
\label{super_char}
\scnstartsubstruct

\scnidtf{Документация Технологии OSTIS}
\scnidtf{Документация открытой технологии онтологического проектирования, производства и эксплуатации семантически совместимых гибридных интеллектуальных компьютерных систем}
\scnidtf{Описание \textit{Технологии OSTIS} (Open Semantic Technology for Intelligent Systems), представленное в виде раздела \textit{базы знаний ostis-системы} на внутреннем языке \textit{ostis-систем} и обладающее достаточной полнотой для использования этой технологии разработчиками \textit{интеллектуальных компьютерных систем}}
\scnidtf{Полное описание текущего состояния \textit{Технологии OSTIS}, представленное в виде раздела \textit{базы знаний}, построенной по \textit{Технологии OSTIS}}
\scnidtf{Основной раздел \textit{базы знаний} \scnbigspace \textit{Метасистемы IMS.ostis}, которая предназначена для комплексной поддержки онтологического проектирования семантически совместимых \textit{гибридных интеллектуальных компьютерных систем}}
\scniselement{раздел базы знаний}
    \scnaddlevel{1}
    \scnidtf{раздел внутреннего представления \textit{базы знаний ostis-системы} -- \textit{интеллектуальной компьютерной системы}, построенной по \textit{Технологии OSTIS}}
    \scnaddlevel{-1}
%\scnrelfromset{основные авторы}{Голенков В.В.;Гулякина Н.А.;Шункевич Д.В.}
%\scnrelfrom{научный редактор}{Голенков В.В.}
\scnrelfromset{рецензенты}{Курбацкий А.Н.;Дудкин А.А.}
\scnrelfrom{финансовая поддержка}{Intelligent Semantic Systems Ltd.}
\scnaddlevel{1}
\scnaddlevel{-1}
\scnreltovector{конкатенация подразделов}{Вводный раздел Документации Технологии OSTIS;Обоснование Технологии OSTIS;Предметная область и онтология Технологии OSTIS;Заключительная часть Документации Технологии OSTIS;Библиографическая часть Документации Технологии OSTIS}

\scnheader{Стандарт OSTIS}
\scntext{эпиграф}{From data science to knowledge science}
\scntext{аннотация}{В настоящее время информатика преодолевает важнейший этап своего развития --- переход от информатики данных (data science) к информатике знаний (knowledge science), где акцентируется внимание на \uline{семантических} аспектах представления и обработки \textit{знаний}.\\
Без фундаментального анализа такого перехода невозможно решить многие проблемы, связанные с управлением \textit{знаниями}, экономикой \textit{знаний}, с \textit{семантической совместимостью интеллектуальных компьютерных систем}.\\
Основной особенностью \textit{Технологии OSTIS} является ориентация на использование компьютеров нового поколения, специально предназначенных для  реализации семантически совместимых гибридных \textit{интеллектуальных компьютерных систем}. Предлагаемая \textit{Документация Технологии OSTIS} оформлена в виде \textit{раздела базы знаний} специальной интеллектуальной компьютерной \textit{Метасистемы IMS.ostis} (Intelligent MetaSystem for ostis-systems), которая построена по Технологии OSTIS и представляет собой постоянно совершенствуемый интеллектуальный \textit{портал научно-технических знаний}, который поддерживает перманентную эволюцию \textit{Документации Технологии OSTIS}, а также разработку различных \textit{ostis-систем} (интеллектуальных компьютерных систем, построенных по \textit{Технологии OSTIS}).}
\scnidtf{Процесс перманентной эволюции \textit{Стандарта OSTIS}, совмещенного (интегрированного) с комплексными учебно-методическим обеспечением подготовки специалистов в области Искуссственного интеллекта и представленного в виде специального раздела базы знаний}
\scnnote{Подчеркнем, что \textit{Стандарт OSTIS} -- это не описание некоторого состояния \textit{Технологии OSTIS}, а \uline{динамическая} информационная модель процесса эволюции этой технологии}
\scnidtf{Стандарт Технологии OSTIS}
\scnidtf{Документация \textit{Технологии OSTIS}, полностью отражающая \uline{текущее} состояние \textit{Технологии OSTIS} и представленная соответствующим \textit{разделом базы знаний} специальной \textit{ostis-системы}, которая ориентирована на поддержку проектирования, производства, эксплуатации и эволюции (реинжиниринга) \textit{ostis-систем}, а также на поддержку эволюции самой \textit{Технологии OSTIS} и которая названа нами \textit{Метасистемой IMS.ostis}}
\scnidtf{Максимальный раздел \textit{Стандарта OSTIS}, т.е. раздел, в состав которого входят все остальные \textit{разделы} (подразделы) \textit{Стандарта OSTIS}}
\scnidtf{Раздел базы знаний, текущее состояние которого отражает текущее состояние (текущую версию) перманентно эволюционируемого Стандарта Комплексной Технологии OSTIS}
\scnidtf{Представленное в форме раздела базы знаний специальной ostis-системы (Метасистемы IMS.ostis) полное описание (спецификация, документация) текущего состояния Технологии OSTIS}
\scnidtf{Мы рассматриваем \textit{Стандарт OSTIS} (Документацию Стандарта Технологии OSTIS) как продукт \textit{научно-технической деятельности}, к которому предъявляются \uline{высокие требования} по полноте, согласованности, непротиворечивости, практической значимости разрабатываемой документации, описывающей текущее состояние \textit{Технологии OSTIS}}

\scnheader{официальная версия Стандарта OSTIS}
\scnidtf{официально издаваемая (публикуемая в бумажном и/или электронном виде) \textit{версия Стандарта OSTIS}}
\scnhaselement{Стандарт OSTIS-2021}

\input{Contents/beginning_fragment4_5}

\scnheader{Публикация Документации Технологии OSTIS-2021}
\scnrelto{публикация ostis-документации}{Документация Технологии OSTIS}
\scnidtf{Предлагаемое Вашему вниманию издание внешнего представления \textit{раздела базы знаний} \scnbigspace \textit{Метасистемы IMS.ostis}, посвященного комплексному описанию \textit{Технологии OSTIS} и отражающего версию указанного раздела, соответствующую весеннему периоду 2021 года}

\bigskip

\scnaddlevel{1}
\scnheaderlocal{публикация ostis-документации*}
\scnidtfexp{Бинарное ориентированное \textit{отношение}, каждая \textit{пара} которого связывает знак некоторого \textit{раздела базы знаний} со знаком \textit{файла}, который является внешним представлением указанного раздела, а также является либо копией электронной публикации материалов этого раздела, либо оригинал-макетом бумажной публикации указанных материалов*}
\scnaddlevel{-1}

\scnheader{Публикация Документации Технологии OSTIS-2021}
\scnidtf{Издание Документации Технологии OSTIS-2021}
\scnidtf{Первое издание (публикация) Внешнего представления Документации Технологии OSTIS в виде книги}
\scniselement{публикация}
	\scnaddlevel{1}
	\scnidtf{Официальная \textit{версия Стандарта OSTIS}, издаваемая непостредственно перед началом \textit{Конференции OSTIS-2021}}
	\scnaddlevel{-1}

\scnheader{Стандарт OSTIS}
\scnrelto{формальная спецификация}{Технология OSTIS}
	\scnaddlevel{1}
	\scnidtf{Перманентно развиваемый в рамках открытого проекта комплекс моделей, методов и средств, ориентированных на онтологическое проектирование, производство, экплуатацию и реинжиниринг семантически совместимых гибридных интеллектуальных компьютерных систем, способных самостоятельно взаимодействовать друг с другом}
	\scnidtf{Технология разработки семантически совместимых и самостоятельно взаимодействующих интеллектуальных компьютерных систем}
	\scnexplanation{\textit{Технология OSTIS} -- это технология принципиально нового уровня, это обусловлено:
		\begin{scnitemize}
		\item высоким качеством интеллектуальных компьютерных систем (ostis-систем), разрабатываемых на ее основе -- их семантической совместимостью, способностью к самостоятельному взаимодействию, способностью адаптироваться к пользователям и способностью адаптировать (обучать) самих пользователей более эффективному взаимодействию с интеллектальными компьютерными системами;
		\item высоким качеством самой \textit{технологии} -- возможностью интегрировать самые различные \textit{виды знаний} и самые различные \textit{модели решения задач}, неразрывной связью процесса разработки интеллектуальных компьютерных систем и процесса повышения квалификации разработчиков.
		\end{scnitemize}}
	\scnaddlevel{-1}
\bigskip
\input{Contents/chapter0/key_signs}

\input{Contents/beginning_6}

\input{Contents/beginning_8}

\newpage

\scnstructheader{Оглавление Стандарта OSTIS-2021}
\scnstartfile

%\vspace{-3\baselineskip}

\end{SCn}

%\DeactivateBG

\normalsize 

\begingroup
\let\clearpage\relax
\tableofcontents
\endgroup

\begin{SCn}
\scnendfile \scninlinesourcecommentpar{Завершили \textit{Оглавление Стандарта OSTIS-2021}}
\end{SCn}

\newpage
%\ActivateBG

\begin{SCn}

\scnnote{Подчеркнем, что в \textit{Оглавлении Публикации Документации Технологии OSTIS-2021} отсутствуют номера разделов.\\
 Это обусловлено тем, что количество и порядок разделов, а также номера их страниц актуальны только для текущего состояния данного внешнего текста базы знаний. База знаний эволюционирует независимо от вводимых в неё знаний, которые в общем случае разрабатываются разными авторами и могут иметь разный объем. В результате такой эволюции появляются новые разделы базы знаний, некоторые разделы могут \scnqq{переместиться} и, соответственно, поменять приписываемый им номер.\\
Это обусловлено эволюцией самой базы знаний, а также тем, какой раздел базы знаний отображается в виде внешнего текста}

\bigskip
\scnaddlevel{1}
\scnheaderlocal{следует отличать*}
\scnhaselementset{Документация Технологии OSTIS;публикация Документации Технологии OSTIS;файл Документации Технологии OSTIS
\scnaddlevel{1}
    \scnidtf{файл соответствующей версии Документации Технологии OSTIS}
    \scnidtf{файл соответствующей версии стандарта Технологии OSTIS}
\scnaddlevel{-1}
}
\scnaddlevel{-1}

\input{Contents/beginning_fragment3}
\end{SCn}
